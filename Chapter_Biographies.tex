To be informed of Nobel laureates (physics, chemistry, economics), Fields Medal... click this link {\href{http://www.nobelprize.org/}{{\color{blue}Nobel Prizes}}} or on this one {\href{http://www.fields.utoronto.ca/aboutus/jcfields/fields_medal.html}{{\color{blue}Fields Medal}}}

This section contains a list of some humans who have a strange reputation. Under the rules of history that is taught in elementary school, they do not exist, they haven't ordered any army, they sent nobody to death, they did not have any empire and they had only a minor part in major historical decisions. Some have acquired some celebrity, but none was ever a national hero. Yet their work has more influenced on the course of history than many acts by statesmen crowned with a far greater glory. They produced more turmoil than the comings and goings of armies in battle over borders, they have done more for the happiness or unhappiness that the edicts of kings and assemblies, because their work, is to have shaped the mind of man!

Whoever diffuse his ideas, has a power much greater than that of the sword or sceptre: this is shy why they have also shaped and directed the world. For the most part they have not lifted any finger to act physically, they worked mainly as intellectuals, in silence and oblivion, without worrying of the surrounding world. But in their wake, empires have crumbled, political regimes have either strengthened or eroded, the classes were pitted against each other, and also did nations. Not under the influence of a dark conspiracy, but by the extraordinary power of their ideas. Who are these humans?: Scientists, economists, chemists, biologists, mathematicians, physicists, computer scientists, engineers, ...

The biographies below of the most famous scientists around the world and cited in the various chapters of this book are sorted alphabetically and almost all texts are simplified copy/paste of the French {\href{http://www.wikipedia.fr}{{\color{blue}Wikipedia}}}. If you want us to add an entry, simply  {\href{mailto:isoz@sciences.ch}{{\color{blue}email}}} us the full name of the concerned person and why you would like to see us include him in the list below. We then study the proposal and take the appropriate decision.

\textbf{We also pay tribute to the millions of scientists\footnote{It seems that at the start of this third millennium there are not far from 7 million high-level scientists around the World according to UNESCO statistics.}, engineers, philosophers, craftsmen, artists, known and anonymous amateurs whose collaboration enabled through millennia the evolution of science and of the human condition!}

The sizes of the biographies are not proportional to the number of articles published or to the discoveries made, but on the amount of information found on the Internet or in the literature. The list is also not exhaustive, but its purpose is to honour and remember the great humans who made of pure sciences what they are today and who have spent part or whole of their life to science: most constrained art.

	\begin{tcolorbox}[colback=red!5,borderline={1mm}{2mm}{red!5},arc=0mm,boxrule=0pt]
	\bcbombe Caution! In physics (as well in mathematics) a theory, an equation or even a constant rarely wears the name of its true inventor. This is widely known among scientists and is often a source of joke from the community.
	\end{tcolorbox}

	\begin{figure}[H]
		\centering
		\includegraphics[scale=0.5]{img/shoulders_of_giants.jpg}	
	\end{figure}

\begin{center}
\hyperref[sec:A]{A} \hyperref[sec:B]{B} \hyperref[sec:C]{C} \hyperref[sec:D]{D} \hyperref[sec:E]{E} \hyperref[sec:F]{F} \hyperref[sec:G]{G} \hyperref[sec:H]{H} \hyperref[sec:I]{I} \hyperref[sec:J]{J} \hyperref[sec:K]{K} \hyperref[sec:L]{L} \hyperref[sec:M]{M} \hyperref[sec:N]{N} \hyperref[sec:O]{O} \hyperref[sec:P]{P} \hyperref[sec:Q]{Q} \hyperref[sec:R]{R} \hyperref[sec:S]{S} \hyperref[sec:T]{T} \hyperref[sec:U]{U} \hyperref[sec:V]{V} \hyperref[sec:W]{W} \hyperref[sec:X]{X} \hyperref[sec:Y]{Y} \hyperref[sec:Z]{Z}
\end{center}

\phantomsection
\addcontentsline{toc}{section}{A}
\label{sec:A}
		
\pichskip{15pt}% Horizontal gap between picture and text
\parpic[l][t]{
  \begin{minipage}{40mm}
    \fbox{\includegraphics[width=110px,height=140px]{img/medaillons/al.eps}}
  \end{minipage}
}		
\textbf{Al-Biruni, Muhammad Ibn Ahmad Abul-Rayhan} (973-1048) is a mathematician, astronomer, physicist, encyclopedist, philosopher, astrologer, traveller, historian, pharmacologist and a tutor, native of Persia, who contributed greatly to the fields of mathematics, philosophy, medicine and science. He is known for his theory on the Earth's rotation around its axis and around the Sun, and this long before Copernicus. He focused particularly on the calculation of the Sun running (apogee) and also corrected some data of Ptolemy. Excellent mathematician, Al-Biruni developed new equations unknown to his predecessors. He calculated also the local meridian and the coordinates of some localities. But the picture would not be complete if we forgot to mention that six centuries before Galileo, Al Biruni already put forward an Earth that revolved around its axis. With the help of an astrolabe, the sea and a nearby mountain, he calculated the circumference of the Earth by solving a complex equation for its time. The main contribution of Al-Biruni to mathematics lies in its work in trigonometry (calculations of some trigonometric functions values that were not well defined at this time).

\pichskip{15pt}% Horizontal gap between picture and text
\parpic[l][t]{
  \begin{minipage}{40mm}
    \fbox{\includegraphics[width=110px,height=140px]{img/medaillons/alkindi.jpg}}
  \end{minipage}
}		
\textbf{Al-Kindi, Abu Yūsuf Ya'qūb ibn} (801-873)  born in Kufa and educated in Baghdad is a philosopher, polymath, mathematician, physician and musician. Al-Kindi was the first of the  peripatetic philosophers, and is unanimously hailed as the "father of Arab philosophy" for his synthesis, adaptation and promotion of Greek and Hellenistic philosophy in the Muslim world. In the field of mathematics, Al-Kindi played an important role in introducing Indian numerals to the Islamic and Christian world. Al-Kindi was also one of the fathers of cryptography. His book entitled Manuscript on Deciphering Cryptographic Messages gave rise to the birth of cryptanalysis, was the earliest known use of statistical inference, and introduced several new methods of breaking ciphers. Using his mathematical and medical expertise, he was able to develop a scale that would allow doctors to quantify the potency of their medication. According to Ibn Al-Nadim, Al-Kindi wrote at least two hundred and sixty books, contributing heavily to geometry (thirty-two books), medicine and philosophy (twenty-two books each), logic (nine books), and physics (twelve books).

\pichskip{15pt}% Horizontal gap between picture and text
\parpic[l][t]{%
  \begin{minipage}{40mm}
    \fbox{\includegraphics[width=110px,height=140px]{img/medaillons/alembert.eps}}
  \end{minipage}
}
\textbf{Alembert, Jean le Rond} (1717-1783), child of a commissioner of artillery, abandoned on the steps of the chapel of Paris Saint-Jean-Le-Rond, the future great philosopher, mathematician and physicist is adopted by a glazier who secretly receive a pension to support the education of the young boy who brilliantly study law, medicine and mathematics. Following the publication of several memoirs (on integral calculus and refraction of solids), d'Alembert entered the Academy of Sciences (1741). He is at the origin of the famous momentum principle, named "D'Alembert's principle" in his \textit{ Traité de dynamique} (1743). In astronomy, he is the author of a treaty on the precession of the equinoxes (1749) explained by using the Newton theory of universal gravitation and with a partial solution to the three bodies problem. D'Alembert also establishes a mathematical theory of vibrating strings by studying the nature of the sound (harmonics).

\pichskip{15pt}% Horizontal gap between picture and text
\parpic[l][t]{%
  \begin{minipage}{40mm}
    \fbox{\includegraphics[width=110px,height=140px]{img/medaillons/ampere.eps}}
  \end{minipage}
}
\textbf{Ampère, André Marie} (1775-1836) at 18 years, he already knows most of the mathematical works of his time. First-class mathematician, he shows how we must use this science, that he was considering as a branch of philosophy, to the study of physical facts to give a definitive relation. Within a few weeks, Ampere gives the foundation to a science to which he gives the name of "Electromagnetism". He tries to understand the magnetism of magnets and draws a hypothesis of "particulate flows" (today: electronic orbits spin orientation). It also equal the number of molecules in equal volumes of gases of different nature, but measured under identical conditions of temperature and pressure (experimental observation of Gay-Lussac).

\pichskip{15pt}% Horizontal gap between picture and text
\parpic[l][t]{%
  \begin{minipage}{40mm}
    \fbox{\includegraphics[width=110px,height=140px]{img/medaillons/archimede.eps}}
  \end{minipage}
}
\textbf{Archimedes of Syracuse} (287-212 BC.), is a famous Greek mathematician and engineer as both a theorist and as a mechanic machine manufacturer. Archimedes had an exceptional mathematical production, part of which was received in treaties such as \textit{On the Sphere and Cylinder}, \textit{Measurement of the circle; Quadrature of the parabola, spiral and the conoid and spheroids}; \textit{The Method of Bodies floating}... This is from his mechanical work that the main legends starts, like the lever or the bath, will be. The famous maxim: « Give me a place to stand and I will move the earth» is an echo of the popular Archimedean contribution to the static in the treaty of Equilibria. Archimedes proves the law of the lever, introduces the basic concept of center of gravity, and determines the centroids for the main plane geometric figures. It is the same for the story of Archimedes springing naked from his bath, crying  «Eureka», because he came, following the legend, to solve the problem posed to him by King Hiero. In fact, the story is a spectacular staging of the discovery of the fundamental principle of hydrostatics (commonly named "Archimedes principle"). In geometry, the work of Archimedes develops that of Eudoxus of Cnidus as we know it by the book XII of Euclid's Elements: this is to compare measurements of plane figures and solids, in particular from curvilinear figures. Thus, Archimedes proved that the volume of the circumscribing cylinder of a sphere is equal to one and half times it's volume and that the side surface of the cylinder is equal to that of the sphere or four times the surface of a great circle. So if we can calculate the area of the circle, we know that of the sphere, cylinder, its volume and that of the sphere, etc. His most famous result and easiest is for the circle. Archimedes brings it's quadrature to another problem: the correction of its circumference, that is to say "find a equal straight line to it is equal", it solves the problem using a geometric curve that is now named "Archimedean spiral". In addition, it calculates the approximate values of the circumference/diameter ratio (what we name the number "Pi" noted $\pi$).

\pichskip{15pt}% Horizontal gap between picture and text
\parpic[l][t]{%
  \begin{minipage}{40mm}
    \fbox{\includegraphics[width=110px,height=140px]{img/medaillons/avogadro.eps}}
  \end{minipage}
}
\textbf{Avogadro, Amedeo} (1776-1856), son of a magistrate of Turin, Amedeo begins his life by following his father path. He pass a law degree in 1795 and practice in his hometown. But his love for physics and mathematics, which he studied alone, drives him to start on the late scientific studies. In 1809 he presents a paper to the Royal Academy of Turin; its success allows him to get a professorship at the Royal College of Vercelli. In 1820, the University of Turin created for him a chair of physics that he will keep until the end of his life. By studying the laws governing the compression and expansion of gases Avogadro states, in 1811 the hypothesis famously known as "Avogadro's law". Based on the atomic theory of Dalton's and Gay-Lussac law on the volume ratios, Avogadro's theory indicates that two equal volumes of different gases, under the same conditions of temperature and pressure, contain the same number of molecules. Under it's seeming simplicity, this law has important implications, because of it, it becomes possible to determine the molar mass of a gas from one another. But the chemists at this time, more interested in experiences, are not attentive to the theoretical studies of Avogadro who will also be recognized only 50 years later. The name Avogadro also remains linked to that of "Avogadro's number" indicating the number of molecules in one mole.

\phantomsection
\addcontentsline{toc}{section}{B}	
\label{sec:B}

\pichskip{15pt}% Horizontal gap between picture and text
\parpic[l][t]{%
  \begin{minipage}{40mm}
    \fbox{\includegraphics[width=110px,height=140px]{img/medaillons/bachelier.jpg}}
  \end{minipage}
}
\textbf{Bachelier, Louis} (1870-1946) was born in Le Havre in a family of merchants. He appears at his majority on the electoral lists of Le Havre in 1892 as a sales representative at the same business address as his father. After completing his military service at age 22, he resumed his studies at the Faculty of Sciences in Paris. He obtained a Bachelor of Science in 1895 (passing grade) and a PhD in 1900 with his famous and unknown subject in mathematics... Although his theory is now considered as a pioneering work in probability and financial theory. From 1913 to 1914 Bachelor teaches probability theory applied to mechanics, ballistics, and biometrics. He was also responsible for additional conferences on general mathematics from 1913 to 1914. It is only after the 1914-1918 war that he obtained a first post of lecturer at the Faculty of Besançon. After several replacements in Dijon and Rennes, he returned to Besançon in 1927 as Chair Professor of calculus, a position he held until his retirement in 1937. Louis Bachelier, among his numerous works, was the first to have introduced the continuity in probabilities problems taking time as a variable. In particular, he developed a mathematical theory of Brownian motion five years before Albert Einstein. He is also, well before Norbert Wiener, the first to have defined the function of Brownian motion and gave many of its properties.

\parpic[l][t]{%
  \begin{minipage}{40mm}
    \fbox{\includegraphics[width=110px,height=140px]{img/medaillons/balmer.eps}}
  \end{minipage}
}
\textbf{Balmer, Johann Jakob} (1825-1898) was a Swiss mathematician and mathematical physicist born in Lausanne (Switzerland) and who died in Basel. During his schooling he excelled in mathematics, and so decided to focus on that field when he attended university. He studied at the University of Karlsruhe and the University of Berlin, then completed his PhD from the University of Basel in 1849 with a dissertation on the cycloid. Johann then spent his entire life in Basel, where he taught at a school for girls. He also lectured at the University of Basel. Despite being a mathematician, he is not remembered for any work in that field; rather, his major contribution (made at the age of sixty, in 1885) was an empirical formula for the visible spectral lines of the hydrogen atom, the study of which he took up at the suggestion of Eduard Hagenbach also of Basel. A full explanation of his formula worked, however, had to wait until the presentation of the Bohr model of the atom by Niels Bohr in 1913.

\parpic[l][t]{%
  \begin{minipage}{40mm}
    \fbox{\includegraphics[width=110px,height=140px]{img/medaillons/banach.eps}}
  \end{minipage}
}
\textbf{Banach, Stefan} (1892-1945) was a Polish mathematician who define the foundations of functional analysis. Born in Krakow in 1892, Austria-Hungary (now Polish city). Banach went to high school in Krakow, where he revealed to be particularly brilliant in mathematics and natural sciences, but his disinterest in other matters prevented him from obtaining the best evaluation. Banach's life (at least mathematically) will switch in the spring of 1916, when he meets Steinhaus in Krakow. With Otto Nikodym, they decided to find a mathematical society. Banach's mathematical research begins at this moment. His first article was co-authored with Steinhaus. Steinhaus told him about a property he could not be able to prove, and after some days of reflection, Banach exhibited a cons-example. It is difficult to say what would have happened to Banach's mathematical activity without the meeting with Steinhaus, but the fact remains that he began only after it's intense and fruitful researches. Banach returns in Lvov in 1920 as an assistant. He submitted his thesis in 1922, and it is in this thesis that appears for the first time the notion of Banach space and where the fundamentals theorems about these objects are prove and also and also where there is a discussion on weak topology... . In short, this thesis marks the birth of functional analysis. In 1929, he founded with Steinhaus the magazine \textit{Studia Math}, dedicated to the development of functional analysis, and in 1939 he was elected president of the Mathematical Society of Poland. In 1945, shortly before the end of World War II, he died of a long cancer. Many theorems are associated with the name Banach, that he has demonstrated himself, or that they refer to it's ideas. These include: the theorem of Hahn-Banach about the extension of continuous linear forms, the theorem of Banach-Steinhaus, Banach-Alaoglu, the Banach fixed point theorem of course the Banach-Tarski paradox.

\parpic[l][t]{%
  \begin{minipage}{40mm}
    \fbox{\includegraphics[width=110px,height=140px]{img/medaillons/bell.eps}}
  \end{minipage}
}
\textbf{Bell, John} (1928-1990) was from earliest childhood attracted to books about science. Because of family financial problems, he could not immediately follow academic studies. He then worked for a year as a technician in the physics department of Queen's University in Belfast before becoming a student in 1945 in the same department. He went out first rank in his class in math-physics. Bell founded in the years 1960 a new inspiration in the foundations of quantum theory, an area supposedly exhausted by the results of the Bohr-Einstein debate thirty years earlier, and ignored by almost everyone who used the quantum theory-between time. Indeed, Bell was intrigued by the Heisenberg quantum uncertainty and wanted to delve deeper by showing that the discussion of concepts such as "realism", "determinism" and "locality" could be affiliated in a rigorous mathematical relation: "the Bell inequalities" experimentally verifiable. Bell pushed very far the doubts he had on the principles of uncertainty to the point that even irritated his teacher (Sloane) who told him that now he was going too far! Bell waited for his thesis to develop his ideas. Unfortunately, because of financial problems again, he had to delay his research and join the Harwell atomic research center. During his career, he married a woman (Mary Bell) who helped him in the development of its work on the fundamentals of quantum theory. It is in 1951, with Rudolf Peierls, that Bell developed his famous CPT theory (Charge, Parity, Time). Unfortunately for Bell, Gerhard Lüders and Wolfgang Pauli came to the same result in the same period and it is to them that were awarded this discovery. The theoretical developments of Bell are at the origin of cryptography and quantum information theory. Another major work of Bell in 1969 was the participation in the development of "the A.B.J. anomaly" (Adler-Bell-Jackiw) in quantum field theory. This three physicists showed that the standard algebraic model merely an error. Indeed, quantification of the fields model broke a symmetry. Bell was nominated for the Nobel Prize, that he certainly would have obtained if he had not died in 1990.

\parpic[l][t]{%
  \begin{minipage}{40mm}
    \fbox{\includegraphics[width=110px,height=140px]{img/medaillons/berners_lee_timothy_john.jpg}}
  \end{minipage}
}
\textbf{Berners-Lee, Timothy John} (1955-) is an English physicist best known as the inventor of the World Wide Web. He is the director of the World Wide Web Consortium (W3C), which oversees the continued development of the Web. He is also the founder of the World Wide Web Foundation, and is a senior researcher and holder of the founders chair at the MIT Computer Science and Artificial Intelligence Laboratory. He is a director of the Web Science Research Initiative (WSRI), and a member of the advisory board of the MIT Center for Collective Intelligence. He worked as an independent contractor at CERN from June to December 1980. While in Geneva, he proposed a project based on the concept of hypertext, to facilitate sharing and updating information among researchers. The address info.cern.ch was the address of the world's first-ever web site and web server, running on a NeXT computer at CERN. The first web page address was http://info.cern.ch/hypertext/WWW/TheProject.htm.

\parpic[l][t]{%
  \begin{minipage}{40mm}
    \fbox{\includegraphics[width=110px,height=140px]{img/medaillons/bernoulli_daniel.eps}}
  \end{minipage}
}
\textbf{Bernoulli, Daniel} (1700-1782) was a Swiss scientist who discovered the basic principles of behaviour of a fluid (it is also the son of Johann Bernoulli and the nephew of Jacques Bernoulli). He cultivated both mathematics and natural sciences, taught mathematics, anatomy, botany and physics. Friend of Leonhard Euler, he worked with him in several areas of mathematics and physics (he shared with him ten times the annual prize of the Academy of Sciences of Paris), he made this prize a kind of income. The various problems he tried to solve (elasticity theory, mechanism of tides) led him to focus and develop mathematical tools such as differential equations or series. He also collaborated with Jean le Rond d'Alembert in the study of vibrating strings. He studied the flow of fluids (1738) and formulated the principle (the famous Bernoulli theorem) that the pressure exerted by a fluid is inversely proportional to its velocity. He used atomistic concepts to outline the first kinetic theory of gases, expressing their behaviour in terms of probabilities under the particular conditions of pressure and temperature. He can be regarded as a founder of hydrodynamics.

\parpic[l][t]{%
  \begin{minipage}{40mm}
    \fbox{\includegraphics[width=110px,height=140px]{img/medaillons/bernoulli_jacques.eps}}
  \end{minipage}
}
\textbf{Bernoulli, Jacques} (1654-1705), was a Swiss mathematician and physicist, brother of Jean Bernoulli and Daniel Bernoulli and the uncle of Nicolas Bernoulli. Born in Basel in 1654, he met Robert Boyle and Robert Hooke on a trip to England in 1676. After that, he devoted himself to physics and mathematics. He teaches at the University of Basel from 1682, becoming professor of mathematics in 1687. He earned by his work and discoveries to be made a member of the Academy of Sciences in Paris (1699) and that in Berlin (1701). His correspondence with Gottfried Wilhelm Leibniz leads him to study infinitesimal calculus in collaboration with his brother Jean. He was among the first to understand and apply the integral and differential calculus, proposed by Leibniz, discovered the properties of numbers named "Bernoulli numbers" and gave the solution of problems considered at his time as insoluble. He sets out the principles of probability theory and introduced the Bernoulli numbers in a book published after his death in 1713.

\parpic[l][t]{%
  \begin{minipage}{40mm}
    \fbox{\includegraphics[width=110px,height=140px]{img/medaillons/bernoulli_jean.eps}}
  \end{minipage}
}
\textbf{Bernoulli, Jean} (1667-1748), was a Swiss mathematician and physicist. Jacques Bernoulli's brother and father of Daniel and Nicolas Bernoulli. He taught mathematics at Groningen (1695), then in Basel, after the death of Jacques Bernoulli (1705), and became a member of the Academies of Sciences in Paris, London, Berlin and St. Petersburg. Trained by his brother Jacques Bernoulli, he had long worked with him to develop the implications of the new infinitesimal calculus invented by Gottfried Leibniz, but he then appears between them on the occasion of solving some problems, a rivalry which degenerated into enmity. He also contributed in many areas of mathematics including the problem of a particle moving in a gravitational field. He found the equation of the chain in 1690 and developed the exponential calculation in 1691. He also had the honour to train Leonhard Euler. He came to Paris in 1690 and became intimate with the most distinguished scholars, especially with Hospital. Jean Bernoulli became a member of the Royal Society in 1712.

\parpic[l][t]{%
  \begin{minipage}{40mm}
    \fbox{\includegraphics[width=110px,height=140px]{img/medaillons/bessel.eps}}
  \end{minipage}
}
\textbf{Bessel, Friedrich} (1784-1846) Born in Minden, Westphalia, Bessel began working very young as a clerk. Attracted by shipping, he became interested in nautical observations, constructing his own sextant and studying astronomy in his free time. He calculated the trajectory of Halley's comet, a result which was immediately released and allowed him to obtain, in 1806, a job as assistant at the Lilienthal Observatory. In 1810 he became director of the new observatory in Königsberg, while pursuing mathematical studies. He had to teach mathematics to his students in astronomy until 1825 (when Jacobi came to teach the subject in Königsberg). His whole life was devoted to astronomy (he wrote over 350 articles) and, shortly before his death, he began the study the motion of Uranus, the problem that led to the discovery of Neptune. In mathematics, Bessel is known for introducing the functions that have his name, used for the first time in 1817  for the study of a Kepler problem, and employing more fully seven years later to study planetary perturbations.

\parpic[l][t]{%
  \begin{minipage}{40mm}
    \fbox{\includegraphics[width=110px,height=140px]{img/medaillons/biot.eps}}
  \end{minipage}
}
\textbf{Biot, Jean-Baptiste} (1774-1862) born and died in Paris was a physicist, astronomer and mathematician. Jean-Baptiste followed secondary education (humanities) in Paris at the Collège Louis-le-Grand until 1791. He began studying engineering at the École des Ponts et Chaussées in January 1794, then joined the École Central des Travaux Publiques (later Polytechnic) when it opened in December 1794 at the Palais Bourbon. One year later (1795) he joined the École des Ponts et Chaussées to complete his training as an engineer. It is to teaching that Biot oriented his career after studying engineering. He became professor of mathematics at the École Central de l'Oise in Beauvais in 1797. With the support of Laplace he was appointed in 1800, aged 26, as professor of mathematical physics at the Collège de France. He is between 1816 and 1826 responsible at 50\% of the trainings of physical acoustics, magnetism and optics, Gay-Lussac, having the Chair of Physics, teaches heat, gas, humidity, the electricity and the galvanism. He formulated with Félix Savart, the Biot-Savart law, which gives the value of the magnetic field produced at a point in space by an electric current as a function of distance from this point to the conductor.

\parpic[l][t]{%
  \begin{minipage}{40mm}
    \fbox{\includegraphics[width=110px,height=140px]{img/medaillons/bohr.eps}}
  \end{minipage}
}
\textbf{Bohr, Niels Henrik David} (1885-1962) was a Danish physicist, Nobel Prize of Physics in 1922 for his contributions to nuclear physics and the understanding of atomic structure. The Bohr theory of atomic structure, for which he received the Nobel Prize, was published between 1913 and 1915. His work was inspired by the nuclear model of the Rutherford atom, in which the atom is considered as a compact nucleus surrounded by a cloud of electrons. The model suppose that the atom emits electromagnetic radiation when an electron moves from one quantum level to another. This model contributed enormously to future developments of theoretical atomic physics.

\parpic[l][t]{%
  \begin{minipage}{40mm}
    \fbox{\includegraphics[width=110px,height=140px]{img/medaillons/boltzmann.eps}}
  \end{minipage}
}
\textbf{Boltzmann, Ludwig} (1844-1906) was an Austrian physicist who helped to establish the foundations of statistical mechanics. Educated in Vienna and Oxford, he taught physics at several universities in Germany and Austria for over forty years. Developing the kinetic theory of gases, especially from the work of Maxwell, it establishes that the second law of thermodynamics could be obtained on the basis of statistical analysis. Calculating the number of particles with a given energy, he established the so-named "Maxwell-Boltzmann statistical". He expressed the entropy S of a system according to the probability W of his state (through his famous equation of transport from which he showed that entropy could only increase over time ... result which was previously recognized experimentally but not theoretically proved). It could also establish theoretically the "Stefan's law" concerning the radiation of a black-body. But he had to explain how the mechanics principles, where the phenomena are reversible, could lead to thermodynamic laws describing phenomena characterized by irreversibility. He advanced the idea that irreversible changes, although they are only possibilities among others, are so likely that they are almost always occurring.

\parpic[l][t]{%
  \begin{minipage}{40mm}
    \fbox{\includegraphics[width=110px,height=140px]{img/medaillons/boole.eps}}
  \end{minipage}
}
\textbf{Boole, George} (1815-1864) was an English mathematician and logician and the creator of symbolic logic. Born in Lincoln, and son of a shopkeeper, he received his first lessons in mathematics from his father, who also taught him to manufacture optical instruments. Outside advice from his father and several years in local schools, Boole is a self-taught. When his father's business declined, he was obliged to work to help his family and, when sixteen, he taught in village schools, at twenty, he opened his own school in Lincoln. During his hobbies he studied mathematics at the Institute of Mechanics, created around this time, that's where he became acquainted with Newton's Principia, Laplace's celestial mechanics and analytical mechanics of Lagrange and he began to solve problems of higher algebra. Boole submitted to the new Cambridge Mathematical Journal a series of original articles, the first being \textit{Searches on the theory of analytical transformations}, these articles focused on differential equations and the invariant linear transformation. In 1844, he studied the links between algebra and infinitesimal calculus in an important paper published in the Transactions of the Royal Society, which awarded him a medal for his contribution to the analysis (that is, the use of algebra in the study of infinitely small and large entities). Developing new ideas about the method in logic and confident in the symbolism he had created from his mathematical research, he published in 1847, a booklet, \textit{Mathematical Analysis of Logic}, in which he argues that the logic must be attached mathematics, not philosophy. Even he had no university degree, Boole was, on the basis of its publications, in 1849 appointed professor at Queen's College in Cork, Ireland. With Boole, in 1847 and in 1854 began the algebra of logic, that is to say, what we name today the "Boolean Algebra". In his book of 1854, Boole states its completely new symbolic method of logical inference, which allows with proposals containing a number of terms, to obtain, by symbolic processing of the premisses, conclusions which were logically contained in the premises. He also search a general method in probability, which would, from the known probabilities of a given event, determine the probability of any other event logically connected to specific events.

\parpic[l][t]{%
  \begin{minipage}{40mm}
    \fbox{\includegraphics[width=110px,height=140px]{img/medaillons/borel.eps}}
  \end{minipage}
}
\textbf{Borel, Emile} (1871-1956) received major at X and ULM, he chooses the last one and dedicated his time to mathematics. He founded the Institute Henri Poincare and was elected mayor of the Aveyron and Saint-Africa. He studies the measures of sets and in particular, defines the sets of measure zero and all Borel sets on which we can define a measure. He then turns to probability and mathematical physics. Borel is also considered a constructivist mathematician. He is at the origin of strategic game theory and cybernetics that will develop later von Neumann and Morgenstern. His pupil Henri Lebesgue use his results in topology and measure theory for his theory of integration.

\parpic[l][t]{%
  \begin{minipage}{40mm}
    \fbox{\includegraphics[width=110px,height=140px]{img/medaillons/born.eps}}
  \end{minipage}
}
\textbf{Born, Max} (1882-1970) born in Breslau and died in Göttingen was a German and British physicist. Initially he followed his studies at the College of König-Wilhelm and continued at the University of Breslau followed by Heidelberg and Zürich Universities. While studying for his PhD he came in contact with mathematicians such as Klein, Hilbert, Minkowski, Runge, Schwarzschild. In 1921, he was appointed professor of theoretical physics at Göttingen. He emigrated to Scotland in 1933 and became a British citizen in 1939. Outstanding theoretical physicist, he is known for his significant contribution to quantum physics: Development (1925) of quantum matrix mechanics introduced by Werner Heisenberg and, most importantly, he will be the first to give to the square of the modulus of the wave function the meaning of a density of probability of presence. He was also a pioneer in the quantum theory of solids (conditions of Born-von Karmann) and non-linear electrodynamics of Born-Infeld. He has won half of the Nobel Prize for Physics in 1954 (the other half was given to Walther Bothe) for his fundamental research in quantum mechanics, especially for his statistical interpretation of the wave function. The Royal Society awarded him the Hughes Medal in 1950.

\parpic[l][t]{%
  \begin{minipage}{40mm}
    \fbox{\includegraphics[width=110px,height=140px]{img/medaillons/bose.eps}}
  \end{minipage}
}
\textbf{Bose, Satyendranath} (1894-1974) was an Indian mathematician and physicist, known for his contributions to quantum theory. Born in Calcutta, Bose was educated at Presidency College in Calcutta. In 1924, he offers a statistical description of quantum systems, echoed by Albert Einstein, which places no restrictions on the energy distribution of particles in the system. This description is known as the "Bose-Einstein statistics", as opposed to the "Fermi-Dirac statistics". Applied to the theory of black-body radiation, this new statistic leads to the "Planck distribution" and treats this radiation as a photon gas. In the field of elementary particle physics, the Bose-Einstein statistics requires the wave function of particles (in the Schrödinger equation) to be perfectly symmetrical for all the variables of space and spin. Particles obeying these statistics (photons, mesons $\pi$, etc.) are named "bosons". Professor of physics at the Universities of Calcutta and Dhaka, Satyendranath Bose was appointed in 1958 National Teacher of India.

\parpic[l][t]{%
  \begin{minipage}{40mm}
    \fbox{\includegraphics[width=110px,height=140px]{img/medaillons/broglie.eps}}
  \end{minipage}
}
\textbf{de Broglie, Louis Victor} (1892-1987) was a  French physicist and Nobel laureate who brought an essential contribution to the quantum theory with his studies of electromagnetic radiation. Born in Dieppe, Louis de Broglie was educated in Paris. He tried to understand the dual nature of matter and energy and suggested the association of a wave with any particle. He proposed also directly to explain how it was possible to obtain the quantization rules of Bohr and Sommerfeld atom's model requiring an integer number of waves in a stationary orbit. His discovery of the wave nature of electrons (1924) won him the Nobel Prize for Physics in 1929, however, he did not proposed a wave equation describing quantum phenomena (what Schrödinger will). He was elected to the Academy of Sciences in 1933 and at the Académie Française in 1943. He was appointed professor of theoretical physics at the Université de Paris (1928), Permanent Secretary of the Academy of Sciences (1942), and advisor to the Atomic Energy Commission (1945).

\parpic[l][t]{%
  \begin{minipage}{40mm}
    \fbox{\includegraphics[width=110px,height=140px]{img/medaillons/brouwer.eps}}
  \end{minipage}
}
\textbf{Brouwer, Luitzen Egbertus Jan} (1881-1966) was a great Dutch mathematician of the early 20th century. Born from a father who was teacher, he performed very fast at high school. At the University of Amsterdam, he was trained by Korteweg, who is known for contributions in Applied Mathematics. He presented his PhD in 1904. From 1909 to 1913, Brouwer is interested in topology, and discovered most of the theorems to which his name has remained attached, including his famous fixed point theorem. For many, Brouwer is the father of modern topology. In 1912 he obtained, through Hilbert referrals, a professorship at the University of Amsterdam. He teaches the theory of sets, of functions, and axiomatic. Later, he refused to join Hilbert in Göttingen. During the first World War his health embrittlement and he left some time the fields of scientific research. When he returned, it was to devote himself to his first love (his thesis was already on this subject): the foundations of mathematics. Brouwer is with Poincaré the spearheading of intuitionist mathematics, as opposed to the logicism of Frege and Russell, and Hilbert formalism. In particular, for Brouwer, an existence theorem can be true only if you can show a process, even formal, of construction. This led in particular to reject the law of excluded middle, which says that a property is either true or false! The proofs thus obtained are often longer, but Brouwer was able to rewrite treaties of set theory, theory of measurement and theory of functions in accordance with the rules of intuitionism. Oddly, Brouwer never taught topology. This is probably because the theorems that he had proven itself did not fit anymore in it axiomatic set. According to testimonies of some of his students, he was a really strange character, madly in love with his philosophy, and a teacher with which it was not recommended to ask questions!

\phantomsection
\addcontentsline{toc}{section}{C}
\label{sec:C}

\parpic[l][t]{%
  \begin{minipage}{40mm}
    \fbox{\includegraphics[width=110px,height=140px]{img/medaillons/cantor.eps}}
  \end{minipage}
}
\textbf{Cantor, Georg} (1845-1918) was a brilliant student, particularly in manual operations. Despite the injunctions of his father, who dreams of making him an engineer, he moved to Berlin in 1862 to study mathematics where his teachers are Weierstrass and Kronecker. He presented his PhD in 1867 (on number theory). The first post-doctoral research of Cantor are devoted to the decomposition of functions into sums of trigonometric series (the famous Fourier series) and especially to the uniqueness of this decomposition. To fully resolve this difficult problem, it was necessary to introduce and study sets named "exceptional sets". This led in 1872 to define precisely what a real number was as limit of a sequence of rational numbers, at the same time his friend Dedekind gives another definition of the straight of the real number using cuts. Cantor and Dedekind note on this occasion there's a lot more real numbers than rational, but there has not robust mathematical definition of this "much more". In 1874, in the prestigious Journal of Crelle, Cantor defines the number of elements of an infinite set which extends naturally that of the cardinal of an infinite set, which extends that of the cardinal of a finite set. It follows, until 1897, a succession of strange discoveries: there are as many even integers than any integers, as many points on a segment than in a square, many more transcendental numbers than rational numbers. This hierarchy of infinite sets gradually led Cantor to establish new numbers, transfinite ordinals, and define an arithmetic on these numbers. Cantor's works had a lot of influence in the 20th century. We have to mention, in 1903, a paradox raised by Russell in the naive set theory: if $A$ is the set of all sets that are not elements of themselves, is $A$ contained in $A$? Logicians will overcome this conceptual difficulty, without changing the conclusions of Cantor. We can refer to the problem of the continuum hypothesis. One of the last lines of research Cantor was to estimate the number of elements of the real line. Specifically, Cantor wanted to prove the absence of any set whose cardinality is strictly between the cardinal integers and the real numbers. This is what we name the "continuum hypothesis". All Cantor's work and this of his successors to prove or disprove the continuum hypothesis were unsuccessful, and for good reason: in 1963, the logician Cohen proved that in a standard theory of sets, the continuum hypothesis is undecidable. We can easily assume it is true or false without obtaining any conflict in the theory.

\parpic[l][t]{%
  \begin{minipage}{40mm}
    \fbox{\includegraphics[width=110px,height=140px]{img/medaillons/carnot.eps}}
  \end{minipage}
}
\textbf{Carnot, Nicolas Léonard Sadi} (1796-1832), physicist and French military engineer, considered as the creator of thermodynamics. Eldest son of Lazare Carnot, nicknamed "the Grand Carnot", Sadi was educated at the École Polytechnique. In 1824, he described his conception of the ideal heat engine, named "Carnot engine", in which all available energy is used. He discovered that heat could pass from a cold body to a hotter body, and the engine performance depended on the amount of heat ha was able to use. This discovery, or Carnot cycle, is the basis of the second law of thermodynamics.\\

\parpic[l][t]{%
  \begin{minipage}{40mm}
    \fbox{\includegraphics[width=110px,height=140px]{img/medaillons/cartan.eps}}
  \end{minipage}
}
\textbf{Cartan, Élie} (1869-1951) received his primary education at the School of Dolomieu, Vienna and then in the Lycée et Collège de Grenoble. He attended Jeanson-de-Sailly high school for the preparation at the École Normale Supérieure, where he entered in 1888. There he followed the teachings of H. Poincare, É. Picard and C. Hermite. The first work of Élie Cartan which would lead to his thesis in 1894, focuses on complex simple Lie groups, where he resumed, corrected and developed the results of structure and classification obtained by W. Killing. Cartan obtained a lectureship at the Université de Montpellier from 1894 to 1896, then at the Faculté de Lyon from 1896 to 1903. The same year he is appointed professor at the Faculté de Nancy, where he remained until 1909. He gives at the same time courses at the École d'Ingénierie Électrique et de Mécanique Appliquées. He wrote two major articles on a generalization in infinite dimensions of simple Lie groups. He develops methods that were to influence the further development of differential geometry. In 1909, he left Nancy to teach at La Sorbonne, where he is appointed professor in 1912. He also provides instruction in the École de Physique et Chimie de Paris. In 1914, he solves the problem of classification of real simple Lie groups, and determines the finite dimensional representations of these groups. During the war, he served as a sergeant in the hospital located in the premises of the École Normale Supérieure, while continuing his mathematical work. His subsequent mathematical work is considerable, with nearly 200 publications and several books. Topics covered include the study of Pfaffian systems, the deformation theory, the study of constant negative curvature varieties, the gravitational theory of Einstein's, the theory of affine connections, holonomy groups, the Riemannian symmetric spaces, spinors. He is also the author of several articles on the history of geometry. He retired in 1940.


\parpic[l][t]{%
  \begin{minipage}{40mm}
    \fbox{\includegraphics[width=110px,height=140px]{img/medaillons/cauchy.eps}}
  \end{minipage}
}
\textbf{Cauchy, Augustin-Louis} (1789-1857). It was at Cherbourg that Cauchy started his math researches on polyhedra, and it's first results are promising. But, tired by the cumulative charge of engineer and long evenings of research, Cauchy had a depression that pushes him to return to live with his parents. In Paris, he sought a position in line with its commitment to pure mathematical research. In 1815, he completed a brilliant memory where it shows a famous Fermat's theorem on polygonal numbers. This will do much for his reputation, and in 1816, he became a member of the Académie des Sciences, replacing Carnot and Monge. The course of analysis that Cauchy professes at the École Polytechnique is decried by both his students as colleagues from other matters. However, this course is published in 1821 and 1823, which was to become the reference for the analysis in the 19th century. highlighting the rigour, not just intuition. This is the first time that real definitions of limits, continuity, convergence of sequences, series, are used. This rigour, however, still remains relative, since Cauchy "proves" that the limit of a series of continuous functions is continuous, which is not true. It is true that Cauchy did not yet have a clear definition of real numbers. This is also the time where Cauchy deeply develops the analysis of functions of a complex variable (i.e. establishing the expression of residues), as well as advances in the theory of finite groups. Cauchy was never the leader of a school of mathematicians, and he behaved sometimes awkwardly with young researchers as Abel or Galois, he underestimates, or even lost, memories of the first importance. His relations with his colleagues are generally not very easy.

\parpic[l][t]{%
  \begin{minipage}{40mm}
    \fbox{\includegraphics[width=110px,height=140px]{img/medaillons/cayley.eps}}
  \end{minipage}
}
\textbf{Cayley, Arthur} (1821-1895), born in Richmond (Surrey), showed very early strong predispositions for mathematics. However, despite the great interest of his first publications, he couldn't emerge as a mathematician, he decided to study law and became a lawyer in 1849. During fourteen years he held this job doing at the same time scientific researches. In 1863, Cayley was appointed professor at Cambridge and was finally able to devote himself entirely to mathematics. Throughout the work of Cayley, especially in his early works, there is a sensitive influence of the founders of the English school algebra who formulated the program of modern algebra by giving priority to the formal approach of problems. Educated mathematician and creator, Cayley, in the tradition of the English school, was able to develop new and fruitful theories. The richness of Cayley approach appears from his early work on group theory (1854). Cayley, addressing the work of Galois, Gauss and Cauchy with the methods of English algebraists, provides a definition of abstract groups which led to the notion of isomorphism. The study of linear equations systems led to Cayley to the determinants. In his early work, he established many rules for calculating the determinants, including the relation of multiplication of determinants that was already in the works of Cauchy, Jacobi and Binet. Next to original studies on the determinants, we meet the concept of rectangular array representing the coefficients of a system of linear equations or coefficients of a linear transformation. Cayley studied the rectangular matrices with real coefficients and complex, he also introduced the matrix operations and describes their properties, including the non-commutative multiplication. This is probably the first appearance of linear algebra. A few years later, Cayley also study non-associative systems and publish the results of multilinear algebra. Cayley has spent many of his publications on the problems of geometry and the study of algebraic curves and surfaces. At twenty-two years, he expressed the idea of the geometry of $n$ dimensions, idea that was also made, almost simultaneously, but in a slightly different form, by Grassman. Cayley did not return until much later (in 1870) on the n-dimensional space, but its algebraic method contributed to important discoveries that took place in other areas of geometry. Thus, in the \textit{Sixth Memoir on Quantics} of 1859 he introduced the projective metric, thereby subordinating the metric geometry to projective geometry; he demonstrated that the basics of the metric geometry (angles and distances) are the invariants and covariants of certain linear transformations of the absolute quadric.

\parpic[l][t]{%
  \begin{minipage}{40mm}
    \fbox{\includegraphics[width=110px,height=140px]{img/medaillons/chandrasekhar.eps}}
  \end{minipage}
}
\textbf{Chandrasekhar, Subrahmanyan} (1910-1995) obtained at the age of 23 his Pd.D. at Trinity College (Cambridge University). Specialist in astrophysics Chandrasekhar made a decisive advance knowledge of the hydrodynamic evolution and hydromagnetic energy transfer by radiation without forgetting the relativistic and quantum effects in the evolution of stars. His major contribution in this area is the transformation of white dwarf stars and beyond with a mass greater than the Chandrasekhar limit (1.44 that of the sun), the collapse in a neutrons star. Objects more massive giving black holes.\\\\

\parpic[l][t]{%
  \begin{minipage}{40mm}
    \fbox{\includegraphics[width=110px,height=140px]{img/medaillons/clairaut.eps}}
  \end{minipage}
}
\textbf{Clairaut, Alexis-Claude} (1713-1765) was a member of the Académie Française des Sciences and was one of the most knew mathematicians and physicists of the 18th century. At age 10, he knew infinitesimal calculus, at 12, he submitted his first study at the Academy of Sciences and at 18, he published a book containing important extensions to the geometry that have permitted him the admission to the academy in 1731. Clairaut was one of the scientists who accompanied Maupertuis in Lapland to acquire the necessary data for determining the shape of the earth. In 1743 he published his Theory of the figure of the Earth, where he calculated more accurately than Newton had done, the shape adopted by a rotating body due to the natural gravitation of its parts. In 1760 he published his \textit{Théorie du mouvements des comètes}, which accurately predicted the date of Halley's comet will arrive at the Sun's nearest point.

\parpic[l][t]{%
  \begin{minipage}{40mm}
    \fbox{\includegraphics[width=110px,height=140px]{img/medaillons/cohen.eps}}
  \end{minipage}
}
\textbf{Cohen, Paul Joseph} (1934-2007) was an mathematician and logician born in the New Jersey and died in Stanford. In 1963, Cohen has discovered a new model building, named "forcing", which now plays a key role in set theory and model theory. He also built models of set theory (assumed consistent) in which the axiom of choice and the continuum hypothesis are not verified, which, given the earlier work of Kurt Gödel, establishes that axiom of choice and the continuum hypothesis are independent of the usual systems of set theory. This work has earned Cohen, in 1966, the Fields Medal of the International Mathematical Union. He is also the author of interesting works in classical analysis.

\parpic[l][t]{%
  \begin{minipage}{40mm}
    \fbox{\includegraphics[width=110px,height=140px]{img/medaillons/connes.eps}}
  \end{minipage}
}
\textbf{Connes, Alain} (1947 -) was born in 1947 in Draguignan (France). Old student of the École Normale Supérieure, he received in 1980, the Prix Ampère, one of the most important award of the Académie des Sciences. He was elected member of the academy in 1981 (he was the youngest member). The first work of Alain Connes enrol directly in the tradition of John von Neumann and his immediate followers. The development of quantum physics to the years 1920 had made the agenda of the study areas in three dimensions rather than as one where we believe we live, or four, as in Einsteinian relativity, but to an infinite size (Hilbert spaces). One of the essential tools of quantum physics is the notion of operator in such a space, generalizing the notion of rotation of a Euclidean space. The theory of operator algebras started around 1930 by the work of von Neumann, who showed the importance of a certain type of operator algebras, now named "von Neumann algebras", and that established for these algebras a theorem of prime factorization quite similar to the well known decomposition theorem for ordinary integers. Since the origin, the factors were classified into three types: type I factors, II and III. We had an early understanding of type I factors and a lot of information on those of type II, but the factors of type III remained for a long time much more mysterious. Even the examples were few and von Neumann said about this: «\textit{This is the most refractory of all, and tools for its study are lacking, at least for now}». The first success of Connes, which immediately gave him international fame, was a major breakthrough towards the elucidation of the structure of type III factors and can be said to be the first to have acquired a practical knowledge of these objects, until now rather enigmatic, as a whole. Very roughly, the results of Connes bring the study of factors of type III to that of type II and their automorphisms. The work of Alain Connes is that of a complete mathematician, capable of solving difficult problems bequeathed by the past, but also to completely change a discipline by introducing new original ideas.

\parpic[l][t]{%
  \begin{minipage}{40mm}
    \fbox{\includegraphics[width=110px,height=140px]{img/medaillons/copernic.eps}}
  \end{minipage}
}
\textbf{Copernic, Nicolas} (1473-1543), studying at Krakow University until 1491, he then went to Italy to take courses in canon law at Universita de Bologna. He also follows the astronomy courses of Domenico Maria Novara, one of the first scientists to question the teachings of Ptolemy. In 1500, he taught mathematics in Rome, before returning for one year in Frauenburg where his uncle take him as canon in 1497. Having obtained permission to continue his studies in Italy, he enrolled in the faculties of law and medicine in Padua and received his PhD in canon law in Ferrara in 1503. Finally, he returned to Frauenburg where he built an observatory and began his famous research in astronomy. He remained there until his death. Cosmology of this time is then based on the geocentric system of Ptolemy. The Earth is motionless at the center of several concentric spheres carrying the Moon, Mercury, Venus, Sun, Mars, Jupiter, Saturn and finally the stars. But this system does not satisfy Copernic, he found it to complicated and flawed. He then consults the authors of antiquity (Cicero, Aristarchus of Samos, etc.) and finds that some of them already consider the rotation of planets, including Earth around the Sun (considered as fixed). Copernic then shows that the combination of movements of the Earth and planets perfectly explains the apparent motion of the planets (in forward and backward). In addition, he establishes that their apparent diameter changes arise as a consequence of their revolution around the Sun. His researches will continue for thirty-six years and show that the Moon is a satellite of the Earth and the Earth's axis is not fixed. His masterpiece \textit{De Revolutionibus orbium coelestium} was published in 1543 in Nuremberg and Copernicus received the first copies only a few hours before his death. In the dedication he made to Pope Paul III, he presents his system as a pure hypothesis, thus avoiding the condemnation of the Church. Adopted a century after his death, after being violently rejected, the Copernican system brought a profound revolution in the conception of the world and more generally in scientific thought.

\parpic[l][t]{%
  \begin{minipage}{40mm}
    \fbox{\includegraphics[width=110px,height=140px]{img/medaillons/coriolis.eps}}
  \end{minipage}
}
\textbf{Coriolis, Gaspard} (1792-1843) was a French mathematician and engineer who brought to light the centrifugal composed forces, named "Coriolis forces". The engineer of Roads and Bridges is the author of important works in mechanics. In 1835, he demonstrated that the acceleration of a moving object in a rotating frame is subjected to an additional (Coriolis force) perpendicular to the direction of movement of the mobile in this referential. Even if the force produced by the rotation of the planet has a low intensity on the surface of the Earth, it influences the direction of ocean and atmosphere currents. It produces a deflection to the east and explains, for example, the circular movement of hurricanes.

\parpic[l][t]{%
  \begin{minipage}{40mm}
    \fbox{\includegraphics[width=110px,height=140px]{img/medaillons/coulomb.eps}}
  \end{minipage}
}
\textbf{Coulomb, Charles Augustin} (1736-1806) was a French physicist pioneer in electrical theory. Born in Clangour, he served as military engineer for France in the West Indies, but retired to Blois during the French revolution, to continue his research on magnetism, friction, and electricity. In 1777 he invented the torsion balance to measure the strength of the electric and magnetic attraction. With this invention, Coulomb was able to formulate the principle, now known as "Coulomb's law", which governs the interaction between electric charges. In 1779 Coulomb published the treaty Theory of simple machines, an analysis of friction in machines. After the revolution, Coulomb left his retirement and assisted the new government to design a metric system of weights and measures. The unit used to express the amount of electrical charge, the "Coulomb", named after the physicist.

\parpic[l][t]{%
  \begin{minipage}{40mm}
    \fbox{\includegraphics[width=110px,height=140px]{img/medaillons/cournot.eps}}
  \end{minipage}
}
\textbf{Cournot, Antoine Augustin} (1801-1877) studied at the Gray College from 1809 to 1816. He won prizes for it's excellence in mathematics. In 1820 he joined the Collège Royal de Besançon and won the prize of honour in mathematics. With two memories and two translations of various treatises in mathematics, he draws the attention to Poisson, who appointed him in 1834 professor of analysis and mechanics at the Faculté de Lyon. Augustin Cournot is a scientific, that is to say, a man of extensive knowledge in all fields of science, but a scientific philosopher, who by his modesty, has not known celebrity. Cournot was first a teacher and great popularizer. Three mathematics books distinguish Carnot: \textit{Elementary Treatise of the theory of functions and infinitesimal calculus} (1841); \textit{Presentations of the theory of chance and probabilities} (1843), \textit{On the origin and limits of the correspondence between algebra and geometry} (1847). But Cournot's genius lies in the introduction of probability in economics. He is the precursor of modern theories in economics, that inspired Léon Walras who in his autobiography completed in 1904, and in several letters, reminded the role played in the development of his thought, on the one hand, the work of Antoine Augustin Cournot, and on the other hand, that of his father, the economist and philosopher Auguste Walras who was a classmate of Augustin Cournot at the École Normale.

\parpic[l][t]{%
  \begin{minipage}{40mm}
    \fbox{\includegraphics[width=110px,height=140px]{img/medaillons/clausius.eps}}
  \end{minipage}
}
\textbf{Clausius, Rudolf} (1822-1888) is one of the greatest physicists of the 19th century. He is known primarily for his contribution to the study of thermodynamics. The first, this German scientist formulated what is commonly named the "second principle of thermodynamics" and proposed a clear definition of the entropy. He is also one of the main creators of the kinetic theory of gases. Born in Köslin (Pomerania), Clausius attended the universities of Berlin and after Halle which he graduated in 1848. Professor until his death, he was responsible of the Chair of Physics at the Royal School of Artillery and Engineering in Berlin (1850-1855) and, simultaneously, at the University and Polytechnic of Zurich (1855-1867), then at the University of Würzburg (1867-1869), and finally at Bonn from 1869 to his death. His first publication in 1850 in the \textit{Annalen der Physik} in Poggendorff, attracted a widespread attention. He was searching to reconcile the idea of equivalence between work and heat. Clausius pointed out that the assumption of the conservation of heat in the process of transfer was not an essential part of the theory of Carnot. He establishes that in an ideal machine, the amount of heat taken to the boiler must always be greater than that which is transferred to the condenser, and an amount exactly equal to the work done. This important synthesis performed, Clausius, in the same publication, enunciated what we now name the "second law of thermodynamics". It was the necessary need, already established by Carnot, of the presence, not just of a warm body (the boiler), but also of a cold body (the condenser) for a steam to provide a work. In 1854, Clausius, pushing further the views expressed in 1850, offered the first clear statement of the concept of entropy. He was looking to measure the ability of the heat energy of any system to provide real non-ideal work. In the case of the heat conduction along a solid rod, for example, heat flows from the hot end to the cold end without providing any work, although this transfer is accompanied by a decrease in the ability of the hot end to serve subsequently as a potential source of work. This decrease occurs because at the end of the process the heat energy is held by a body located at a lower temperature than the initial state. It has not been lost, but only deteriorated because, according to the second law of thermodynamics, we can't find the initial temperature without the help of external work. The last major contributions of Clausius are from 1857 and 1858 and related to the kinetic theory of gases. Although he is not the first to have developed it, already proposed and discussed by Joule and Krönig in particular, it ranks with Maxwell among its founders. He introduced the concept of mean free path and establishes the important distinction between the translational energy and internal energy of a gas particle. In addition, we generally recognized him the merit of having, by his theoretical work, make a bridge between atomic theory and thermodynamics.

\parpic[l][t]{%
  \begin{minipage}{40mm}
    \fbox{\includegraphics[width=110px,height=140px]{img/medaillons/curiepierre.eps}}
  \end{minipage}
}
\textbf{Curie, Pierre} (1859-1906) is considered as one of the pioneers of studies on chemistry/physics radioactivity. He is even in its thesis published in 1898 that the term "radioactivity" was used for the first time by his wife Marie and him. The education of Pierre began at a very young age by his father, who was a military General Surgeon. The Curies had the habit of visiting the countryside near Paris on Sunday, Pierre, during his walks, quickly learned all the names of plants and animals. Since the school was not obliged at this time (not before 1881 when Jules Ferry edicted a law for this), Pierre was educated at home with his mother, then with his brother and after with tutors and finally, alone. At the age of 14, the education of Pierre was delegated to Mr. Bazille, who taught him elementary and special mathematics, this greatly developed the mental abilities of Pierre, who had a clear interest in mathematics. At the age of sixteen, he was received Bachelor of Science. In 1877, he obtained a degree in physics from the École des Pharmacies... In subsequent years, he studied crystals and magnetism, which will eventually lead to the discovery of piezoelectricity. In 1877, he took a position as an assistant. Afterwards he became demonstrator of physics experiments for laboratories until 1882 when he became director of all practical work in the schools of physics and industrial chemistry. Pierre married his wife, Marie Sklodowska in 1895 and they had two children together, Irene and Eve. Pierre Curie won in 1903 with his wife, the Nobel Prize in Physics for their work on radioactive substances and their discovery of two new elements: radium and polonium.

\parpic[l][t]{%
  \begin{minipage}{40mm}
    \fbox{\includegraphics[width=110px,height=140px]{img/medaillons/cramer_gabriel.jpg}}
  \end{minipage}
}
\textbf{Cramer, Gabriel} (1704-1752) was a swiss mathematician. Cramer showed promise in mathematics from an early age. At 18 he received his doctorate and at 20 he was co-chair of mathematics at the University of Geneva. In 1728 he proposed a solution to the St. Petersburg Paradox that came very close to the concept of expected utility theory given ten years later by Daniel Bernoulli. He published his best-known work in his forties. This included his treatise on algebraic curves (1750). He edited the works of the two elder Bernoullis, and wrote on the physical cause of the spheroidal shape of the planets and the motion of their apsides (1730), and on Newton's treatment of cubic curves (1746). In 1750 he published Cramer's rule, giving a general formula for the solution for any unknown in a linear equation system having a unique solution, in terms of determinants implied by the system. This rule is still standard. He did extensive travel throughout Europe in the late 1730s, which greatly influenced his works in mathematics. Cramer lived a busy life, for in addition to his teaching and correspondence with many mathematicians, he produced articles of considerable interest although these are not of the importance of the articles written by most of the top mathematicians with whom he corresponded. He published articles in various places including the \textit{Memoirs of the Paris Academy} in 1734, and of the Berlin Academy in 1748, 1750 and 1752. The articles cover a wide range of subjects including the study of geometric problems, the history of mathematics, philosophy, and the date of Easter. He published an article on the aurora borealis in the \textit{Philosophical Transactions of the Royal Society of London} and he also wrote an article on law where he applied probability to demonstrate the significance of having independent testimony from two or three witnesses rather than from a single witness. He died in 1752 at Bagnols-sur-Cèze while travelling in southern France to restore his health. 

\parpic[l][t]{%
  \begin{minipage}{40mm}
    \fbox{\includegraphics[width=110px,height=140px]{img/medaillons/curiemarie.eps}}
  \end{minipage}
}
\textbf{Curie, Marie} (1867-1934) was a chemist and physicist born in Warsaw and died in Haute-Savoie. Daughter of a father professor in mathematics and physics and of a mother who was teacher, she is the youngest of a family of 4 sisters. Between 1876 and 1878 she lost a sister and his mother. She took refuge in studies where she excelled in all subjects, and where the maximum score was granted to her. She thus obtains his graduation from high school with a gold medal in 1883. She wants to pursue higher education and teach, but these studies are forbidden to women. When her older sister, Bronia, left to study medicine in Paris, Marie agrees as a governess in province hoping to save enough money to join her sister while having originally intended to return in Poland to teach. After three years, she returned to Warsaw, where his cousin helped her to enter in a laboratory. In 1891, she moved to Paris, where she was hosted by her sister and brother. The same year, she enrolled to study physics at the Faculté de Paris. Three years later, she graduated in physics, being first in her class. During the summer, a scholarship is granted to Marie, which allows her to continue his studies in Paris. A year later, she graduated in mathematics, being second of her class. Then she hesitates to return to Poland. At a party she met Pierre Curie (her future husband), who is head of physics works at the Écoloe Municipale de Chimie et Physique Industrielle and also studied magnetism, with which she will work. Mary receives (with her husband Pierre Curie) one half of the Nobel Prize for Physics in 1903 (the other half is given to Henri Becquerel) for research on radiation. In 1911, she won the Nobel Prize in Chemistry for his work on polonium and radium.

\parpic[l][t]{%
  \begin{minipage}{40mm}
    \fbox{\includegraphics[width=110px,height=140px]{img/medaillons/irene_joliot_curie.jpg}}
  \end{minipage}
}
\textbf{Joliot-Curie, Irène} (1897-1956) the daughter of Marie Curie and Pierre Curie, born in Paris, and the wife of Frédéric Joliot-Curie. She went to the Sorbonne in Paris to complete a degree in mathematics and physics in 1918. Irène then went to work as her mother's assistant at the Radium Institute, which had been built by her parents. Her doctoral thesis was concerned with the alpha decay of polonium, the element discovered by her parents (along with radium) and named after Marie's country of birth, Poland. Irène became Doctor of Science in 1925.in 1933, Joliot-Curie and her husband were first to discover the accurate weight measurement of the neutron. Jointly with her husband, Joliot-Curie was awarded the Nobel Prize in Chemistry in 1935 for their discovery of artificial radioactivity. Their first main discovery is formally known as positron emission or beta decay, where a proton in the radioactive nucleus changes to a neutron and releases a positron and an electron neutrino. . This made the Curies the family with the most Nobel laureates to date. She was also one of the first three women to be a member of a French government, becoming undersecretary for Scientific Research, under the Popular Front in 1936. In 1945, she was one of the six commissioners of the new French Alternative Energies and Atomic Energy Commission (CEA) created by de Gaulle and the Provisional Government of the French Republic. In 1948, using the works of nuclear fission, the Joliot-Curies along with other scientists created the first French nuclear reactor. She died in Paris on 17 March 1956 from an acute leukemia linked to her exposure to polonium and X-rays (she had been accidentally exposed to polonium when a sealed capsule of the element exploded on her laboratory bench in 1946), the same disease that had killed her mother.

\parpic[l][t]{%
  \begin{minipage}{40mm}
    \fbox{\includegraphics[width=110px,height=140px]{img/medaillons/frederic_joliot_curie.jpg}}
  \end{minipage}
}
\textbf{Joliot-Curie, Frédéric} (1900-1958) was a French physicist, husband of Irène Joliot-Curie with whom he was jointly awarded the Nobel Prize in Chemistry in 1935 for their discovery of artificial radioactivity. Born in Paris, France, he was a graduate of ESPCI Paris. In 1925 he became an assistant to Marie Curie, at the Radium Institute. He fell in love with her daughter Irène Curie, and soon after their marriage in 1926 they both changed their surnames to Joliot-Curie. At the insistence of Marie, Joliot-Curie obtained a second baccalauréat, a bachelor's degree, and a doctorate in science, doing his thesis on the electrochemistry of radio-elements. While a lecturer at the Paris Faculty of Science, he collaborated with his wife on research on the structure of the atom, in particular on the projection, or recoil, of nuclei that had been struck by other particles, which was an essential step in the discovery of the neutron by Chadwick in 1932. In 1935 they were awarded the Nobel Prize in Chemistry for their discovery of "artificial radioactivity", resulting from the creation of short-lived radioisotopes by nuclear transmutation from the bombardment of stable nuclides such as boron, magnesium, and aluminium with alpha particles. In 1937 he left the Radium Institute to become a professor at the Collège de France. Joliot-Curie was mentioned in Albert Einstein's 1939 letter to President Roosevelt as one of the leading scientists on the course to nuclear chain reactions. Frédéric Joliot-Curie devoted the last years of his life to the creation of a center for nuclear physics at Orsay, where his children were educated.

\phantomsection
\addcontentsline{toc}{section}{D}
\label{sec:D}	

\parpic[l][t]{%
  \begin{minipage}{40mm}
    \fbox{\includegraphics[width=110px,height=140px]{img/medaillons/dalton.eps}}
  \end{minipage}
}
\textbf{Dalton, John} (1766-1844) was a British chemist and physicist who developed the atomic theory upon which was founded modern physical science. Dalton began in 1787 a series of meteorological observations that he continued for fifty-seven years, accumulating some two hundred thousand observations and measurements of time in the Manchester area. Dalton's interest in meteorology led him to study different phenomena and instruments used to measure them. He was the first to prove the validity of the idea that rain is precipitated by a drop in temperature, not by a change in atmospheric pressure. Dalton arrived at his atomic theory by studying the physical properties of atmospheric air and other gases. During his research, he discovered the law of partial pressures of gases mixed, often known as the "Dalton's law" that the total pressure exerted by a mixture of gases is equal to the sum of the individual pressures which would be exerted if each gas alone occupied the whole volume.

\parpic[l][t]{%
  \begin{minipage}{40mm}
    \fbox{\includegraphics[width=110px,height=140px]{img/medaillons/davinci.eps}}
  \end{minipage}
}
\textbf{Da Vinci, Leonardo} (1452-1519) was an Italian painter, sculptor, architect and man of science. Man of universal mind, both artist, scientist, inventor and philosopher Leonardo embodied the universal spirit of the Renaissance and remains one of the great men of that time. At the age of five, his father having noticed his gift for drawing, send it as an apprentice in the workshop of Verrocchio in Florence. He enters at the age of twenty years in the Guild of Painters, and began his career as a painter with famous works such as \textit{La vierge à l'oeillet}, or the \textit{L'Annonciation} (1473). He improves the sfumato technique (printing mist) to a point of refinement never achieved before him. In 1481, the monastery of San Donato order the \textit{Adoration des Mages}, but Leonard annoyed to being selected for the decoration of the Sixtine Chapel in Rome, would never finish this painting and left Florence for Milan. After the completion of \textit{La Vierge aux rochers} for the chapel of San Francesco Grande, and that of the equestrian statue of Francesco Sforza, he finds fame throughout Italy. In 1495, the Dominicans of Santa Maria delle Grazie order him \textit{La Cène}. In 1498, he realized the ceiling of the Sforza palace. During this period he realized the \textit{Mona Lisa} and \textit{La Bataille d'Anghiari}. Leonardo also carries a large amount of studies on zoology, botany, anatomy and geology. He imagines multiple devices and machines, the first flying machine, which will remain at the stage of design. More than itself as a scientist, Leonardo has impressed his contemporaries and subsequent generations by his methodical approach to knowledge, learning skills, observation knowledge, analyse knowledge. The approach he exhibited in all the activities he was interested for, both technical and art (both are also not distinguished in his mind), stemmed from a prior accumulation of detailed observations, knowledge scattered here and there, which tended towards surpassing what was already there, with perfect aim. Many drafts, notes and treated by Leonardo da Vinci are not, strictly speaking, original discoveries, but are the result of research carried out in a encyclopedic purpose. In 1516, he joined the court of Francis I., where he participated in planning urbanistic projects. Form Leonardo da Vinci, remains today 7,000 notes and drawings, forty certified works (eight have disappeared).

\parpic[l][t]{%
  \begin{minipage}{40mm}
    \fbox{\includegraphics[width=110px,height=140px]{img/medaillons/dantzig.eps}}
  \end{minipage}
}
\textbf{Dantzig, George Bernard} (1914-2005) was a mathematician born in Portland and died in Stanford, inventor of the famous "Simplex algorithm" in linear optimization. His father, Tobias, is a Russian mathematician who had studied with Henri Poincaré in Paris. He married a colleague from La Sorbonne, Anja Ourisson, and the couple emigrated to the United States. He is the main actor of a famous story in mathematics. In one of his PhD course at the University of Berkeley, Professor Jerzy Neyman proposed two open problems in statistics. An open problem is a problem that although it was formulated, has not yet been resolved. Such problems are a significant challenge and require research that can extend over several years. Dantzig was late and thought it was homework. Without taking several years but a few days, he solved the problems. He received his PhD from Berkeley in 1946. Six years later, he was hired to do mathematical research at the RAND Corporation, where he implements the simplex algorithm in computers. In 1960, UC Berkeley hired him to teach computer science, and eventually to became the head of the operational research center. Six years later, he held a similar position at Stanford University, a position he held until his retirement in the 1990s. In addition to his work on the simplex algorithm and linear optimization, he also worked on methods for decomposing large problems, sensitivity analysis, methods of resolution matrix with pivot, the non-linear optimization and linear stochastic optimization.

\parpic[l][t]{%
  \begin{minipage}{40mm}
    \fbox{\includegraphics[width=110px,height=140px]{img/medaillons/debye.eps}}
  \end{minipage}
}
\textbf{Debye, Peter Joseph Wilhelm} (1884-1966) was a  physicist and chemist born in Maastricht and died in New-York. Debye subscribe in 1901 at the Universität von Aix-la-Chapelle in Germany. He studied there mathematics and classical physics and holds in 1905 a degree in electrical engineering. In 1907 he produced his first scientific publication, an elegant mathematical solution of a problem involving Foucault currents. He studied at Aix-la-Chapelle under the direction of Arnold Sommerfeld. In 1906, he accompanied Sommerfeld in Munich as an assistant. He obtained his PhD in 1908 with a dissertation on radiation pressure. In 1910, he proved Planck's law with a method that Max Planck admitted that she was simpler than his. In 1911 Debye was appointed professor at Zurich in Switzerland. He then went to Utrecht in 1912 (Germany), in Göttingen in 1913, he returned to Zurich in 1920, went to Leipzig (Germany) in 1927 and in Berlin in 1934 where he became director of the Kaiser Wilhelm Society that will in 1938 take the name of Max Planck Society. In 1912, he extended Albert Einstein theory of specific heat at low temperatures including contributions from low-frequency phonons know today as the "Debye model". In 1913, he extended the Niels Bohr theory of atomic structure by introducing elliptical orbits, a concept also proposed by Arnold Sommerfeld. Debye benefits in 1938 of a proposal for a conference at Cornell University in Ithaca to go to the United States and then stay at Cornell University, where he became professor and then, for 10 years, director of the Department of Chemistry. He remains to Cornell the rest of his career. He retired in 1952 but continued his research until his death.

\parpic[l][t]{%
  \begin{minipage}{40mm}
    \fbox{\includegraphics[width=110px,height=140px]{img/medaillons/descartes.eps}}
  \end{minipage}
}
\textbf{Descartes, René} (1596-1650) was French philosopher, scientist and mathematician, founder of modern rationalism. Born in La Hague, of a father consultant at the Parliament of Rennes, Descartes received from 1607 to 1614 a teaching decisive for him from the Jesuits of the Collège Royal de La Flèche. This experience led him to propose an overhaul of Sciences, criticizing the lack of foundation of professed education. He was trained as a lawyer in 1616 and then started a military career in 1618, undertook trips, mixed scientific life and meet high society people, before devoting himself fully to philosophy. He spent his life between France and the Netherlands, fleeing the cities, frequenting libraries and meeting the most illustrious minds of his time, including Bérulle, Fermat, Gassendi, Hobbes, and Pascal. He died of pneumonia in Stockholm, bequeathing to posterity a work surrounded by legends and imbued with a new spirit.

\parpic[l][t]{%
  \begin{minipage}{40mm}
    \fbox{\includegraphics[width=110px,height=140px]{img/medaillons/dirac.eps}}
  \end{minipage}
}
\textbf{Dirac, Paul Adrien Maurice} (1902-1984) was born in Bristol and studied at the Universities of Bristol and Cambridge. In 1926, for his PhD (the first thesis in the world having for subject "quantum mechanics"), he introduced a general formalism for quantum physics shortly and independently after Heisenberg (he finds the non-commutativity of position and momentum operators). In 1928, he developed a relativistic theory to describe the properties of the electron. This theory led to the postulate of a particle identical to the electron in all its aspects but of opposite charge, that is to say positive and that have to annihilate with a negative electron in a collision. Dirac's theory was confirmed in 1932 when the physicist Carl Anderson discovered the positron. Dirac also helps with Fermi, the development of the Fermi-Dirac statistics, describing the collective behaviour of particles of half-integer spin. In 1933 Dirac shared the Nobel Prize in Physics with the Austrian physicist Erwin Schrödinger. In 1939, he became a member of the Royal Society. He was professor of mathematics at Cambridge from 1932 to 1968, professor of Physics at the State University of Florida from 1971 until his death, and a member of the Institute of Advanced Studies regularly between 1934 and 1959.

\parpic[l][t]{%
  \begin{minipage}{40mm}
    \fbox{\includegraphics[width=110px,height=140px]{img/medaillons/dirichlet.eps}}
  \end{minipage}
}
\textbf{Dirichlet (-Lejeune), Peter Gustav} (1805-1859) was born in Düren (Germany). Dirichlet was a brilliant student who completed his secondary education at age of 16. Because of the low quality of universities in Germany at this time, Dirichlet decided to go study in Paris, taking with him the \textit{Disquisitiones Arithmeticae} of Gauss as a bible. In the French capital, his personal situation is facilitated by the General Foy, an old important general of the Napoleonic battles who show him kindness and for whom he became the tutor of his children. Dirichlet then met some of the greatest mathematicians, including Legendre, Poisson, Laplace and Fourier. This last especially impress many Dirichlet, and will cause his interest for trigonometric series and mathematical physics. It was in Paris that Dirichlet wrote his first significant contribution to mathematics, in 1825 he is at the initiative of the proof of the case n = 5 of Fermat's last theorem, proof completed by Legendre a little bit later. End of 1825, General Foy died and Dirichlet decides to return to Germany. He first taught at the University of Breslau, at the military school in Berlin and at the University of Berlin in 1829, where he remained for 27 years. Among his students, we note the names of Kronecker and Riemann. Dirichlet is described as a good teacher, but not a perfect one. He gives the appearance of someone dirty, always wearing a cigar and a beer, apparently not really concerned about the image he gives. It was also told that he was often late. In 1848, his master and friend Karl Jacobi is diagnosed as being ill with diabetes. Dirichlet accompany Jacobi on a journey of 18 months in Italy. Back in Germany, Dirichlet begins to be tired of the heavy teaching tasks that he must assume. At Gauss's death, he succeeded him in Göttingen. This is unfortunately not for long, because he also died in 1859 or a heart attack. The scope of work of Dirichlet illustrates the depth of the German Mathematical early at his Golden Age. He is also the first to define a sufficient condition for convergence of a Fourier series (in the case of piecewise continuous functions), the theorem of arithmetic progression, the extension of harmonic functions defined on the boundary of an open and a whole class of partial differential equations is named "Dirichlet problem". We also owe him many contributions in arithmetic that bear his name like the theorem of Dirichlet units, Dirichlet series, etc.

\parpic[l][t]{%
  \begin{minipage}{40mm}
    \fbox{\includegraphics[width=110px,height=140px]{img/medaillons/doppler.eps}}
  \end{minipage}
}
\textbf{Doppler, Christian} (1803-1853) was an Austrian mathematician and physicist famous for his discovery of the Doppler effect. After studying at the Wien Universität, Doppler became assistant professor in this institution in 1829. This job position being not renewed, he has in mind to emigrate to the United-States. He renounces to leave his country after being named in Prague in 1837 and in 1849 at the Wien Polyteknische Schule. In 1850, he founded the institute of physics of the Wien Universität which he is the only professor and first director. Suffering from a lung disease, tuberculosis, he stopped his jobs in 1852. His scientific work is varied: optics, astronomy, electricity ... His most famous publication was presented in 1842 at the Royal Academy of Sciences of Bohème and has for title \textit{On the coloured light of the double stars and other stars of the sky}, using the Doppler effect. His calculations were wrong, the real offset of the light frequency was too low to be detected at this time. In 1846 Doppler published a correction of the initial work that takes into account the relative speeds of the light source and the observer.

\parpic[l][t]{%
  \begin{minipage}{40mm}
    \fbox{\includegraphics[width=110px,height=140px]{img/medaillons/drude.eps}}
  \end{minipage}
}
\textbf{Drude, Paul Karl Ludwig} (1863-1906) was a physicist born in Braunschweig and died in Berlin. Drude began his studies in mathematics at the Göttingen Universität, but then went then to physics. He completed his PhD in 1887 and wrote a thesis on the reflection and diffraction of light in crystals. In 1894 he was appointed professor at the Leipzigs Universität of Leipzig. In 1900 he obtained the post of editor of the scientific journal Annalen der Physik. The same year, he developed a model know today as "Drude model" explaining the thermal, electrical and optical properties of the material that will be taken in 1933 by Arnold Sommerfeld and Hans Bethe and will become the "Drude-Sommerfeld model". He teaches at the University of Giessen from 1901 to 1905 and was promoted Director of the Department of Physics at the University of Berlin. In 1906 he became a member of the Berlin Academy.

\phantomsection
\addcontentsline{toc}{section}{E}
\label{sec:E}

\parpic[l][t]{%
  \begin{minipage}{40mm}
    \fbox{\includegraphics[width=110px,height=140px]{img/medaillons/einstein.eps}}
  \end{minipage}
}
\textbf{Einstein, Albert} (1879-1955), born in Ulm and die in Princeton, is a theoretical physicist who was successively German and stateless (1896), Swiss (1901) and finally under the Swiss-American dual citizenship (1940). He published his theory of Relativity in 1905, and a theory of gravity named "General Relativity" in 1915. He contributed to the development of quantum mechanics and cosmology, and received the Nobel Prize in Physics in 1921 for his explanation of the photoelectric effect. His work is best known for the equation of equivalence which establishes an equivalence between matter and energy of a system. He is also known for his hypothesis on the corpuscular nature of light. But he also contributed to the development of many other theories (including quantum physics like LASER or EPR paradox). In 1905, Einstein received his PhD from the University of Zurich for a theoretical dissertation on the dimensions of molecules. He also published three theoretical papers of central importance on the development of the physics of the twentieth century. In the first of these articles, on Brownian motion, he made important predictions on the movement of particles randomly distributed in a fluid. During the rest of his life, Einstein devoted a considerable time to generalize even more his theory of General Relativity. He was trying to find a unified field theory, which was not completely successful, and made numerous attempts to describe the electromagnetic interaction and gravitational interaction in a common model.

\parpic[l][t]{%
  \begin{minipage}{40mm}
    \fbox{\includegraphics[width=110px,height=140px]{img/medaillons/erdos.eps}}
  \end{minipage}
}
\textbf{Erdös, Paul} (1913-1996) was the most prolific mathematicians of the 20th century, with about 1,500 articles (we have to go back to Euler for the same volume of publications). More than someone who was building theories, he solved problems, often with elegance and simplicity. Erdös was born in Budapest. Both his parents were teachers of mathematics in secondary. While Erdös was aged just one year, his father was taken prisoner by the Russians and deported to Siberia. These events contributed to the development of a very strong mother/son relation, which greatly influence the course of the life of Paul Erdös. At the age of 19, he began his studies at the university and quickly became known in mathematical associations. He publishes a new proof of Bertrand's postulate, which asserts that there exists a prime number between $n$ and $2n$ for all n. Two years later, he obtained his PhD (21 years old), then goes to a post-doc in Manchester. As Erdös is of Jewish origin, he can not return to Hungary in the late 30s, and he emigrated to the United-States. After several visits in Europe to survivors of his family after the Holocaust, he has problems in the United-States with the McCarthyism, and he sees himself banned from entering the country. Erdös is forced to settle down in Israel. With 1,500 publications, the contributions from Erdös are very important: number theory, combinatorics, discrete mathematics, where he was a master. Erdös had an exceptional ability to surround himself with the most competent mathematicians to solve it's conjecture. As a result, Erdös had many collaborators: 500 mathematicians wrote an article in common with him. Mathematicians had fun to define an Erdös number: any mathematician who published a paper together with Erdös has an Erdös number equal to 1. Anyone who has published a paper with someone who has an Erdös number equal to 1 has an Erdös number equal to 2. And so on ... Albert Einstein Erdös number is 2. However, among all these collaborations, at least one went wrong, and it is all the more regrettable as it concerns the most successful subject of Erdös. At the end of the 19th century, Hadamard and de La Vallée Poussin had proved the prime number theorem, that the number of primes less than or equal to $n$ is equivalent, when $n$ is large, $n/\ln(n)$. Their proof is particularly difficult! In 1949, Atle Selberg found an inequality he thinks that can be an important step towards an elementary proof of the prime number theorem. The inequality is presented to Erdös, who finds the missing key to complete the proof. An co-authored publication would probably have been the most appropriate to measure the contributions of each. But after a misunderstanding related to sending triumph postcards form Erdös, Selberg fears that Erdös take the advantage only for him. He published alone the full proof. He will receive the Fields Medal in 1950, and Erdös will only receive the Wolf Prize in 1984. Erdös life was really strange. He had no home, no wife, material contingencies were painful for him. He travelled alone, with two suitcases that contained all his possessions, going from university to university, living in a hotel or with mathematician friends... He is also the author of numerous "erdosismes" as this famous sentence: «A mathematician is a machine for turning coffee into theorem». Himself was doped with all kinds of amphetamines! Until the end of his life, Erdös didn't slow his mathematical activity. Die meant to him to stop doing math. He died in Warsaw, during a congress.

\parpic[l][t]{%
  \begin{minipage}{40mm}
    \fbox{\includegraphics[width=110px,height=140px]{img/medaillons/erlang.eps}}
  \end{minipage}
}
\textbf{Erlang, Agner Krarup} (1878-1979) was a Danish mathematician who worked on the theory of queues and management of telephone networks. Erlang has worked, on the basis of the work of Poisson that law of rare events has found its application dimension to telecommunications networks, the development of a mathematical model for the design of telecommunications networks on a statistical approach to achieve operating costs likely to enable a mass market.\\\\\\

\parpic[l][t]{%
  \begin{minipage}{40mm}
    \fbox{\includegraphics[width=110px,height=140px]{img/medaillons/euclide.eps}}
  \end{minipage}
}
\textbf{Euclid} (3rd century BC.) We know very little thing about the life of Euclid. He apparently taught mathematics in Alexandria at the request of Ptolemy 1st. He would thus appear as the founder of the famous Alexandrian school which influenced the work of Archimedes. In contrast, theories of Euclid are well known and constitute a reference in the history of mathematics. The masterpiece is undoubtedly of \textit{Euclid's Elements}. This book represents a remarkable synthesis of mathematical results and has left its mark on the discipline as a whole. It consists of thirteen books. The first four deal with plane geometry with the definitions of point, and the straight line and the surface. They also expose the calculation of areas of different polygons. The Book V contains the rudiments of analysis. The VI deals with the similarity of figures and gives the solution of quadratic equations using geometric constructions. Books VII, VIII, and IX deal with arithmetic. The X studies irrational numbers, and finally the last three deal with geometry in space. Euclid, also written books on geometric analysis, optics and astronomy. Perfect representation of the scientific statement, the \textit{Euclid's Elements} consist of various proposals classified into two groups: the axioms and assumptions. Among the five axioms, we find the famous postulate of «Euclid by any point of the plane passes one and only one line parallel to another line». This axiom is the foundation of Euclidean geometry, as opposed to non-Euclidean geometries that appeared some 2000 years later.

\parpic[l][t]{%
  \begin{minipage}{40mm}
    \fbox{\includegraphics[width=110px,height=140px]{img/medaillons/euler.eps}}
  \end{minipage}
}
\textbf{Euler, Leonhard} (1707-1783), Swiss mathematician born in Basel but also physicist, engineer and philosopher, he was one of the founders of the methods of differential and integral calculus.  His father, Paul Euler, was pastor of the Reformed Church, and Marguerite Brucker, his mother, the daughter of a pastor. Shortly after the birth of Leonhard Euler's family moved from Basel to the town of Riehen, where Euler spent most of his childhood. Paul Euler was a friend of the Bernoulli family. Jean Bernoulli is considerate as the main European mathematician and the one who had the greatest influence on the young Leonhard. The official education of Euler started early in Basel, where he was sent to live with his maternal grandmother. At the age of thirteen, he enrolled at the University of Basel, and in 1723 he received his Master of Philosophy with a dissertation that compared the philosophy of Descartes to this of Newton. At that time, he received every Saturday afternoon lessons from Johann Bernoulli, who quickly discovered that his student has incredible talent for mathematics. Euler began to study theology, Greek and Hebrew at the request of his father to become a pastor, but Jean Bernoulli convinced Paul Euler that Leonhard was destined to become a great mathematician. Euler was the first to treat analytically the algebra, equations, trigonometry and analytic geometry. In this work, he treated the subject of the development of series of functions and formulated the rule that only the convergent infinite series could be properly evaluated. He also discussed the three-dimensional surfaces and proved that the conic sections are represented by the general equation of second degree in two dimensions. Other works deal with simple algebra, the calculus of variations, the theory of numbers, imaginary and transcendental numbers, determinate and indeterminate algebra and graph theory. Euler brought also contributions in the fields of astronomy, analytical mechanics (variational calculus), hydrodynamics, optics and acoustics. Euler is considered as an eminent mathematician of the 18th century and one of the best and most prolific of all time and who introduced much of the notations still used in the early 21st century (symbols for the sum function, logarithm , exponential, etc.).

\phantomsection
\addcontentsline{toc}{section}{F}
\label{sec:F}

\parpic[l][t]{%
  \begin{minipage}{40mm}
    \fbox{\includegraphics[width=110px,height=140px]{img/medaillons/faraday.eps}}
  \end{minipage}
}
\textbf{Faraday, Michael} (1791-1867) was an English scientist who contributed to the fields of electromagnetism and electrochemistry. The young Michael Faraday, who was the third of four children, having only the most basic school education, had to educate himself. At 14 he became apprentice in a local bookbinder. During his seven year apprenticeship he read many books. At this time he also developed an interest in science, especially in electricity. Faraday was particularly inspired by the book \textit{Conversations on Chemistry} by Jane Marcet. It was by his research on the magnetic field around a conductor carrying a direct current that Faraday established the basis for the concept of the electromagnetic field in physics. Faraday also established that magnetism could affect rays of light and that there was an underlying relation between the two phenomena. He similarly discovered the principle of electromagnetic induction at the same time as Joseph Henry, diamagnetism, and the laws of electrolysis. His inventions of electromagnetic rotary devices formed the foundation of electric motor technology, and it was largely due to his efforts that electricity became practical for use in technology.

\parpic[l][t]{%
  \begin{minipage}{40mm}
    \fbox{\includegraphics[width=110px,height=140px]{img/medaillons/feigenbaum.eps}}
  \end{minipage}
}
\textbf{Feigenbaum, Mitchell} (1994-)  was born in New York City, from Polish and Ukrainian immigrants. He attended Samuel J. Tilden High School, in Brooklyn, New York, and the City College of New York. In 1964 he began his graduate studies at the Massachusetts Institute of Technology (MIT). Enrolling for graduate study in electrical engineering, he changed his area to physics. He completed his doctorate in 1970 for a thesis on dispersion relations. After short positions at Cornell University and the Virginia Polytechnic Institute and State University, he was offered a longer-term post at the Los Alamos National Laboratory in New Mexico to study turbulence in fluids. Although that group of researchers was ultimately unable to unravel the currently intractable theory of turbulent fluids, his research led him to study chaotic maps. Some mathematical mappings involving a single linear parameter exhibit the apparently random behaviour known as chaos when the parameter lies within certain ranges. As the parameter is increased towards this region, the mapping undergoes bifurcations at precise values of the parameter. At first there is one stable point, then bifurcating to an oscillation between two values, then bifurcating again to oscillate between four values and so on. In 1975, Dr. Feigenbaum, using the small HP-65 calculator he had been issued, discovered that the ratio of the difference between the values at which such successive period-doubling bifurcations occur tends to a constant of around 4.6692... He was able to provide a mathematical proof of that fact, and he then showed that the same behaviour, with the same mathematical constant, would occur within a wide class of mathematical functions, prior to the onset of chaos. For the first time, this universal result enabled mathematicians to take their first steps to unraveling the apparently intractable "random" behaviour of chaotic systems. This "ratio of convergence" is now known as the first Feigenbaum constant.

\parpic[l][t]{%
  \begin{minipage}{40mm}
    \fbox{\includegraphics[width=110px,height=140px]{img/medaillons/fermat.eps}}
  \end{minipage}
}
\textbf{Fermat, Pierre de} (1601-1665) was a French mathematician, author of a famous theorem without proof in arithmetic and nicknamed "the prince of amateurs". He is at the origin of Femat's principle (optics) and with his friend Blaise Pascal of probabilities. He also created the theory of numbers and made several discoveries in this field. Thus, some consider him the father of the modern theory. He outran the differential calculus for his work on calculus. He left to posterity the task of proving a theorem, the famous "Fermat's last theorem", on which mathematicians are bent for more than three centuries. It was not until 1993 that the British researcher Andrew Wiles proposed a very complexed proof.

\parpic[l][t]{%
  \begin{minipage}{40mm}
    \fbox{\includegraphics[width=110px,height=140px]{img/medaillons/fermi.eps}}
  \end{minipage}
}
\textbf{Fermi, Enrico} (1901-1954) was an Italian physicist, known for making the first controlled nuclear reaction. Very young Enrico Fermi showed an exceptional memory and high intelligence, allowing it to excel in studies. Enrico, deeply marked by the death of one of his very young brother, then throws in the study of physics to overcome his pain. Good student, he developed a passion for physics and mathematics and began studying various books dealing with mechanics, optics, astronomy and acoustics. A friend of his father, Adolfo Amidei engineer, who becomes aware of the unusual qualities of the young Fermi lends him various books on mathematics. Thus, at age seventeen, Enrico Fermi masters analytical geometry, projective geometry, calculus, integral calculus and mechanics. Starting 1918 Fermi studied at the University of Pisa. As usual, he studied alone various problems of mathematical physics and consult the works of Poincaré, Poisson or Appell. From 1919, he is interested in new theories such as relativity and atomic physics, and he acquired a great knowledge of theories such as relativity, the theory of black-body or the Bohr's hydrogen atomic model. Also Enrico Fermi, who was the only one at university aware of these theories, comes at the insistence of his teachers to give lectures where he exposes teachers and assistants the latest discoveries in atomic physics. In 1922, after four years at the university, Enrico Fermi published his first paper on General Relativity. In an Italian scientific community hostile to the work of Einstein, he is the only one with Levi-Civita to defend the theory of relativity. In 1922, Fermi received his graduate diploma after a submission on the X-ray diffraction. He attended various senior physicists in Italy, before becoming, for two years, a lecturer at the University of Florence. In 1926, he became professor of theoretical physics at the University of Rome La Sapienza. It was during this period that he developed the quantum statistical theory later named the "Fermi-Dirac statistics". From 1932, he focus more specifically on nuclear physics, and it is this same year he wrote an article on the beta radioactivity. In 1934, he developed his theory of the emission of beta radiation by including the neutron postulated in 1930 by Wolfgang Pauli that he renamed neutrino (neutron name was already used for another particle), and he develops the creation of artificial radioactive isotopes by slow neutron bombardment (for which he received the Nobel Prize in 1938).

\parpic[l][t]{%
  \begin{minipage}{40mm}
    \fbox{\includegraphics[width=110px,height=140px]{img/medaillons/feynman.eps}}
  \end{minipage}
}
\textbf{Feynman, Richard Phillips} (1918-1988) was  born in St Far Rockaway, Queens district of New York (United States) of Russian and Polish parents. His father, who encouraged him to ask questions and to challenge the commonly accepted things, has had a major influence. From his mother, he herited a strong sense of humor that never left him. Feynman is one of the most influential physicists of the second half of the 20th century, partly because of his work on relativistic quantum electrodynamics, quarks and superfluid helium. During his last year in the high school of Far Rockaway, Feynman won the championship of Mathematics of the University of New York... He also received a scholarship to study at the Massachusetts Institute of Technology (MIT) where he received his bachelor of advanced studies in 1939 after having initially study electronics and mathematics, and finally he attended all courses offered including physics during its second year course of theoretical physics reserved for graduate students. Feynman gets a remarkable score at the entrance examinations to Princeton University in mathematics and physics, but he had a very low score in the literature exam. During his studies at the Institute for Advanced Study at Princeton (IAS) (recently created and directed by Albert Einstein), Feynman worked under the direction of John Wheeler on the principle of least action applied to quantum mechanics. He established the foundations of Feynman diagrams and the approach of quantum mechanics trough integral paths. He obtained his PhD in 1942. He completely reformulated quantum mechanics using the path integral which generalizes the action principle of classical mechanics and invented diagrams that bear his name and are now widely used in quantum field theory (including electrodynamics quantum part). Musician, teacher remarkable writer of many popular books, he has also been involved in the development of the atomic bomb. After World War II, he taught at Cornell University and then at Caltech where he conducted fundamental research in the theory of superfluidity and quarks. Sin-Itiro Tomonaga, Julian Schwinger and Feynman are co-winners of the Nobel Prize for Physics in 1965 for their work in quantum electrodynamics.

\parpic[l][t]{%
  \begin{minipage}{40mm}
    \fbox{\includegraphics[width=110px,height=140px]{img/medaillons/fisher.eps}}
  \end{minipage}
}
\textbf{Fisher, Ronald Aymler} (1890-1962) born in London was a British biologist and statistician, who contributed greatly to finding modern statistics. Thanks to his works on statistics he earned the Darwin Medal in 1948, the Copley Medal in 1955 and the silver Darwin-Wallace  medal in 1958. In the field of statistics, he introduced many concepts such as Maximum Likelihood, Fisher information and Analysis of Variance (ANOVA). He is considered as a great precursor of Shannon. He is also one of the founders of modern genetics and a great follower of Darwin, in particular through the use of statistical methods, essential in population genetics. He contributed to the mathematical formalization of the principle of natural selection. He was first attracted by physics and obtained in 1912 a degree in astronomy at the University of Cambridge. From 1915 to 1919, he taught mathematics in London in private schools. In 1919, he was hired at Rothamsted Experimental Station to analyse the effect of rainfall on the yield of wheat where remained until 1933. In his publication \textit{On the mathematical foundations of theoretical statistics} of 1922, he defines a couple of basic concepts in statistics such as the notion of convergence, efficiency, likelihood and sufficient statistics. He proposed the maximum likelihood estimator in 1922 after making a first version in 1912. He also introduced in 1924 the analysis of variance. In 1925 he published some innovations in time series analysis and multiple correlations.

\parpic[l][t]{%
  \begin{minipage}{40mm}
    \fbox{\includegraphics[width=110px,height=140px]{img/medaillons/foucault.eps}}
  \end{minipage}
}
\textbf{Foucault, Leon} (1819-1868) was a French physicist famous for his demonstration of the movement of the Earth by the rotation of the plane of oscillation of the pendulum. Born in Paris, he worked with the French physicist Armand Fizeau on the measure of the speed of light. Foucault proved independently, that the speed of light in air was higher than in water. In 1851, he made a spectacular demonstration of the rotation of the Earth by suspending a pendulum with a long cable attached to the dome of the Pantheon in Paris. The pendulum demonstrated the rotation of the Earth on its axis. In 1855 he discovered that the force required to rotate a disk of copper increases when it should rotate with its rim between the poles of a magnet, the disk heating at the same time because of the "foucault's currents" induced in the metal. He also created a method for measuring the curvature of the mirrors of telescopes. He developed other instruments like a prism polariser and a gyroscope which is the basis of modern gyrocompass.

\parpic[l][t]{%
  \begin{minipage}{40mm}
    \fbox{\includegraphics[width=110px,height=140px]{img/medaillons/fourier.eps}}
  \end{minipage}
}
\textbf{Fourier, Joseph} (1768-1830) was a French physicist and mathematician known for the discover of trigonometric series and transformation that bear his name. Fourier lost his father and mother at the age of ten. The organist of Auxerre, Joseph Pallais, take Fourier in a boarding school. Recommended by the Bishop of Auxerre, he studied at the École Militaire d'Auxerre, held by the Benedictines of the Congrégation de Saint-Maur. Destinated at the monastic life, he prefers to devote himself to science for which he won the most first prizes. Brilliant student, he was promoted to professor at the age of sixteen and can therefore start his own research. He joined the École Normale Supérieure at the age of 26, where he has as teachers great scientists like Joseph-Louis Lagrange and Pierre Gaspard Monge-Simon Laplace, whom he succeeded to the chair at the École Polytechnique in 1797. Fourier has contributed to the numerical resolution of equations and the diffusion where one of the laws have his name. His work has a direct involvement in the convergence of series and infinite sum. He participated with Monge at the Egypt campaign as scientific observer. Ennobled under Napoleon, he was a professor at the École Polytechnique, secretary of the Institute of Egypt and Prefect of Isère. He was also elected to the Académie des Sciences and at the Academie Française. He is considered as one of the founders, with the French Poisson and the Swiss Daniel Bernoulli to what we now name the "Physics-Mathematics".

\parpic[l][t]{%
  \begin{minipage}{40mm}
    \fbox{\includegraphics[width=110px,height=140px]{img/medaillons/fraunhofer.eps}}
  \end{minipage}
}
\textbf{Fraunhofer, Joseph von} (1787-1826) was  German physicist and optician, born in Straubing. Fraunhofer brought many improvements in the manufacture of optical glass, to grinding and polishing of lenses and to the construction of telescopes and other optical instruments. Fraunhofer Joseph was the eleventh child of a glassblower. He was eleven years when his parents died: they also sent him for a apprenticeship in Munich during six years so that he learns the manufacturing of mirrors. In 1801 he nearly being killed in the collapse of the mirror workshop. At the end of his apprenticeship in 1806, he had the opportunity to continue training as an optician in the Mechanics Institute of Reichenbach. The workshops were transferred in 1807 to Benediktbeuern and Fraunhofer was appointed the foreman. There, he developed new polishing machines mirrors and new types of optical glass (flint achromatic glass), which brought a decisive improvement in the quality of the lenses. Fraunhofer also invented many scientific instruments. His name is associated with fixed and black lines in the solar spectrum named "Fraunhofer lines" that he was the first to describe in detail. His research in the field of refraction and dispersion of light led to the invention and development of the spectroscopy.

\parpic[l][t]{%
  \begin{minipage}{40mm}
    \fbox{\includegraphics[width=110px,height=140px]{img/medaillons/fresnel.eps}}
  \end{minipage}
}
\textbf{Fresnel, Augustin Jean} (1788-1827) was French physicist, founder of modern optics, he proposed an explanation of all optical phenomena in the context of the wave theory of light. He began to realize many experiments on light interference, for which he postulated the concept of wavelength and created the Fresnel integrals. He was the first to prove that two beams of light polarized in different planes have no interference effect. He rightly inferred from this experiment that the vibration of the polarized light is transverse and not longitudinal (such as sound) as we thought before him. In addition, he was the first to produce a circularly polarized light. To explain the propagation of light waves, Fresnel used to the notion of ether, unfortunately inconsistent with other experiments. This theory will be left with relativity, but the so-called "Fresnel relations" are always used today. In the field of applied optics, Fresnel designed levelling lens used to increase the illuminating power of the lighthouses. During his lifetime, the scientific work of Fresnel were known only to a small group of scientists and some of his articles were published only after his death.

\parpic[l][t]{%
  \begin{minipage}{40mm}
    \fbox{\includegraphics[width=110px,height=140px]{img/medaillons/friedmann.jpg}}
  \end{minipage}
}
\textbf{Friedmann, Alexander} (1888-1925) was a Russian and Soviet physicist and mathematician. He is best known for his pioneering theory that the universe was expanding, governed by a set of equations he developed now known as the Friedmann equations.  Alexander Friedmann was born to the composer and ballet dancer Alexander Friedmann (who was a son of a baptized Jewish cantonist) and the pianist Ludmila Ignatievna Voyachek. Friedmann was baptized into the Russian Orthodox Church as an infant, and lived much of his life in Saint Petersburg. Friedmann obtained his degree from St. Petersburg State University in 1910, and became a lecturer at Saint Petersburg Mining Institute. From his school days, Friedmann found an inseparable companion in Jacob Tamarkin, who at the end of his career was one of Brown University's most distinguished mathematicians. Friedmann fought in World War I on behalf of Imperial Russia, as an army aviator, an instructor and eventually, under the revolutionary regime, as the head of an airplane factory. Friedmann in 1922 introduced the idea of an expanding universe that contained moving matter; Belgian astronomer Georges Lemaître would later independently reach the same conclusion in 1927. In June 1925 he was given the job of the director of Main Geophysical Observatory in Leningrad. In July 1925 he participated in a record-setting balloon flight, reaching the elevation of 7,400 [m]. The classic solution of the Einstein field equations that describes a homogeneous and isotropic universe is called the Friedmann–Lemaître–Robertson–Walker metric, or FLRW, after Friedmann, Georges Lemaître, Howard P. Robertson and Arthur Geoffrey Walker, who worked on the problem in 1920's and 30's independently of Friedmann

\phantomsection
\addcontentsline{toc}{section}{G}
\label{sec:G}

\parpic[l][t]{%
  \begin{minipage}{40mm}
    \fbox{\includegraphics[width=110px,height=140px]{img/medaillons/galilee.eps}}
  \end{minipage}
}
\textbf{Galileo, Galilei} (1564-1642) was an Italian physicist and astronomer born in Pisa and at the origin of the scientific revolution of the 17th century. His theories and those of the German astronomer Johannes Kepler served as the basis for the work of British physicist Sir Isaac Newton's law of universal gravitation. His main contribution to astronomy was a considerable improvement (when the technique worked...) of the telescope (which allowed him to make observations that revolutionize the discipline) and the discovery of sunspots, lunar mountains and valleys, the four largest satellites of Jupiter and the phases of Venus. In physics, he discovered the law of falling bodies and projectiles parabolic movements. His studies on the oscillations of the pendulum weight led him to invent the pulsometer. This device enabled pulse measurement and provided a standard time, which did not exist at that time. He also started his studies on falling bodies. In the history of culture, Galileo is the symbol of the battle against the religious authorities for the freedom of research (he had however a very good reputation and good relations with religious people that helped...). In mathematics and physics, he helped to advance the knowledge about the kinematics and dynamics, thus laying the foundations of the mechanical sciences. He is therefore considered as the founder of modern physics.

\parpic[l][t]{%
  \begin{minipage}{40mm}
    \fbox{\includegraphics[width=110px,height=140px]{img/medaillons/galois.eps}}
  \end{minipage}
}
\textbf{Galois, Evariste} (1811-1832) was a French mathematician, who gave his name to a branch of mathematics: Galois theory. His life is so legendary that it is sometimes difficult to distinguish between myth and reality. Starting 1827-1828, the fury of mathematics dominates. Galois reads Legendre, Lagrange, Euler, Gauss, Jacobi. Professor, Louis-Paul-Émile Richard admires the mathematical genius of his student and keeps his copies and entrust him to another of his students: Charles Hermite. This is the time when he published his first article in the\textit{Annales des Mathématiques} of Joseph Gergonne (he proves a theorem on periodic continued fractions). He also wrote a first paper on the theory of equations, sent to the Académie des Sciences, lost by Cauchy... He failed the entrance exam to Polytechnique. Some people say that Galois threw the cloth to erase the chalk at the head of his examiner because of the stupidity of the questions. On the advice of his teacher, Galois entered the Preparatory School (the future École Normale). He wrote the results of his research in a paper - \textit{Requirements for an equation to be solvable by radicals} - to compete for the grand prize of mathematics of the Académie des Sciences. Fourier took the manuscript at home and died shortly after: the manuscript is lost, and the grand prize is awarded to Abel (die the year before), and Jacobi. For political reasons, Galois goes in prison, where he continued his research work. Released in 1832, he fell in love in May 1832 of a woman with whom he broke the same year. It is unclear why, but a duel seems to result a few days later. The night preceding May 29, Galois resume his latest discoveries in a beautiful letter to his friend Auguste Chevalier. From this letter was born the legend that Galois made his major discoveries in one night, caught by the fever of death. On the morning of May 30, Galois, abandoned, severely wounded, is raised by a peasant and leads to the Cochin's Hospital . He died the day after in the arms of his younger brother and was buried in a common grave in the cemetery of Montparnasse. The work of Galois are rediscovered a decade later by Liouville, who announced in 1843 at the Academy of Sciences that he has found in the papers of Galois a solution as accurate as deep to the problem of solvability by radicals. It was only in 1846 that he publishes the texts without adding comments. Starting from 1850, the writings of Galois are finally accessible by the best mathematicians.

\parpic[l][t]{%
  \begin{minipage}{40mm}
    \fbox{\includegraphics[width=110px,height=140px]{img/medaillons/gamow.eps}}
  \end{minipage}
}
\textbf{Gamow, George} (1904-1968) was a Russian-American theoretical physicist, astronomer, cosmologist and scientific author born in Odessa, Ukraine. Gamow came in 1928 in Göttingen, where he uses quantum physics to a develop a quantum theory of alpha radioactivity. Two months later, he joined Niels Bohr in Copenhagen. It makes the idea of an atomic nucleus behaves like a nuclear fluid, model set almost a decade later by Bohr. In 1929, he received a new award and he joined Ernest Rutherford at the University of Cambridge. He develops the idea of tunnelling in order to makes protons interact to obtain nuclei with atomic numbers higher. There he met John Cockcroft, who built shortly after the first particle accelerator, thus achieving validate the Gamow model by a transmutation of lithium. Professor at Washington in 1934, Gamow worked with Edward Teller to formulate the theory of beta decay (1936). Interested by astrophysics, Gamow and Teller give a model of the internal structure of red giant stars (1942). In 1954, interested by biochemistry, he proposed the concept of genetic code determined by the order of the components of DNA. In 1956, he was appointed professor of physics at Boulder (Colorado).

\parpic[l][t]{%
  \begin{minipage}{40mm}
    \fbox{\includegraphics[width=110px,height=140px]{img/medaillons/gauss.eps}}
  \end{minipage}
}
\textbf{Gauss, Carl Friedrich} (1777-1855) was German mathematician who made major contributions to many branches of pure and applied sciences. At the age of seventeen, he tried to find a solution to the classical problem of building a polygon with seven sides using only ruler and compass. He managed to prove the impossibility of this construction and continued his approach by providing methods for constructing polygons with 17, 257, and 65'537 sides. More generally, he proved that the construction, using always only ruler and compass, of a regular polygon with an odd number of sides is possible only if the number of sides is one of the prime numbers 3, 5, 17, 257, and 65'537, or a product of these numbers. For his PhD, he showed that any algebraic equation has at least one root. This theorem, whose proof had resisted to the most famous mathematicians, is still named the "Fundamental theorem of algebra" or "Alembert-Gauss theorem". Gauss then turned his attention to the field of astronomy where he developed a new method for calculating the orbits of celestial bodies, developing a theory of errors of observation known as the least squares method (for "overdetermined" linear systems). In the field of probabilities, his name is attached to the Normal distribution (also named "Laplace-Gauss distribution"), whose is described by the famous bell curve or "Gaussian curve". He also worked in geodesy. With the German physicist Wilhelm Eduard Weber, Gauss did, starting from 1831, extensive research in the field of magnetism and electricity. He also conducts research in optics, particularly lenses systems. To return to mathematics, he was the first, studying the hypergeometric series, to give rigorous conditions of convergence of a series. He studied successful generalizations of the law of quadratic reciprocity and discovered their links with the theory of elliptic functions. His memoir of 1828 on the theory of intrinsic surface was the starting point for a general theory of curved spaces (Riemann's work and successors). He also introduced the arithmetic of Gaussian integers (of the form $a+ib$) based on a geometric representation of complex numbers as points in the plane.

\parpic[l][t]{%
  \begin{minipage}{40mm}
    \fbox{\includegraphics[width=110px,height=140px]{img/medaillons/gibbs.eps}}
  \end{minipage}
}
\textbf{Gibbs, Josiah Willard} (1839-1903) was a physicist and mathematician, born and died in New Haven, Connecticut (after spending almost his entire life as single). Coming from a family of scholars, he studied latin and physics, and he began a career as a professor of mathematical physics at Yale College. He lived successively in Paris and Berlin where he took lessons from Heinrich Gustav Magnus and Heidelberg and met Gustav Kirchhoff and Helmholtz Herman Ludwig. Gibbs will be remembered as a scholar of proverbial modesty and with extraordinary power of scientific investigation. His work was first remarkably compact and little known. Today it is considered as a monumental scientific contributions in the 19th century. The two main publications dating from 1877 and 1902. The first is titled \textit{On the Equilibrium of Heterogeneous Substances} and compared in importance to weighted chemistry created by Antoine Laurent Lavoisier. The second, still considered more original, is titled \textit{Elementary Principles in Statistical Mechanics}, and compared, for it's genius, to the analytical mechanics of Joseph Louis Lagrange. Although Gibbs papers are distinguished by exceptional clarity, and how the basic idea is always carefully presented, the first of two papers hardly retained first the attention of chemists of his time, unaccustomed to rigorous language sciences. The wealth of thermodynamic methods on which it relies has however defined the foundations of a unified basis of physico-chemical theory of equilibrium states and their stability. Most of the laws that relate to this discipline, which first bore other names were later rediscovered in it's first memory. This is, for example, "the law of phases" giving the variance of equilibrium systems, long attributed to Bakkuis Roozeboom (laws also named "Van't Hoff law" or "Le Chatelier's law"), on the displacements of equilibrium at a constant temperature and constant pressure. It is still the same with the stability criteria of balance, or the moderation theorem also named "theorem of Le Chatelier and Braun". In short, most of the properties that are present in chemical thermodynamics equilibrium states, such as osmotic pressure, the influence of surface tension, the elastic deformation, the Law on the entropy of the gas mixtures and the associated Gibbs paradox have the same memory for origin. Gibbs developed in two previous communications to the previous one, a complete diagrams and thermodynamic surfaces catalogue which contributed to the spread of their employment by practitioners. Gibbs theory used for the first time the notion of a set as well as the distinction between a canonical set and a microcanonical set and between a large and a small set. Gibbs theory also introduces the concept of phase space, characterized by the coordinates and momenta of each element. This theory also establishes, from the Liouville equation, the law of conservation element extension phase, as well as density and probability of the statistics sate. He finally achieves a remarkable formal agreement with macroscopic laws of thermodynamics governing the behaviour of material media in equilibrium. Current developments in statistical mechanics are still on more than one point, extensions of the method of Gibbs. He also defined for chemical reactions two useful quantities, namely the "enthalpy" that is the heat of reaction at constant pressure and the "free energy" that determines whether a reaction can proceed spontaneously at room temperature and constant pressure. This latter quantity is now named "Gibbs energy" in his honour. Gibbs seems to be at origin of the usage to designate the scalar product with a dot, the vector cross product with a St Andrew cross $\times$ and the adoption of nabla and del vector differential operators.

\parpic[l][t]{
  \begin{minipage}{40mm}
    \fbox{\includegraphics[width=110px,height=140px]{img/medaillons/godel.eps}}
  \end{minipage}
}
\textbf{Gödel, Kurt} (1906-1978) born in Brünn and died at Stanford, was a mathematician and logician, that in the entire 20th century, has the most revolutionized the logical foundations of mathematics. He was a man so obsessed with the logic that when he tried to get his American citizenship, he dared to show the judge the contradiction of some articles of the constitution of the United States. His thesis, and especially an article published in 1931 under the title \textit{Über formal unentscheidbare Sätze und der Principia Mathematica verwandter System} (About the undecidability of formal Principia Mathematica and similar systems), will give Gödel an international reputation. Gödel puts an end to hopes of Hilbert about a complete axiomatized mathematics system, and to make of mathematics a field where only mechanical deductions are possible leaving no place for intuition. Thus, Gödel shows that there are true propositions about integers, but that they can not be proved. He shows that even if we add other axioms, there will always be true undecidable propositions (we can not prove). He shows in particular that the continuum hypothesis and the axiom of choice is not in contradiction with the other axioms of set theory. Then he turned to relativity, being directly related to Princeton with his friend Einstein. He is known to physicists as having demonstrated that travel to the past is possible within the framework of the equations of General Relativity.

\parpic[l][t]{%
  \begin{minipage}{40mm}
    \fbox{\includegraphics[width=110px,height=140px]{img/medaillons/goeppertmayer.eps}}
  \end{minipage}
}
\textbf{Göpper-Meyer, Maria} (1906-1972) was a German-born American physicist, Nobel Prize in 1963 for her study of nuclear structure. She was married to the physicist Joseph Mayer, specialized in solid state physics (1904-1983). But in this couple, each worked separately in his speciality. Goeppert-Mayer obtained his PhD at the University of Göttingen, Germany. She taught in many institutions before returning to the University of California at San Diego in 1960. In 1963 she shared with H.D. Jensen and  E.Wigner the Nobel Prize in physics and was cited by the Nobel committee for his independent work in the late 1940s. She proved that the nucleus has a number of neutrons and protons well defined: she introduced a structural model of the atomic nucleus in layers. This model, developed in detail in 1948 assumed that the strong interaction between the intrinsic rotation (quantified by the spin) of nucleons and their orbital motion was responsible for the structure of the energy levels of the nuclei. Many consequences deduced from this hypothesis were verified by experimental measurements. A few years later, James Rainwater, Aage Bohr and Ben R. Mottelson (all three Nobel Prize in Physics 1975) completed the theory taking into account the coupling between the motion of the nucleons in the outer layer and the collective motion of the nuclear core.

\parpic[l][t]{%
  \begin{minipage}{40mm}
    \fbox{\includegraphics[width=110px,height=140px]{img/medaillons/gosset.eps}}
  \end{minipage}
}
\textbf{Gosset, William Sealy} (1876-1937) known under the pseudonym "Student" was an English statistician. Employee of the Guinness brewery to stabilize the flavour of the beer, he invented the $T$-test used actually as a standard in many fields of industry or the economy. He also determined in 1908 the experimental distribution he obtained through his job and after taking a statistics course with Karl Pearson, he obtained his famous result that he published under the pseudonym "Student" with the law that bears and test that still bear his name today.\\

\parpic[l][t]{%
  \begin{minipage}{40mm}
    \fbox{\includegraphics[width=110px,height=140px]{img/medaillons/gottlob.eps}}
  \end{minipage}
}
\textbf{Gottlob, Frege Friedrich Ludwig} (1848-1925) was a German mathematician and philosopher, founder of modern logic. Frege was born in Wismar in 1848, and was educated at the universities of Jena and Göttingen where he received his PhD in philosophy in 1873. From 1879 to 1917 he was professor at the Faculty of Philosophy in Jena. His work focuses on particular mathematical logic and its applications. Faced with the ambiguity of ordinary language and imperfect logic systems available, he invented many symbolic notations, such as quantifiers and variables, then putting the foundations of modern mathematical logic. He is the first to have presented a coherent theory of predicate calculus and the propositional calculus. He was also the first to derive the logical arithmetic. He defined in particular the following integers from the empty set, by applying a few simple rules.

\parpic[l][t]{%
  \begin{minipage}{40mm}
    \fbox{\includegraphics[width=110px,height=140px]{img/medaillons/grothendieck.eps}}
  \end{minipage}
}
\textbf{Grothendieck, Alexander} (1928-2014) was born in Berlin, his father was a Russian anarchist who was killed by the Nazis, and his mother a woman of letters refugee in France. He obtained his Bsc. at the Faculté de Montpellier, then spent a year in 1948-1949 at the École Normale Supérieure in Paris, before moving in 1949 to the Université de Nancy. He became there a student of Schwartz and Dieudonné in functional analysis. Dieudonné feel that Grothendiek was a bit pretentious, and also asked him to work on issues that neither Schwartz have solved. Here is what Schwartz says in his autobiography: "Dieudonné, with the aggression (always passing), of which he was capable, reprimanded Grothendiek, arguing that he should not work this way, by generalizing just for the pleasure to generalize. [...] The article ended with 14 questions, open problems that we had not been able to resolve, I and Dieudonné. Dieudonné proposed to Grothendieck to consider some of the problems that he would choose. We never saw him again during a few weeks. When he reappeared he had found the solution of half of them!". Quickly, Grothendieck wrote his thesis on Topological tensor products and nuclear spaces, and became a worldwide specialist in the theory of topological vector spaces. He also became a member of the famous Bourbaki group. In the early 1960s, he gets a function at the recent Institute of Advanced Scientific Studies (IHES), and his focus is directed towards on algebraic geometry. There he made gigantic works, which earned him the Fields Medal in 1966. However, Grothendieck refuses to go to the USSR to receive the prize, to protest against the repression of the Hungarian uprising in 1956. The Fields Institute gives him the Fields medal later, but Grothendieck offers it to Vietnam to use its gold. He also teaches there several weeks under the American bombing. In the late 60s, Grothendieck, who lost the habit of writing (Dieudonné wrote during his seminary years), becomes less and less clear. He will never forgive other mathematicians do not understand him and distorted his ideas. If his relations with the mathematical community had never been easy (he worked a lot alone, his days were 27 or 28 hours, so that sometimes he was shifted - He despised slightly Dieudonné, sequel of the first reprimand - his disputes with Weil caused his departure from Bourbaki ...), they are more strained than ever ... He gradually abandoned mathematics and the IHES after a dispute over military funding in 1970, to retire to his home in the Hérault, where he devoted himself to meditation and ecology. In 1985 he wrote a sort of autobiography that was not published. Those who have read it are unanimous in saying that it contained many attacks against the community of mathematicians.

\phantomsection
\addcontentsline{toc}{section}{H}
\label{sec:H}

\parpic[l][t]{%
  \begin{minipage}{40mm}
    \fbox{\includegraphics[width=110px,height=140px]{img/medaillons/hall.eps}}
  \end{minipage}
}
\textbf{Hall, Edwin Herbert} (1855-1938) was a physicist born in the Main and died in Cambridge (U.S.A.). Hall did his undergraduate work at Bowdoin College, graduating in 1875. He did his graduate schooling and research, and earned his PhD degree (1880) at the Johns Hopkins University where his experiments were performed. The Hall effect was discovered by Hall in 1879, while working on his doctoral thesis in Physics. Hall was appointed as Harvard's professor of physics in 1895. He was notable for lecturing without shoes and wrote numerous books on physics.\\

\parpic[l][t]{%
  \begin{minipage}{40mm}
    \fbox{\includegraphics[width=110px,height=140px]{img/medaillons/hamilton.eps}}
  \end{minipage}
}
\textbf{Hamilton, William Rowan} (1805-1865) was an Irish mathematician, physicist and astronomer (born and died in Dublin) who was the object during his lifetime of the highest honours and was called the "Irish Lagrange" and even "Irish Newton" yet his work was little known and rarely studied. He is known for his discovery of quaternions, but also contributed to the development of optics, dynamics and algebra. His research was important for the development of quantum mechanics. The mathematical work of Hamilton include the study of geometrical optics, the adaptation of dynamic methods for optical systems, applying quaternion and vector problems of mechanical and geometric possibilities of solving polynomial equations, including the general equation of the fifth degree, linear operators, for which he proves a result for these operators in the space of quaternions, which is a special case of Cayley-Hamilton theorem. His scientific career was predestined by its studies at Trinity College, in Dublin, where, at the age of nineteen, he finished a remarkable job on the lens. At the age of 23 years, he became professor of astronomy at Dublin and Royal Astronomer at Dunsink Observatory where he will stay for the rest of his life. Hamilton tries to provide the fundamental principles of mechanics to a simple form to build a deductive theory. To do this, he modifies the principles of previous variations, including the principle of least action, and introduced what is named today the "Hamilton's principle". Finally, we note that he is at the origin of the "canonical" expression of the equations of the dynamics that brings nothing new to physics but provides a more powerful method for solving the equations of motion. In his work of the years 1832 to 1835 Hamilton attaches a great importance to the geometric interpretation of complex numbers, and it is from there that one seeks to interpret algebraic calculation in three-dimensional space. He arrives at this goal in 1843, building the quaternions. In the years following this discovery, he devoted himself to his development and its dissemination, by finding applications in various fields of mathematics and physics. The Hamilton's quaternions are one of the first vector systems and, through their theoretical consequences, contributed significantly to the development of algebra and quantum physics in the 20th century.

\parpic[l][t]{%
  \begin{minipage}{40mm}
    \fbox{\includegraphics[width=110px,height=140px]{img/medaillons/hawking.eps}}
  \end{minipage}
}
\textbf{Hawking, Stephen} (1942-2018) born in Oxford, was a British theoretical physicist and cosmologist. Just as Albert Einstein, Hawking was not particularly brilliant at the high school, but his taste for the physical sciences leads him to the University of Oxford, a place of relative boredom to where he exits with honours. After receiving his B.A. degree at Oxford in 1962, he stayed to study astronomy. He decided to quit when he found that studying sunspots was not attractive and that he was more interested in theory than by observation. He left Oxford with honours, for Trinity Hall, where he took part in the study of theoretical astronomy and cosmology theory. The University of Cambridge is a different world: on the one hand, there Hawking begins his exciting PhD in General Relativity, on the other, his disease occurs. Despite this difficulty, the study of singularities, allows the researcher to develop theories that will lead him to the Big Bang and Black Holes theory. First, Roger Penrose and Hawking build the mathematical structure answering the question of a singularity as the origin of the Universe. Then, starting from the 1970s, Hawking deepened his research on local infinite densities, and his studies on Black Holes have advanced many other areas. Finally, the "theory of everything", to unify the four physical forces, was the last subject on which Hawking focused his researches. The aim is to demonstrate that the universe can be described by a mathematical stable model, determined by the known physical laws, the principle of finite growth but not limited, model for which Hawking gave a lot of credit. His severe handicap can not alone explain the great success of his research, Hawking has tried to popularize his work, and his book A Brief History of Time is one of the most successful scientific literature. In 2001, he released his second book, The Universe in a nutshell that explains the latest state of his thoughts, addressing supergravity and supersymmetry, quantum theory and M-theory, holography and duality theory superstring and $p$-branes ... He also wondered about the possibility of time travel and the existence of multiple universes.

\parpic[l][t]{%
  \begin{minipage}{40mm}
    \fbox{\includegraphics[width=110px,height=140px]{img/medaillons/hausdorff.eps}}
  \end{minipage}
}
\textbf{Hausdorff, Felix} (1868-1942) The reputation of the German mathematician Felix Hausdorff is mainly based on his book \textit{Grundzüge der Mengenlehre} (1914), who made him the founder of the topology and the theory of metric spaces. Born in Breslau in a wealthy merchant family, Hausdorff followed a high-school education in Leipzig. After high-school he studied mathematics and astronomy at Leipzig, Freiburg im Breisgau and Berlin. In 1891, he obtained his PhD in Leipzig and taught there from 1896 to 1902. Throughout this time, Hausdorff, while publishing several papers on astronomy, optics and mathematics, was particularly interested in philosophy, literature and art. From 1910 to 1935 he was professor of mathematics at the University of Bonn, with the exception of the years 1913 to 1921, where he taught at Greifswald. Since his forced retirement in 1935, the work of Hausdorff were no longer published in Germany. Jewish, Hausdorff risked the concentration camp, when internment became imminent in 1942, he committed a suicide in Bonn with his wife and sister in law. Hausdorff contributions to the development of mathematics lie in several areas. His study of the series led to the demonstration of theorems on methods of summation and Fourier coefficients (1921). Considering the properties of digital sets, he introduced an important class of measures. He studied in the general theory of sets, partially ordered sets and several theorems on ordered sets (1906-1909). In descriptive set theory, he demonstrated the theorem on the cardinality of Borel sets (1916). Apart isolated but deeps results in topology and set theory, Hausdorff had especially in his \textit{Grundzüge der Mengenlehre} laid the foundations of a discipline. Hausdorff developed a theory of topological spaces and metric encompassing perfectly the previous results. He chose to build his theory of abstract spaces on the notion of neighbourhood. He added many new results in the theory of metric spaces, the most profound is the theorem stating that every metric space can be extended in a unique way to a complete metric space. Hausdorff was a methodic teacher, but his courses, with their rich content and rigorously structured, passed above the level of his listeners.

\parpic[l][t]{%
  \begin{minipage}{40mm}
    \fbox{\includegraphics[width=110px,height=140px]{img/medaillons/heaviside.eps}}
  \end{minipage}
}
\textbf{Heaviside, Oliver} (1850-1925) was born in the city of Camden in London (England) and died in Torquay in Devon (England). This is where he lived the last twenty five years of his life. He comes from a family quite poor. He caught scarlet fever when he was a toddler, which affected his hearing, he remained partially deaf. This has had an impact on his life making difficult his childhood especially in relationships with other children. He compensated by shyness and sarcasm. However, despite all this, his academic performance was rather high. One can even say that a sixteen years old he was a top student, but he failed in Euclidean geometry. He hated having to infer a fact of another. The primacy of rigorous proof in arithmetic, an idea strongly disliked by Heaviside, made him the subject where he was the weakest. Although he stopped his studies at sixteen, he continued to learn by himself. He learned Morse code, studied electricity and other languages in particular Danish and German. He was self-taught. In 1868, after leaving his studies, Heaviside went to Denmark and became a telegraph operator. He progressed rapidly in his profession and he returned to England in 1871. It is his work that led him to study electricity. He then read the new treaty of Maxwell on electricity and magnetism. After reading this treatise, he made changes in his life. He stopped working and he locked himself in a room of the family home to work on Maxwell's theory. Heaviside reduced Maxwell's theory, and it is from this time that the electrical theory took its modern form. Maxwell has written twenty equation with twenty variables. Heaviside reduced the twenty equations into four equations with two variables. Today, we name these equations "The four Maxwell equations", forgetting that they are in fact the "Heaviside equations". However, it was Hertz who got the credit for it, but he admits that his ideas came him from Heaviside.

\parpic[l][t]{%
  \begin{minipage}{40mm}
    \fbox{\includegraphics[width=110px,height=140px]{img/medaillons/heisenberg.eps}}
  \end{minipage}
}
\textbf{Heisenberg, Werner Karl} (1901-1976), born in Würzburg and died in Munich was a German physicist. He was the founder of the rigorous theoretical concepts of quantum mechanics. He was awarded the Nobel Prize for Physics in 1932. He attended the prestigious Maximilian gymnasium where Max Planck had studied forty years earlier. At the age of twelve, he began to learn integral calculus and later fascinated by mathematics, he followed as free auditor followed several courses at the University of Munich, including mathematical methods of modern physics. He completed his studies in physics in the record time of three years, and defended his thesis (that he almost missed because of gaps in basic experimental physics) under the direction of Arnold Sommerfeld, with whom he developed a theory explaining the anomalous Zeeman effect at the age of twenty years, which attracted on him the attention of major European physicists (he was regarded as brilliant as Pauli himself who was already considered most brilliant than Einstein). From 1924 he became the assistant of Max Born in Göttingen and he worked with Niels Bohr in Copenhagen. It was during the following years with Max Born and Pascual Jordan, that he threw the theoretical foundations of quantum mechanics. Heisenberg was recruited in 1927 as a professor at the University of Leipzig aged only twenty six. He made of this University one of the highest places of theoretical physics (especially nuclear physics) in Europe. He developed the first formalization of quantum mechanics, in 1925, Erwin Schrödinger at the same time. However, the mathematical formalism was different. Heisenberg adopted a complex matrix formulation (although he did not know what was a matrix as most physicists of his time...) from which emerged naturally the non-commutativity while Schrödinger used an approach based on differential equations (simple wave equation). For this reason we thought at first that the two theories were distinct but the following year, Schrödinger establishes the mathematical equivalence of the two formulations. His uncertainty principle, discovered in 1927, says that the determination of certain pairs of values, such as position and momentum, can not be done with infinite precision. From 1929, he worked with Wolfgang Pauli in the development of quantum field theory. After the discovery of the neutron by James Chadwick in 1932, Heisenberg proposed the proton-neutron model of the atomic nucleus, and used it to explain the nuclear spin isotopes.

\parpic[l][t]{%
  \begin{minipage}{40mm}
    \fbox{\includegraphics[width=110px,height=140px]{img/medaillons/hemlholtz.eps}}
  \end{minipage}
}
\textbf{Helmholtz, Hermann Ludwig Ferdinand von} (1821-1894) was born in Potsdam and died in Berlin. There is no a field of science for which Helmholtz has not made some research. We could also say about him what he says himself about Friedrich von Humboldt in his famous inaugural lecture of scientific the symposium of Innsbruck (on the purpose and progress of the science of Nature, 1869): «He managed to dominate all of the natural sciences of his time and penetrate until each of their specialities». Even if Helmholtz said that in the second half of the 19th century that encyclopaedic knowledge is now impossible, and we must resign ourselves to focus in a defined area, just take a look at all of his work to see that it was concerned with matters as diverse as thermodynamics, hydrodynamics, electrodynamics and the theory of electricity, physical meteorology, physiology, and especially the theory of acoustic and physiological optics. Having a remarkable gift for the popularization of the latest scientific findings, he wrote numerous articles and delivered many lectures where scientists subjects were presented side by side with popular aesthetic or philosophical concerns. His name is mainly linked with the formulation of the principle of conservation of energy, even if some of his assertions may seem as uncompromising mechanism and were able to give him the reputation of the last taking of Galilean physics. His name is also linked to some notable inventions like the ophthalmoscope or spherical resonators. At the end of his life, Helmholtz recognize the importance and universality of another physical principle, the "principle of least action", that he will apply, in particular, in electrodynamics.

\parpic[l][t]{%
  \begin{minipage}{40mm}
    \fbox{\includegraphics[width=110px,height=140px]{img/medaillons/henry.eps}}
  \end{minipage}
}
\textbf{Henry, Joseph} (1797-1878) was a scientist born in New-York and died in Washington. His parents were poor, and Henry's father died while he was still young. For the rest of his childhood, Henry lived with his grandmother in New York. He attended a school which would later be named the Joseph Henry Elementary School in his honour. After school, he worked at a general store, and at the age of thirteen became an apprentice watchmaker and silversmith. His interest in science was sparked at the age of sixteen by a book of lectures on scientific topics titled \textit{Popular Lectures on Experimental Philosophy}. In 1819 he entered The Albany Academy, where he was given free tuition. He was so poor, even with free tuition, that he had to support himself with teaching and private tutoring positions. He intended to go into the field of medicine, but in 1824 he was appointed an assistant engineer for the survey of the State road being constructed between the Hudson River and Lake Erie. From then on, he was inspired to a career in either civil or mechanical engineering. Henry excelled at his studies (so much, that he would often be helping his teachers to teach science) that in 1826 he was appointed Professor of Mathematics and Natural Philosophy at The Albany Academy. Some of his most important research was conducted in this new position. His curiosity about terrestrial magnetism led him to experiment with magnetism in general. He was the first to coil insulated wire tightly around an iron core in order to make a more powerful electromagnet. While building electromagnets, Henry discovered the electromagnetic phenomenon of self-inductance. He also discovered mutual inductance independently of Michael Faraday, since Faraday published his results first, he became the officially recognized discoverer of the phenomenon. Using his newly-developed electromagnetic principle, Henry in 1831 created one of the first machines to use electromagnetism for motion. This was the earliest ancestor of modern DC motor. The SI unit of inductance, the henry, is named in his honour.

\parpic[l][t]{%
  \begin{minipage}{40mm}
    \fbox{\includegraphics[width=110px,height=140px]{img/medaillons/hermite.eps}}
  \end{minipage}
}
\textbf{Hermite, Charles} (1822-1901) was born in Dieuze, he published his first work while he was still a student at the École Polytechnique, and at the age thirteen, he was already considered one of the best mathematicians of his time. He was successively professor at the École Polytechnique, the Collège de France and then at the Sorbonne in 1869, where his teaching and his voluminous correspondence had a considerable influence. He lived in Paris until his death. He was elected member of the Academie des Sciences at the age of thirty-four years. In algebra, Hermite took an active part in the early development of the theory of invariants, initiated by Arthur Cayley and James Joseph Sylvester, he completed, among others, the determination of invariants of binary forms of the fifth degree, begun by Sylvester, and discovered the law of reciprocity between covariants of binary forms of various degree. He is also at the origin of improvement of the Lagrange's interpolation method taking into account the values of the first derivatives, and of the discovery of the family of orthogonal polynomials that bear his name. The analytical work of Hermite is marked by his algebraist temperament. His favourite subject throughout his life was the theory of elliptic functions and abelian functions, which he loved to explore the hidden links with algebra and number theory. One of the results that most struck his contemporaries is the solution of equation of the fifth degree using elliptic functions. His virtuosity in the calculation of functions allowed him to directly obtain the remarkable relations on class numbers of quadratic ideals that Kronecker had derived from the complex multiplication. He was a pioneer in the study of Abelian functions, where he developed the theory of transformation and met on this occasion for the first time the symplectic group. Finally, the most famous of Hermite memories is when, in 1872, he proved the transcendence of the number $e$, using results of his research on algebraic continued fractions, and his method has remained almost the only one available today to address issues of transcendence.

\parpic[l][t]{%
  \begin{minipage}{40mm}
    \fbox{\includegraphics[width=110px,height=140px]{img/medaillons/hertz.eps}}
  \end{minipage}
}
\textbf{Hertz, Heinrich Rudolf} (1857-1894) was a German physicist born in Hamburg and died in Bonn. He studied at the Berlin Universität. In 1879, he was the student of Gustav Kirchhoff and Hermann von Helmholtz at the Berlin Physik Institut. He became a lecturer at the University of Kiel in 1883 where he conducts research on electromagnetism. From 1885 to 1889, responsible for wireless telegraphy, he was professor of physics at the Karlsruhe Teknische Schule, and from 1889, professor of physics at the Bonn Universität. Hertz clarified and expanded the electromagnetic theory of light proposed by the English physicist James Maxwell in 1884. He proved that the power could be transmitted by electromagnetic waves which travel at the speed of light and have many other properties of the light. His experiments with these waves led to the development of wireless telegraphy and radio. The unit of frequency, one period per second, was named the "Hertz".

\parpic[l][t]{%
  \begin{minipage}{40mm}
    \fbox{\includegraphics[width=110px,height=140px]{img/medaillons/hilbert.eps}}
  \end{minipage}
}
\textbf{Hilbert, David} (1862-1943) was born in Königsberg, and died in Göttingen. He was a student under the supervision of Lindemann with whom he obtained his PhD in 1885 and he was a comrade of Herman Minkowski, with whom he remained bound by a deep friendship. Although mathematical interests of Hilbert were vast, he preferred to work on one subject at a time. His main areas of interest were: until 1892, the algebraic theory of invariants, from 1892 to 1899 the theory of algebraic numbers, from 1899 to 1905, the calculus of variations, from 1901 to 1912, integral equations, 1912 in 1917, the mathematical foundations of physics. Around 1910, Hilbert supports the efforts of Emmy Noether, mathematician of the first order, who wishes to teach at the Göttingen Universität. To circumvent the system established against women, Hilbert lends its name to Noether who can announce the schedule of his course without damaging the reputation of the university. From 1917 until the end of his life he devoted himself to mathematical logic. He gave a decisive impetus to the development of research on the foundations of mathematics. During the International Congress of Mathematics in 1900, Hilbert presented a list of twenty-three problems many of which remain unresolved today. He adopted and vigorously defended the ideas of Georg Cantor set theory and transfinite numbers. He is also known as one of the founders of proof theory, mathematical logic and clearly distinguished mathematics of meta-mathematics. He is considered by many as on of the greatest mathematician of the 20th century.

\parpic[l][t]{%
  \begin{minipage}{40mm}
    \fbox{\includegraphics[width=110px,height=140px]{img/medaillons/hotelling.eps}}
  \end{minipage}
}
\textbf{Hotelling, Harold} (1895-1973) was a mathematical statistician and an influential economic theorist, known for Hotelling's T-squared distribution in statistics. He was Associate Professor of Mathematics at Stanford University from 1927 until 1931, a member of the faculty of Columbia University from 1931 until 1946, and a Professor of Mathematical Statistics at the University of North Carolina at Chapel Hill from 1946 until his death. A street in Chapel Hill bears his name. In 1972 he received the North Carolina Award for contributions to science. Hotelling is known to statisticians because of Hotelling's T-squared distribution which is a generalization of the Student's t-distribution in multivariate setting, and its use in statistical hypothesis testing and confidence regions. He also introduced canonical correlation analysis. In the United States, Harold Hotelling is known for his leadership of the statistics profession, in particular for his vision of a statistics department at a university, which convinced many universities to start statistics departments.

\parpic[l][t]{%
  \begin{minipage}{40mm}
    \fbox{\includegraphics[width=110px,height=140px]{img/medaillons/hoyle.eps}}
  \end{minipage}
}
\textbf{Hoyle, Fred} (1915-2001) was born in Bingley, Yorkshire and died in Bournemouth. Hoyle studied mathematics and theoretical physics at Cambridge from 1933 to 1939. When hostilities started, he enlisted in the Royal Navy and worked on the development of radar at the Witley's research center. There he met the two physicists Hermann Bondi and Thomas Gold. All three passionate of cosmology, they consider with scepticism the standard model of the universe (in a philosophical point of view it was unacceptable). At the time, the standard model stumbled over a serious difficulty: an estimated Hubble age of the universe was about 2 billion years, yet geological data led to a age of the Earth at least 4 billion years. During the war and in the years following the end of hostilities, Hoyle published several studies on the theory of accretion and the theory of stellar structure, especially for giant stars and white dwarfs. The war ended, the three men returned to Cambridge, where Hoyle gets a chair of mathematics. In 1948, they expose their theory in two articles, one of Bondi and Gold, the other of Hoyle. In 1963, the first quasar is discovered. Its intrinsic brightness is much higher than any other celestial object known: it is a hundred times more luminous than any galaxy! In 1962, Hoyle and William A. Fowler had proposed a theory that could account for the huge luminosity of quasars, it was the theory of supermassive stars. Theoretical considerations can demonstrate that normal stars of masses greater than about 60 solar masses would be the seat of violent instability due to radiation pressure and the generation of nuclear energy. This hypothesis is supported by the fact that we do not observe normal stars beyond the limit of instability. Despite this argument, Hoyle and Fowler proposed the concept of a supermassive star, star that would be maintained almost entirely by radiation pressure. Thus, to achieve the brightness characteristic of a quasar, the supermassive star must have a mass of about 100 million solar masses. When the density becomes sufficiently high, a supermassive star of less than 1 million solar masses explode, while a more massive star undergoes a cataclysmic collapse and form a supermassive black holes. These two possibilities are very important for understanding quasars, and they have been studied by many researchers. Another explanation for the quasar phenomenon, suggested for the first time by Donald Lynden-Bell, suppose the accretion of matter into a supermassive black hole at the center of a galaxy (consensus currently adopted by the scientific community).

\parpic[l][t]{%
  \begin{minipage}{40mm}
    \fbox{\includegraphics[width=110px,height=140px]{img/medaillons/huygens.eps}}
  \end{minipage}
}
\textbf{Huygens, Christian} (1629-1695) was a Dutch astronomer, mathematician and physicist. Many scientific original discoveries earned him widespread recognition and honours among the eminent scientists of the 17th century. With his \textit{Treatise on Light} (1690), he is at the origin of the wave theory of light (which later took its name): each point of wave motion is itself a source of new waves. He studies in 1652 the rules set by Descartes in the \textit{ Principia philosophiae}. Building on the Cartesian conservation of momentum $mv$, he had the idea to solve algebraically the problem of shocks by comparing the quantities $mv^2$ which are introduced only for the harmony of calculations, without particular physical meaning. While discovering these quantities are conserved before and after the shock, he can write the rules in the general case, that Descartes had not done so, including conservation of momentum and kinetic energy. In 1655, he invented a method of grinding and polishing of optical lenses. The finer definition obtained allowed him to discover a satellite of Saturn and to provide the first accurate description of the rings of Saturn. The need for an accurate measurement of time for the observation of the sky led him to apply the laws of the pendulum to adjust the movements of clocks and watches. In 1656, he designed a telescope that bears his name. Between 1658 and 1659, Huygens worked on the theory of pendulum. He has indeed the idea of regulating clocks with a pendulum to make the most accurate measurement of time. He discovered the rigorous isochronism formula in 1659 when the extremity of the pendulum travels an arc of cycloid, the period of oscillation is constant regardless of the amplitude. In \textit{Horologium Oscillatorium} (1673), he determined the true relationship between the length of a pendulum and the period of oscillation, and presented his theories on centrifugal force of circular motions, which helped the English physicist Isaac Newton to formulate the laws gravity. In 1673, Huygens and his young assistant Denis Papin, highlight the principle of internal combustion engines, which will lead to the nineteenth century with the invention of the automobile. In 1678, he found the polarization of the light by the birefringent calcite.

\phantomsection
\addcontentsline{toc}{section}{I}
\label{sec:I}

\parpic[l][t]{%
  \begin{minipage}{40mm}
    \fbox{\includegraphics[width=110px,height=140px]{img/medaillons/ibn.eps}}
  \end{minipage}
}
\textbf{Ibn Al Haytham} (965-1039) was an Arabic mathematician, philosopher and physicist. He is one of the fathers of quantitative physics and modern optics, the pioneer of the modern scientific method and the founder of experimental physics and some, for these reasons, described him as the first scientist. Al Haytham began his scientific career in his hometown of Basra. However, he was summoned by the caliph Hakim who wanted to control the flooding of the Nile that beated Egypt year after year. After leading an expedition in the desert to trace the source of the famous river, Al Haytham realized that this project was impossible. Back to Cairo, he feared that the caliph, who was furious at his failure, avenge and so he decided to feign madness. The caliph only assigned Al Haytham at residence. Al Haytham took advantage of this enforced leisure to write several books on various subjects such as astronomy, medicine, mathematics, scientific method and optics. The exact number of his writings is not known with certainty, but there is an approximation of a number between 80 and 200. Few of these works, in fact, have survived until today. Some of them, those on cosmology and his treatises on optics in particular, have survived only thanks to their translation to latin. Most of his research involved geometrical optics and physiology. Contrary to the popular belief, he was the first to explain why the sun and moon appear bigger (it was long believed that it was Ptolemy), he also establishes that the light of the moon comes from the sun. He also contradicted Ptolemy about that the eye emit light. For Al Haytham, if the eye really emit light we could see at night. He realized that the sunlight was diffused by the object and then entered the eye. In astronomy he attempted to measure the height of the atmosphere and found that the phenomenon of twilight is due to a phenomenon of refraction. He also spoke on the attraction of masses and it is believed that he knew the gravitational acceleration. Al Haytham was ahead a few centuries on several discoveries made by occidental scientists during the Renaissance. He was one of the first to use a scientific method of analysis and greatly influenced scientists like Roger Bacon and Kepler.

\phantomsection
\addcontentsline{toc}{section}{J}
\label{sec:J}

\parpic[l][t]{%
  \begin{minipage}{40mm}
    \fbox{\includegraphics[width=110px,height=140px]{img/medaillons/jacobi.eps}}
  \end{minipage}
}
\textbf{Jacobi, Carl} (1804-1851) was born in Potsdamand and died in Berlin (Germany). He was with N. H. Abel, the founder of the theory of elliptic functions for which he gave many applications in the most various fields of mathematics. We also owed him papers on theoretical mechanics where he goes back on the results of W. R. Hamilton, and applications of the theory of differential equations to dynamics. When Jacobi entered the High school in 1816, he already completed alone the graduate level, quite refractory to traditional teaching, he studied directly works of the great mathematicians, particularly those of Euler and Lagrange. Registered in 1821 at the University of Berlin, he learned philology and mathematics, to which he devoted himself almost completely. In 1825 he obtained a PhD with a thesis in which he generalized some Lagrange's formulas. He taught in Berlin for about a year, then Koenigsberg where he was transferred by ministerial decision. End of 1827, he was appointed extraordinary professor at the Vienna University, where he came into contact with the astronomer Friedrich Wilhelm Bessel. Pensioned by the Prussian government, he was, after a trip to Italy in 1843, named Academician in Berlin, exempt from teaching but authorized to work on any subject that interested him. Presented as a candidate to the elections in 1848, he was persecuted for a time for his liberal views. Jacobi devoted many works to integrals transformations and brought an important contribution to the theory of differential equations and partial differential equations. This is to what are attached his contributions to the calculus of variations, the dynamics of solids and celestial mechanics - three-body problem, perturbations of planetary motion. Algebra owes him important research on quadratic forms and a relation with the theory of determinants that has become a classic, a prelude to his memory on the functional determinants named nowadays "Jacobians". He perfected the theory of elimination and teaches to represent the roots of an algebraic equation by definite integrals or series. He studies the common points between curves and algebraic surfaces, and found directly the number of double tangents of a plane curve, already established by J. Plücker using duality.

\parpic[l][t]{%
  \begin{minipage}{40mm}
    \fbox{\includegraphics[width=110px,height=140px]{img/medaillons/jordan_camille.jpg}}
  \end{minipage}
}
\textbf{Jordan, Camille} (1838-1921) was born in Lyon and died in Paris (France). He was the undisputed specialist in the theory of groups throughout the late 19th century and we owe him numerous famous results, both on finite groups as the so named "classics groups", which he was the first to measure the importance. His analysis courses contributed to the development of the theory of functions of a real variable. In 1855, at the age of seventeen, he is received as best student at the École Polytechnique and finished the École des Mines in 1861. He will be, at least officially, engineer responsible for overseeing the Carrières de Paris until 1885, which will not prevent him from an intense mathematical research. Appointed examiner at the École Polytechnique in 1873 and professor in 1876, he entered the Académie des Sciences in 1881 and succeeded to Joseph Liouville at the Collège de France two years later. From 1885 to 1921, he assumed the management of the \textit{Journal de Mathématiques pures et appliquées} founded by Liouville. Despite the efforts of Liouville, the work of Evariste Galois remained almost totally unknown to the world of mathematics (Leopold Kronecker had only used some of his results), and this is Jordan with his \textit{Treatise of algebraic substitutions and equations}, published in Paris in 1870, that we owe the first systematic presentation of group theory, enriched with ten years of personal research. Jordan limits his study to the finite groups, specifically the groups of permutations, and introduced many new concepts. In later submissions, Jordan studied in detail, mainly in terms of the factors of composition, the linear and orthogonal group and symplectic groups on a prime body. Jordan studies on the linear group involve considerations of matrix reduction, and in particular, on the shape named "Jordan form". Finally, we note the efforts of Jordan to determine all finite solvable groups in response to the problem posed by Niels Henrik Abel, to find all given degree equations solvable by radicals. In addition to the results given above for the linear group, we owe to Jordan a complete study of the real Euclidean $n$-dimensional geometries using entirely analytical methods. The teaching of Jordan at the École Polytechnique and the Collège de France, leads him to clarify many concepts of the theory of functions of real variable and his \textit{Cours d'analyse de Polytechnique} (first edition in 1880) will help to train generations of mathematicians. We also owe him the concept of a function with bounded variation, which allows him to give a correct definition of the length of a curve and to obtain the general form of the theorem of convergence for Fourier series, but the most famous result is one that says that a simple closed curve (known nowadays, "Jordan curve") divides the plane into two regions.  We also owe him a classical proof of Euler's theorem on polyhedra and the fact that two surfaces of the same kind are applicable to the other one (which, as shown by Poincaré, is not generally true for hypersurfaces).

\parpic[l][t]{%
  \begin{minipage}{40mm}
    \fbox{\includegraphics[width=110px,height=140px]{img/medaillons/jordanp.eps}}
  \end{minipage}
}
\textbf{Jordan, Pascual} (1902-1980) was born in Hanover and died in Hamburg (Germany). He was a theoretical physicist, professor at the University of Göttingen. Jordan passed in 1921 a part of his studies at the Hanover Teknische Universität where he studied a mixture of zoology, mathematics and physics. In 1923 he specialized when he entered at Göttingen Universität, who was then at its zenith from the point of view of mathematics and physics. In Göttingen, Jordan became the assistant to Richard Courant and especially Max Born, who greatly influenced him. He contributed decisively to the foundation of quantum mechanics and quantum field theory. Because of its affiliation with the Nazi party, he was, however, rejected by the physicists community. In 1925, with Max Born, Jordan wrote the canonical commutation relation between momentum and position. In the same article, he also offers the idea that we must also quantify the electromagnetic field, paving the way to quantum field theory. Also in 1925 with Max Born and Werner Heisenberg, Jordan develops the Heisenberg's matrix formulation of quantum mechanics. They introduce the canonical transformations, perturbation theory, the treatment of degenerate systems, and finally the famous canonical commutation relation of the components of angular momentum.

\parpic[l][t]{%
  \begin{minipage}{40mm}
    \fbox{\includegraphics[width=110px,height=140px]{img/medaillons/joule.eps}}
  \end{minipage}
}
\textbf{Joule, James Prescott} (1818-1889) was a  British physicist, born in Salford, Lancashire, and died in Sale. He was one of the greatest physicists of his time. Joule is famous for his research in electricity and thermodynamics. During his research on the heat emitted by an electrical circuit, he formulated the law, known always today as "Joule's law" on heat supply, which indicates that the amount of heat generated per second in a conductor by the passage of electric current is proportional to the electrical resistance of the conductor and the square of the electric current. Joule experimentally verified the law of conservation of energy in his study of the transformation of mechanical energy into thermal energy (relation between joules and calories: it takes 1 calorie or 4.18 joules to raise 1 gram of water of 1 degree).

\phantomsection
\addcontentsline{toc}{section}{K}
\label{sec:K}

\parpic[l][t]{%
  \begin{minipage}{40mm}
    \fbox{\includegraphics[width=110px,height=140px]{img/medaillons/kepler.eps}}
  \end{minipage}
}
\textbf{Kepler, Johannes} (1571-1630) was German astronomer and physicist, famous for his formulation and verification of the three laws of planetary motion. These laws are still known as "Kepler's laws". His main treatise contains the formulations of two laws of planetary motion. The first states that the planets move in elliptical orbits with the Sun as focal point and the second, or "area law" states that the imaginary line that we would trace between the Sun and a planet sweeps out equal areas of an ellipse during equal intervals of time, in other words, the more the planet approaches the Sun, the more quickly it moves. Another treaty contains another discovery of planetary motion: the cube of the distance between a planet and the Sun divided by the squared orbital period of this planet is a constant and is the same for all planets. The English mathematician and physicist Isaac Newton strongly based on the theories and observations of Kepler to formulate his theory of gravity. Kepler also brought his contribution in the field of optics and developed in mathematics an infinitesimal system which was the precursor of infinitesimal calculus.

\parpic[l][t]{%
  \begin{minipage}{40mm}
    \fbox{\includegraphics[width=110px,height=140px]{img/medaillons/keynes.eps}}
  \end{minipage}
}
\textbf{Keynes, John Maynard} (1883-1946) was a British economist. He is the founder of the "Keynesian" economic theory that promotes government intervention in the economy to ensure full employment. Keynes was born into a family of academics. At the age of seven, he entered the Perse School. Two years later, he entered preparatory class at St Faith's. Over the years, he showed great dispositions and in 1894, he finished first in his class and received an award for the first time in mathematics. A year later, he joined the Eton's College where he shines and wins in 1899 and 1900, the price of mathematics. In 1901, he finished first in mathematics, history and English. In 1902, he earned his place for the Cambridge King's College. Keynes is undoubtedly an important figure in the history of economic science that he revolutionized with his main work, The\textit{ General Theory of Employment, Interest and Money}, published in 1936. The book is considered as the most influential treaty of social science's in the 20th century because it has rapidly and continuously changed the way the world viewed the economy and the role of political power in society. Some believe that no other book has had such importance for Europe, even if the book of Friedrich Hayek, who received a Nobel Prize, \textit{The Road to Serfdom}, make a dramatic demonstration of the limits of Keynesian theory. With the \textit{General Theory}, Keynes developed a theory that could explain the level of production and hence this of employment; the determining factor being the demand. Among the revolutionary concepts introduced by Keynes, we note: those of underemployment equilibrium where unemployment is possible for a given level of effective demand, the absence of a regulatory mechanism for prices to reduce unemployment, a theory of money based on the preference for liquidity, the introduction of uncertainty and forecasts, the concept of marginal efficiency of investment breaking Say's Law (and therefore reversing the causal savings-investment relation). These concepts accredit interventionist policies to eliminate recessions and slow down economic overheating. All of these concepts are now named "Macroeconomics".

\parpic[l][t]{%
  \begin{minipage}{40mm}
    \fbox{\includegraphics[width=110px,height=140px]{img/medaillons/kirchhoff.eps}}
  \end{minipage}
}
\textbf{Kirchhoff, Gustav Robert} (1824-1887) was born in Könisberg (now Kaliningrad, Russia) and died in Berlin (Germany). Kirchhoff studied mathematical physics with Franz Neumann. After a PhD in 1847, he became a lecturer at the Berlin Universität  before obtaining, in 1850, the position of extraordinary professor of physics at the Breslau Universität. This is where he met the chemist Robert Wilhelm Bunsen, with whom he will be working for many years. Their collaboration will continue beyond 1854, when Kirchhoff was appointed professor of physics at the  Heidelberg Universität. Elected vice-president of the same university in 1865, he finally accepted a professorship in theoretical physics at Berlin in 1875. Kirchhoff was still a student when he began to take an interest in issues related to electricity. In 1845, he established the concept of electric potential and sets the laws of networks that bear still bear his name today (Kirchhoff's laws). He also generalizes Ohm's law on the electric current of three dimensional conductors and, later, shows that the flow of current through a conductor occurs at the speed of light. His relation with Bunsen led to the birth of spectroscopy. Together, the two researchers discover the specific nature of the spectrum of light emitted by each chemistry body. With this new analysis tool, they track down two unknown elements: caesium (1860) and rubidium (1861). The development of prism spectroscope to analyse the light burning substances, also allows to establish Kirchhoff's radiation law: the ratio of powers of absorption and emission of a body, independent of the properties of this body is a function of temperature and wavelength. The emittance is thus proportional to that of the "black-body" defined by Kirchhoff (1862) as the perfect absorbent body. This law, which explains the presence of such dark lines of absorption (called "Fraunhofer lines") in the spectrum of solar radiation, marks the beginning of a new era in astrophysics and announce the beginning of Planck's quantum theory.

\parpic[l][t]{%
  \begin{minipage}{40mm}
    \fbox{\includegraphics[width=110px,height=140px]{img/medaillons/klein.eps}}
  \end{minipage}
}
\textbf{Klein, Felix} (1849-1925) made his studies at Bonn, Göttingen and Berlin (Germany). In 1872 he became professor of mathematics at the University of Erlangen, where his inaugural lecture was the statement of the outline of his famous Erlangen Program. He then taught at Munich (1875-1880), then at the Leipzig Universität (1880-1886) and finally Göttingen (1886-1913). From 1872, he published the \textit{Mathematische Annalen of Goettingen} and founded in 1895, the great \textit{Mathematical Encyclopedia}, he oversaw the writing until his death in Göttingen. He was the undisputed leader of the German school of mathematics, and his influence was great (he gave numerous lectures in foreign countries, including in the United States), especially on the development of geometry, with his Erlangen's program. With the text, published in his book\textit{ Gesammelte mathematische Abhandlungen} (1921-1923), Klein gives a definition of the geometry including both classical geometry (that is to say, Euclidean) and projective geometry, non-Euclidean geometries, etc., ending the sterile controversies between supporters of those synthetic geometry and analytic geometry. For him, a geometry is the study of invariant properties under a given group of transformations: in this way theorems of classical geometry are the expression of invariants relations of the group of similarities, those of projective geometry between covariants of the projective group. We are indebted to Klein extensive works on the hypergeometric differential equation, on Abelian functions on the group of the regular icosahedron (\textit{Lectures on the Icosahedron}, 1914), on elliptic functions, from which he emerged the notion modular function (\textit{Vorlesungen über die Theorie der Funktionen automorphen}, 1897-1902).

\parpic[l][t]{%
  \begin{minipage}{40mm}
    \fbox{\includegraphics[width=110px,height=140px]{img/medaillons/kolmogorov.eps}}
  \end{minipage}
}
\textbf{Kolmogorov, Andrei} (1903-1987) was a Russian mathematician whose contributions are significant in mathematics. Kolmogorov was born at Tambov. His single mother died at his birth and he was educated by his aunt with the savings of his grandfather. Is supposed that his father was killed during the Russian Civil War. Kolmogorov was educated at the village school of his aunt, and his first literary efforts and mathematical papers were printed in the school newspaper. Teenager, he designed perpetual motion machines, hiding their intrinsic defects so well that its secondary school teachers could not discover the tricks. In 1910, he was adopted by his aunt and they moved to Moscow, where he entered a Gymnasium and graduated there in 1920. After completing his secondary education, he studied at the University of Moscow and Mendeleev Institute. He studied not only mathematics, but also russian history and metallurgy. In 1922, Kolmogorov published his first results on the theory of sets, and in 1923 he published his work on the theory of integration, Fourier analysis and for the first time on probability theory and is starting to become known abroad. After completing his studies in 1925, he started his PhD with Nikolai Louzine, which he completed in 1929. In 1931, he received a professorship at the University of Moscow. In 1933, published in German, his manual \textit{Fundamentals of probability theory} in which he presents his axiomatization of probability and an appropriate method to treat stochastic processes. The same year, he became director of the Institute of Mathematics of the University of Moscow. In 1934, he published his work on the cohomology and gets, thanks to this thesis, a PhD in mathematics and physics. He gets prizes from Soviet authorities, such as the Order of socialist science (1940), the Stalin Prize (1941) and Lenin Prize several times. In 1941, he developed a famous theory of fluid turbulence. In 1953 and 1954, he describes the KAM theory (Kolmogorov-Arnold-Moser) stability of dynamical systems (a complex mechanical systems exactly solvable is stable only if we disturbs it a little bit). He also introduces the notion of metric entropy for measured dynamic systems. In 1955, he became an honorary doctorate from the Sorbonne (France). In 1962, he awarded the Balzan Prize for mathematics.

\parpic[l][t]{%
  \begin{minipage}{40mm}
    \fbox{\includegraphics[width=110px,height=140px]{img/medaillons/kronecker.eps}}
  \end{minipage}
}
\textbf{Kronecker, Leopold} (1823-1891) was a German mathematician who appears as one of the greatest number theorists of the 19th century and one of the founders of the theory of algebraic numbers. His works on particular class fields prepared those of Hilbert. Born in Liegnitz, in a family of wealthy merchants, Kronecker followed at the gymnasium the courses of Ernst Kummer, who he was to meet later as a professor at the Breslau Universität, then as a colleague in Berlin. Peter Gustav Lejeune-Dirichlet and Ernst Kummer have had a profound influence on the development of his thought. Having argued, in 1845, a highly original theory of cyclotomic units he held for several years, family affairs, and could not deliver entirely new mathematical research until 1853. Elected in 1860 member of the Academy of Sciences in Berlin, he gave, from that time, free courses at the university, where he was appointed professor in 1883 and where he ended his life. Instead wielding virtuosity with all the resources of the analysis (as show his works on elliptic functions, Dirichlet series or even the integral formula giving the number of roots of a system of equations in n-dimensional space ), Kronecker is before all an algebraist and arithmetician. Towards the end of his life, he professed a doctrine to reject the actual infinite in mathematics as valid only keeping what could only be based on integers (his polemics with Cantor remained famous). In algebra, Kronecker was one of the most active leaders of the group of mathematicians who, in the years 1860-1890, succeeded to develop linear and multilinear algebra inaugurated by Arthur Cayley and Hermann Grassmann around 1845. So he went and completed the works of Karl Weierstrass and was one of the first to understand and use the work of Evariste Galois (published in 1846).

\phantomsection
\addcontentsline{toc}{section}{L}
\label{sec:L}

\parpic[l][t]{%
  \begin{minipage}{40mm}
    \fbox{\includegraphics[width=110px,height=140px]{img/medaillons/lagrange.eps}}
  \end{minipage}
}
\textbf{Lagrange, Joseph Louis} (1736-1813) was born in Turin and died in Paris. He was as one of the greatest mathematician and astronomer of the 18th century. Brilliant student from a wealthy family, he studied at the College of Turin. He takes a liking for mathematics by chance at the age of seventeen after reading a paper by Edmund Halley on applications of algebra in optics. The subject interests him at the highest point. Therefore, he is passionate about mathematics that he studied diligently and alone. He quickly became a confirmed mathematician and his first important results arrive quickly. In a letter to Leonhard Euler he laid the foundations of the calculus of variations. This exchange is the beginning of a long correspondence between the two men. Lagrange was then nineteen years old and teaches at the Artillery School in Turin where he was appointed in 1755. He founded in 1758 the Academy of Sciences of Turin which will publish his first results on the application of variational calculus to mechanical problems (sound propagation, vibrating string ...). In 1764, his work on the libration of the Moon (small variations in its orbit) are awarded by the Grand Prix de l'Académie des Sciences in Paris. He introduced new methods for the calculus of variations and the study of differential equations, which enabled him to give a systematic presentation of mechanics in his famous book \textit{Analytical Mechanics} (1788). He worked on additive number theory. We owe him the theorem on the decomposition of an integer into four squares. In the study of algebraic equations, he introduced the concepts that lead to group theory later developed by Abel and Galois. In physics, precising the principle of least action, with the calculus of variations, about 1756, he invented the Lagrange function, which verifies the Lagrange equations, then he develops analytical mechanics, about 1788, where he introduced the Lagrange multipliers. He also undertakes extensive research on the three-body problem in astronomy, one of the results being the highlight of the libration points still today named "Lagrange points" (1772).

\parpic[l][t]{%
  \begin{minipage}{40mm}
    \fbox{\includegraphics[width=110px,height=140px]{img/medaillons/landau.eps}}
  \end{minipage}
}
\textbf{Landau, Lev Davidovich} (1908-1968) was born in Azerbaijan and died in Moscow. He was the son of an engineer and a doctor. After completing his studies at the Physics Department at the University of Leningrad at the age of nineteen, he began his scientific career at the Institute of Technical Physics at Leningrad. From 1932 to 1937 he was the Head of the Theoretical Technical Physics  Institute in Kharkov (Ukrain) and in 1937 he was appointed head of the Department of Theoretical Institute for Physical Problems at the USSR Academy of Sciences of Moscow. Landau's work covers all branches of theoretical physics at the limits of fluid mechanics to quantum field theory. Much of his papers refers to the theory of condensed state. They started in 1936 by a formulation of a general theory of phase transitions of the second order. After the discovery of Kapitsa, in 1938, the superfluidity of liquid helium, Landau has initiated extensive research that has led to the construction of the complete theory of quantum liquids at very low temperatures. Among his writings, covering a wide range of topics related to physical phenomena, there are more than one hundred articles and several books, including the famous \textit{Course of Theoretical Physics}, published in 1943 with E.M. Lifchitz. Landau has dominated the theoretical physics from 1930 to 1965. He created a series of tests of theoretical physics, named the "theoretical minimum" that students or senior researchers had to pass to get into his research group, which included examination of problems in all branches of mathematics.

\parpic[l][t]{%
  \begin{minipage}{40mm}
    \fbox{\includegraphics[width=110px,height=140px]{img/medaillons/langevin.eps}}
  \end{minipage}
}
\textbf{Langevin, Paul} (1872-1946) was a French physicist born and died in Paris (France). Very young Langevin manifest exceptional gifts. Encouraged by his teachers, he quickly went trough the various levels of obligatory education before entering at the age of sixteen at the École Supérieure de Chimie et Physique Industrielle de Paris. Langevin there follows the laboratory courses and teaching of Pierre Curie, with whom he became friends. On leaving the school, he abandoned a career as an engineer and decided, on the advice of Pierre Curie, to focus on research and teaching. Also, he postulated for a job position at the École Normale Supérieure where he was received first in 1894. In 1897 he received a scholarship to go to work one year at the Cavendish Laboratory of Cambridge, high center of European science where E. Rutherford and J. J. Thomson work. Back in France, he defended his thesis in 1902, was appointed deputy professor, then professor at the Collège de France. In 1904, he succeeded Pierre Curie at the School of Physics and Chemistry, where he became director in 1925. Langevin's work lies in this long transition period from 1900 to 1930 that leads from classical physics to modern physics dominated by relativity theory and quantum theory. His first work (on the ionization of gases) led him to develop his main theoretical model in 1905, which should then form the basis for many other explanations of macroscopic properties of matter, in which electrons within atoms define closed orbits, thereby conferring atoms properties similar to those of small magnets. In 1906 he founded the surprising result that inertia is a property of energy... at least in the case of the electron. It is only a few months later that he will read the Einstein's memory on theory of relativity which he will devote his teaching during his courses at the Collège de France. Langevin is also at the origin of the famous Solvay Conference that, starting from 1911, met periodically all the great names of physics, where the concepts of quantum theory were widely discussed. It is thanks to Langeving that the work of his pupil Louis de Broglie on wave mechanics knew the diffusion that deserved him: first surprised, Langevin was quickly convinced of the correctness of De Broglie's ideas and planned immediately the new wave mechanics program's lectures at the Collège de France. Faithful to the ideal of teaching clarity, Langevin has also conducted, on the concepts still being developed quantum theory, a job and redesign analysis for which we always measure today the epistemological significance.

\parpic[l][t]{%
  \begin{minipage}{40mm}
    \fbox{\includegraphics[width=110px,height=140px]{img/medaillons/langevin.eps}}
  \end{minipage}
}
\textbf{Laplace, Pierre Simon} (1749-1827) was born in Beaumont-en-Auge and died in Paris (France). Son of a farmer, Laplace was initiated in mathematics at the military school of Beaumont-en-Auge and began there his teaching. He was able to follow this education thanks to his affluent neighbours who detected his exceptional intelligence. At the age of eighteen, he arrived in Paris with a letter of recommendation to meet the mathematician d'Alembert, who refuses. But Laplace insists and sends to d'Alembert an article he wrote on classical mechanics. D'Alembert is so impressed that he is happy to sponsor Laplace and found him teaching math job position. The most important work of Laplace is about probability calculus, differential equations (laplace operator) and celestial mechanics. He also establishes, through his work with Lavoisier between 1782 and 1784 the relation of adiabatic transformations of a gas, as well as two fundamental laws of electromagnetism. In Mechanics, it is with the mathematician Joseph-Louis Lagrange, that Laplace summarizes his work and merged those of Newton, Halley, Clairaut, d'Alembert and Euler, on universal gravitation (especially the problem of stability of the solar system) in the five volumes of his Celestial mechanics (1798-1825). It is reported (but it is most likely a legend) that reading \textit{Celestial mechanics}, Napoleon remarked that there was no mention of God. «I do not need this hypothesis», replied Laplace who was not otherwise modest (considering himself - probably rightly - as the best mathematician of his generation). He is also one of the first scientists to conceive the existence of black holes and the notion of gravitational collapse.

\parpic[l][t]{%
  \begin{minipage}{40mm}
    \fbox{\includegraphics[width=110px,height=140px]{img/medaillons/anonymous.eps}}
  \end{minipage}
}
\textbf{Laurent, Pierre Alphonse} (1813-1854) was a French mathematician born in Paris and who became famous for the discovery of the Laurent series in complex analysis that has a great impact in the calculation of certain integrals in physics. He entered the École Polytechnique de Paris in 1830. Laurent was graduated in 1832 as one of the best students of the year and entered the engineering corps as a lieutenant. During the management of the development projects of the port of Le Havre, Lawrence wrote his first mathematical publication on Laurent series. This research was contained in a memorandum submitted to the Grand Prix de l'Académie des Sciences in 1843, but his application was too late, the article has not been included in the price. However, Cauchy made a reference in his works to Laurent's paper three months later. The same problem occurred again for another major publication of Laurent a few months later. After these events, Laurent, disappointed changed his research field to focus on physics (Applied Mathematics). Cauchy offered him a vacancy job at the Academy of Sciences in 1846, but his application was not accepted. Laurent died in Paris at the age of forty-one. His writings were published after his death.

\parpic[l][t]{%
  \begin{minipage}{40mm}
    \fbox{\includegraphics[width=110px,height=140px]{img/medaillons/lavoisier.eps}}
  \end{minipage}
}
\textbf{Lavoisier, Antoine Laurent} (1743-1794), was a French chemist know as the founder of modern chemistry. Lavoisier was born in Paris and studied at the Collège Mazarin. He was elected member of the Académie des Sciences in 1768. He held several positions, including Director of Poudreries Nationales in 1776, member of the Commission pour l'établissement du nouveau système de poids et mesures in 1790 and Secretary of the Trésorerie in 1791. He tried to introduce reforms in the French monetary and fiscal policy, as well as in the agricultural system. Lavoisier was one of the first to realize truly quantitative chemical experiments. He showed that despite the change of state of the material in a chemical reaction, the amount of material remained constant between the start and the end of each reaction. These experiments have provided evidence in favour of the law of conservation of matter. Lavoisier also did research on the composition of the water, which he named the components: "oxygen" and "hydrogen". One of the most important experiments of Lavoisier was about the nature of the combustion (or burning). He demonstrated that the combustion process implies the presence of oxygen. He also demonstrated the role of oxygen in the respiration of animals and plants. Lavoisier's explanation of combustion replaced the doctrine of phlogiston. This indeed postulated that a substance emerged, the "phlogiston", when the material is consumed. As one of twenty-eight general farmers, Lavoisier is stupidly branded as a traitor by the revolutionists in 1794 and guillotined during the Terror in Paris in 1794, at the age of 50 years, along with all colleagues.

\parpic[l][t]{%
  \begin{minipage}{40mm}
    \fbox{\includegraphics[width=110px,height=140px]{img/medaillons/lebesgue.eps}}
  \end{minipage}
}
\textbf{Lebesgue, Henri Léon} (1875-1941) was born in Beauvais and died in Paris (France) is a former student of the École Normale Supérieure, he had Émile Borel as teacher (who we own the first major work in measure theory). After a few years in the high school of Nancy, Lebesgue will teach at Rennes. It was during this period that he will be known for his elegant theory of measurement. Professor at the Sorbonne and the College de France, he was elected to the Académie des Sciences in 1922. By his theory of measurable functions (1901) based on the Borel tribe (named after the mathematician Émile Borel), Lebesgue extensively revised and generalized integral calculus. His theory of integration (1902-1904) addresses the needs of physicists to the research and the existence of primitive for "irregular" functions. We owe him also the Fourier transform established in the late 30s. He was appointed professor at the Sorbonne in 1910 and at the Collège de France in 1921. He also teaches at the École de Physique Industrielle et Chimie of Paris from 1927 to 1937 and at the École Normale Supérieure of Sèvres.

\parpic[l][t]{%
  \begin{minipage}{40mm}
    \fbox{\includegraphics[width=110px,height=140px]{img/medaillons/lee.eps}}
  \end{minipage}
}
\textbf{Lee Tsung-Dao} (1926-) is born in Shanghai (China). He is the son of a businessman. The Sino-Japanese War of 1937-1945 made him leave the Kweichow University in the province of Zhejiang to join that of Kunming in Province of Yunnan, where he met Yang Chen-Ning, which will be a long time his friend and collaborator. A Chinese government scholarship enabled him to finish his studies at the University of Chicago (USA), where he defended his thesis on the hydrogen content of white dwarfs in 1950. Member of the Institute for Advanced Study in Princeton (New Jersey) from 1951 to 1953, he soon became, at 29 years old, the youngest professor at Columbia University in New York. In 1956, physicists were subjected to a puzzle emerged from analysis of the data provided by the particle accelerator at Brookhaven National Laboratory, near New York, two particles, called "tau" and "theta", seemed to have the same mass and even nuclear interactions, but differed in their decay products. Lee and Yang proposed that they were a single particle, now denoted "$K_0$", and that the weak interaction responsible for the decay does not respect parity symmetry. They concluded that it was necessary to submit to experimental verification that the weak interaction distinguishes right from left. Six months sufficed for the team from the National Bureau of Standards in Washington, mobilized by the Chinese physicist Chien-Shiung Wu, to show that radioactive Cobalt-60 polarized nuclei emitted more electrons in one direction than in the opposite direction. Quickly confirmed by several other experimental groups, the violation of mirror symmetry earned Tsung-Dao Lee and Chen Ning Yang to share the Nobel Prize in Physics 1957.

\parpic[l][t]{%
  \begin{minipage}{40mm}
    \fbox{\includegraphics[width=110px,height=140px]{img/medaillons/anonymous.eps}}
  \end{minipage}
}
\textbf{Legendre, Adrien Marie} (1752-1833) was French mathematician born in Paris and died in Auteil. He holds the Chair of Mathematics at the École Militaire of Paris from 1775 to 1780. In 1783, he became a member of the Académie des Sciences. In 1787, he was appointed Commissioner for geodetic operations. Legendre interests were varied: analysis, number theory, geometry, statistics (least squares methods) and mechanics (Legendre transform in analytical mechanics and thermodynamics). About a century before we get evidence, he conjectured the prime number theorem and the law of quadratic reciprocity. Throughout his life, he became interested in elliptic integrals, whose work would eventually give rise to elliptic curves, subject studied a lot by contemporary mathematicians. He lets as heritage to the mathematical community of the 19th century a treatise on elementary geometry, which is very precious in the world of education.

\parpic[l][t]{%
  \begin{minipage}{40mm}
    \fbox{\includegraphics[width=110px,height=140px]{img/medaillons/leibniz.eps}}
  \end{minipage}
}
\textbf{Leibniz, Gottfried Wilhelm} (1646-1716) was born in Leipzig and died in Hanover. He was a philosopher, mathematician, lawyer and considered as one of the most brilliant minds of the 17th century. Son of a lawyer he graduated in 1663 in ancient philosophy and later wrote a theory of probability in Law... He then entered the Leipzig Universität and in 1666 obtained his PhD as Lawyer. In 1669 he became an adviser to the Chancellor of the electorate of Mainz. He was sent to Paris in 1672, for a supposed diplomatic mission, to convince Louis XIV shift his conquests to Egypt rather than Germany. He stayed there until 1676 and met the great scientific of this time. It was during this period that Leibniz worked on his main scientific topics. In 1676 he was appointed librarian of Brunswick-Luneburg and also managed mathematics, physics, religion and diplomacy. Leibniz contributed to mathematics by discovering, in 1675, the fundamentals of infinitesimal calculus. This discovery was made independently of the discoveries of Newton, who invented the system of infinitesimal calculation in 1666. Leibniz system was published in 1684, that of Newton in 1687, that's when the notation imagined by Leibniz was adopted and he is also considered as a pioneer in the development of mathematical logic.

\parpic[l][t]{%
  \begin{minipage}{40mm}
    \fbox{\includegraphics[width=110px,height=140px]{img/medaillons/lemaitre.jpg}}
  \end{minipage}
}
\textbf{Lemaître, Georges } (1894-1966) was a Belgian Catholic priest, astronomer and professor of physics at the Catholic University of Leuven. He proposed on theoretical grounds that the universe is expanding, which was observationally confirmed soon afterwards by Edwin Hubble. He was the first to derive what is now known as "Hubble's law" and made the first estimation of what is now called the "Hubble parameter", which he published in 1927, two years before Hubble's article. Lemaître also proposed what became known as the "Big Bang theory" of the origin of the universe. After a classical education at a Jesuit secondary school, the Collège du Sacré-Coeur, in Charleroi, Lemaître began studying civil engineering at the Catholic University of Leuven at the age of 17. In 1914, he interrupted his studies to serve as an artillery officer in the Belgian army for the duration of World War I. At the end of hostilities, he received the Belgian War Cross with palms. After the war, he studied physics and mathematics, and began to prepare for the diocesan priesthood, not for the Jesuits. He obtained his doctorate in 1920 with a thesis entitled l'\textit{Approximation des fonctions de plusieurs variables réelles} (Approximation of functions of several real variables), written under the direction of Charles de la Vallée-Poussin. He was ordained a priest in 1923. In 1923, he became a graduate student in astronomy at the University of Cambridge, spending a year at St Edmund's House (now St Edmund's College, Cambridge). He worked with Arthur Eddington, who introduced him to modern cosmology, stellar astronomy, and numerical analysis. He spent the next year at Harvard College Observatory in Cambridge, Massachusetts, with Harlow Shapley, who had just gained renown for his work on nebulae, and at the Massachusetts Institute of Technology (MIT), where he registered for the doctoral program in sciences. On his return to Belgium in 1925, he became a part-time lecturer at the Catholic University of Leuven. He began the report which brought him international fame when it was published in 1927 in the \textit{Annales de la Société Scientifique de Bruxelles} (Annals of the Scientific Society of Brussels) under the title \textit{Un Univers homogène de masse constante et de rayon croissant rendant compte de la vitesse radiale des nébuleuses extragalactiques} (A homogeneous Universe of constant mass and growing radius accounting for the radial velocity of extragalactic nebulae). In this report, he presented his new idea that the universe is expanding, which he derived from General Relativity; this later became known as "Hubble's law", but Lemaître provided the first observational estimation of the Hubble constant. The initial state he proposed was taken to be Einstein's own model of a finitely sized static universe. The paper had little impact because the journal in which it was published was not widely read by astronomers outside Belgium; Arthur Eddington reportedly helped translate the article into English in 1931. Lemaître's proposal met with scepticism from his fellow scientists. Eddington found Lemaître's notion unpleasant. Einstein thought it unjustifiable from a physical point of view, although he encouraged Lemaître to look into the possibility of models of non-isotropic expansion, so it is clear he was not altogether dismissive of the concept. Einstein also appreciated Lemaître's argument that Einstein's model of a static universe could not be sustained into the infinite past. He died on 20 June 1966, shortly after having learned of the discovery of cosmic microwave background radiation, which provided further evidence for his proposal about the birth of the universe.

\parpic[l][t]{%
  \begin{minipage}{40mm}
    \fbox{\includegraphics[width=110px,height=140px]{img/medaillons/levicivita.eps}}
  \end{minipage}
}
\textbf{Levi-Civita, Tullio} (1873-1941) was born in Padua and died in Rome. He graduated in 1892 from the Faculty of Mathematics of the University of Padua. In 1894, he obtained a teaching degree at the College of Education of the Faculty of Pavia. In 1898, he was appointed head of the chair of celestial and analytical mechanics of Padua where he met Libera Trevisani, one of his students, whom he married in 1914. He remained in Padua until 1918, then was appointed to the chair of analysis at the University of Rome, where he took two years later the chair of professor of mechanical engineering. Foremost physicist, his works are mainly related to electromagnetism and to the theories of Lorentz and Maxwell. In 1900, he published with Ricci his Theory of tensors in the methods of differential calculus and their applications that Einstein used to better control the tensor calculus, a key tool for the development of his theory of General Relativity. Levi-Civita also discussed a series of issues about the static gravitational field in his correspondence with Einstein in the years 1915-1917. Their correspondence revolved around the variational formulation of the equations of gravitational fields and their covariant properties, and the definition of the gravitational energy and the existence of gravitational waves. Levi-Civita in 1933 also contributed to the Dirac equations of quantum mechanics.

\parpic[l][t]{%
  \begin{minipage}{40mm}
    \fbox{\includegraphics[width=110px,height=140px]{img/medaillons/lie.eps}}
  \end{minipage}
}
\textbf{Lie, Sophus} (1842-1899) was a Norwegian mathematician educated at the University of Christiana. He gave private lessons to earn money, and spent the winter of 1869-1870 with Klein in Berlin, the summer of 1870 in Paris. In 1872, a mathematics chair was created for him at Christiana, and in 1886 he succeeded Klein in Leipzig. In addition to his work in projective geometry of space,we retain from Lie his studies on new algebraic structures that he applies to geometry, until the creation of the theory of groups and algebras that bear his name. In the concept of Lie group and algebra, involved continuity properties (topological group), announcing the new important branch of mathematics that will be the topology. Lie's work in this area will be mainly pursued by Élie Cartan.

\parpic[l][t]{%
  \begin{minipage}{40mm}
    \fbox{\includegraphics[width=110px,height=140px]{img/medaillons/lindemann.eps}}
  \end{minipage}
}
\textbf{Lindemann, Ferdinand} (1852-1939) was born in Hanover and died in Munich. He was the first mathematician to prove the transcendence of $\pi$. When Ferdinand was two years old he moved to Schwerin where he spent his childhood and primary schooling. As it was the tradition at that time in Germany during the second half of the 19th century, Lindemann moved frequently from one university to another. He began his studies at Göttingen in 1870, where he was greatly influenced by Clebsch. Later, Lindemann who had established very good relations with Clebsch wrote again his geometry lecture notes after his death for their publication in 1876. Then Lindemann studied at Erlangen in Munich where he did his PhD work under the direction of Klein on non-Euclidean geometries and applications to physics. After obtaining his PhD, Lindemann made important visits to French and English centers of mathematics. In England, he visited Oxford, Cambridge and London, while in France, he spent most of his time in Paris where he was greatly influenced by Chasles, Bertrand, Jordan and Hermite. When he returned to Germany, Lindemann worked on publications subjects to reintegrate and obtain the recognition of the German scientific community. In 1877 he was finally nominated professor at the University of Würzburg and Professor at the University of Freiburg in 1879. The main work carried by Lindemann was on geometry and analysis and is particularly known for his famous proof of transcendence. In 1873, when Lindemann had just received his PhD, Hermite proved the transcendence of the Euler number. Shortly after, Lindemann met Hermite in Paris and discussed the methods used for the demonstration. Thus, using a similar reasoning, Lindemann proved in 1882 the transcendence of $\pi$ (based on the fact that the Euler number is itself transcendent).

\parpic[l][t]{%
  \begin{minipage}{40mm}
    \fbox{\includegraphics[width=110px,height=140px]{img/medaillons/liouville.eps}}
  \end{minipage}
}
\textbf{Liouville, Joseph} (1809-1882) was born in St-Omer and died in Paris. He an active author for the deployment of mathematics and had a considerable activity in the teaching and dissemination of mathematical ideas of his time. He is the founder of the\textit{ Journal de Mathématiques Pures} traditionally named "Journal de Liouville". His main research focuses on the analysis and we owe him an important theorem on the approximation of algebraic irrational. The election of Joseph Liouville in the Constituent Assembly of 1848 is the only event that break the unity of his whole scientific career: He finished the École Polytechnique in 1827, then he returned there in 1833 as a coach and teacher on Analysis. At the age of 31, he was elected to the Academie des Sciences in the section of astronomy. He was one of the best teachers of his time, and his lectures at the École Polytechnique and the Collège de France, took a large part of its time. Liouville founded the \textit{Journal de Mathématiques Pures} in 1836 and managed it during 39 years. Its academic and editor tasks, for which he complained, stripped him the necessary freedom of mind for thorough research. But he took advantage of the one and the other task to help several young mathematicians with a great future, eg C. Hermite and C. Jordan by glowing reports at the Academy, or the publication of their work in its journal. Meanwhile, he published mostly short notes on a number of issues: analysis, arithmetic, geometry, mechanics, astronomy. He shares with A. Cauchy the merit of having submitted analysis to strict rules often violated in the 18th century, and this merit is even higher as the mathematical language level of his time was not helping...

\parpic[l][t]{%
  \begin{minipage}{40mm}
    \fbox{\includegraphics[width=110px,height=140px]{img/medaillons/lobatchevski.eps}}
  \end{minipage}
}
\textbf{Lobachevsky, Nikolai Ivanovich} (1792-1856) was a Russian mathematician born in Nizhny-Novgorod and died in Kazan. Lobatchevski studied at the Kazan University, where he taught from 1812 and occupied the chair of pure mathematics from 1822 to 1846. Under the influence of Gauss and Laplace, his first works are: \textit{Theory of elliptical motion of celestial bodies} and \textit{On the solution of the simple complex algebraic equation}. But his main research concerns the geometry. His first book, \textit{Geometry} (1823), considered too revolutionary (he used the metric system), won't be published during his lifetime. In 1826 Lobachevsky exposed to his colleagues from the university a memory that shows that he was one of the first mathematicians to be convinced of the possibility of a different geometry than this of Euclid. Despite the scepticism of his colleagues, he continues to study this new geometry (where the Euclidean postulate is replaced by "Lobachevsky postulate": from any point outside a line, it goes an infinite number of parallel to this line) and devotes his life as a mathematician trying to convince the scientific world. He published successively \textit{Elements of Geometry} (1829), \textit{New Elements of geometry with full theory of parallels} (1838) and \textit{Pangeometry} (1855). But the full recognition of the value of his work will come after his death (when Eugenio Beltrami in 1868, built a model of the Lobachevsky geometry: the pseudo-sphere). In addition to his mathematical research, Lobachevsky was the host of the Kazan University: Rector from 1827 to 1846, he was in charge of the university library, set up his observatory, managed the museum and directed the construction of new university building structures.

\parpic[l][t]{%
  \begin{minipage}{40mm}
    \fbox{\includegraphics[width=110px,height=140px]{img/medaillons/lorentz.eps}}
  \end{minipage}
}
\textbf{Lorentz, Hendrik} (1853-1928) was born in Arnhem and died at Haarlem (Netherlands). Lorentz has improved the Maxwell's electromagnetic theory in his PhD thesis on the theory of reflection and refraction of light which he presented in 1875. He was appointed professor of mathematical physics at the University of Leiden in 1878. He remained in this establishment until 1912 where Ehrenfest was appointed in his place. Lorentz was then appointed Director of Research at the Teyler's Institute of Teyler (Haarlem). He held an honorary position in Leiden, where he continued to give some courses. Before the existence of electrons was proved, Lorentz proposed that light waves were due to oscillations of electric charge in the atom. Lorentz developed his mathematical theory of the electron for which he received jointly with Zeeman (one of his students) the Nobel Prize in 1902.  Zeeman has experimentally verified the theoretical work of Lorentz on atomic structure, showing the effect of a strong magnetic field on the oscillations by measuring the change of the wavelength of the light produced. Lorentz is also famous for his work on the Fitzgerald-Lorentz contraction, a contraction in the length of an object at relativistic speeds. The Lorentz transformations, which he presented in 1904, form the basis of the special theory of relativity of Einstein, that was at the beginning named "Einstein-Lorentz special theory of relativity". They describe the increase of the mass, the shortening of the length, and the time dilation of a body moving at speeds near that of light. Lorentz was chairman of the first Solvay Conference that held in Brussels in autumn 1911. This conference was about the two approaches of the atomic theory, namely the classical theory and quantum physics. However, Lorentz never fully accepted quantum theory and has always hoped it would be possible to incorporate it back into the classical approach.

\parpic[l][t]{%
  \begin{minipage}{40mm}
    \fbox{\includegraphics[width=110px,height=140px]{img/medaillons/lucas.eps}}
  \end{minipage}
}
\textbf{Lucas, Edward} (1842-1891) was a French arithmetician born in Amiens and died in Paris. Child from a very modest family, he received a scholarship and passes the entrance examination at the École Normale Supérieure in 1861. On leaving the school, he became assistant astronomer at the Obsérvatoire de Paris, and after the Franco-Prussian war, he obtained a professorship of special mathematics at Moulins from 1872 to 1876. Then he held a professorship at Paris, first at the Lycée Charlemagne in Paris, then to the already prestigious Lycée Saint-Louis. His mathematical works concern non-elementary Euclidean geometry (projective geometry seen through its homographies), and especially the theory of numbers. His main contribution is made to primality tests. Partially forgotten in France (where the algebraic number theory is relegated to the background, waiting Weil), the work of Lucas is recovered and enhanced by the Anglo-Saxons, and especially by Lehmer, which will improve the primality test and prove totally some results of Lucas, to obtain the Lucas-Lehmer test, which is still used in the late 20th century to break records of large prime numbers. These studies are particularly important since the advent of computers that makes the cryptography hungry of very large prime numbers. Lucas is also known for being the inventor of many mathematical recreations. The most common of them is the Hanoi Tower problem, which he published under the name of Claus de Siam, professor at the College of Li-Sou-Tsiam anagram of Lucas d'Amiens, a professor at St. Louis.

\phantomsection
\addcontentsline{toc}{section}{M}
\label{sec:M}

\parpic[l][t]{%
  \begin{minipage}{40mm}
    \fbox{\includegraphics[width=110px,height=140px]{img/medaillons/malthus.eps}}
  \end{minipage}
}
\textbf{Malthus, Thomas Robert} (1766-1834) was born in Guildford and died at Bath. He was an Anglican pastor, who worried about the excessive growth of population in England at the beginning of the industrial revolution (1750 to 1900). His fear revolved around the idea that population growth is faster than the increase of resources, that implies the impoverishment of the population. Because the old regulators of population (wars and epidemics) no longer play their roles, he imagined new obstacles, such as limiting the size of families and the rising age of marriage. These proposals are implemented so far, both, in China, which is indeed obliged to severely limit its demography. The predictions of Malthus are in reality undermined because he could not imagine such a large increase in resources and crop yields (green revolution: chemistry applied to agriculture which is not necessarily beneficial .. .); new means of international exchange of subsistence goods (contributing the way... to the pollution of the oceans the way); the fact that the overflow of people emigrate to the United States or the colonies. However, if the predictions of Malthus are not realized, his theory retains all the attention. It is true that the population is increasing in some countries (Saudi Arabia: 6 children per woman) it is also true (and happy) that advances in health and medicine increase the size of the population, it is true that renewable resources on Earth are limited ultimately by the solar energy it receives, which itself determines biomass, except major scientific discovery... and under these conditions, mathematics is clear: it will not be possible for the terrestrial population to increase indefinitely, and the regulation must occur at one time or another, in one way or another!

\parpic[l][t]{%
  \begin{minipage}{40mm}
    \fbox{\includegraphics[width=110px,height=140px]{img/medaillons/marconi.eps}}
  \end{minipage}
}
\textbf{Marconi, Guglielmo} (1874-1937) was born in Rome and died in Rome. Marconi was a physicist, inventor and Italian businessman. He shared with Karl Ferdinand the Nobel Prize in Physics of 1909 in recognition of their contributions to the development of wireless telegraphy (we can consider that he is the inventor of transmission/reception equipment for electromagnetic waves and so radio and broadcast television). Marconi was born in a wealthy family, the second son of Giuseppe Marconi, an Italian owner, and an Irish mother, Annie Jameson, granddaughter of the founder of the Jameson Whiskey Distillery. He studied at Bologna in the lab of Augusto Righi, in Florence at the Cavallero Institute and, later, in Livorno. He mades in 1985 experiments on waves discovered by Heinrich Rudolf Hertz seven years ago. He reproduces the equipment used by Hertz but improved the Branly coherer to increase the sensitivity and the antenna of Alexander Popov. After his first experiences in Italy, he made in the Swiss Alps at Salvan a link of 1.5 km in the summer of 1895. The following year, being not followed by his compatriots, he went to England to continue and patent his experiments. In 1897 he established the first morse communication over 13 km between Lavernock (Wales) and Brean (England) over the Bristol Channel. The following year, he opened the world first radio factory, at Chelmsford, England. In the early 20th century the name Marconi is (unfortunately) best known as the owner of the Pathé cinema group (who's real complete name is Pathé-Marconi).

\parpic[l][t]{%
  \begin{minipage}{40mm}
    \fbox{\includegraphics[width=110px,height=140px]{img/medaillons/mandelbrot.eps}}
  \end{minipage}
}
\textbf{Mandelbrot, Benoit} (1924-2010) was born in Warsaw and died in Cambridge. His family left Poland for Paris to escape the Nazi threat. It was in Paris that he was introduced to mathematics by two uncles, whom one was a professor at the Collège de France. The German invasion forced the family to flee to Brive-la-Gaillarde. After attending high school at Edmond-Perrier de Tulle, he studied at the Lycée du Parc, in Lyon. After leaving the École Polytechnique (1944), where he studied with a specialist of probabilities (Paul Levy), he became interested in the phenomena of information (the ideas Claude Shannon were at this time in full growth). Mandelbrot made his main studies in France and in the United States and received his PhD in mathematics at the University of Paris in 1952. He taught economics at Harvard University, engineering at Yale, physiology at the Faculty of Medicine and mathematics in Paris and Geneva. From 1958, he worked for IBM at the Thomas B. Watson Research Center in New York on the optimal transmission in noisy environments, while continuing his work on strange objects neglected by mathematicians: objects with recursively defined complexity as the Von Koch curve, which he foresaw a unity: fractal geometry. Fractal geometry is characterized by a more abstract approach to the dimension as it is in the traditional geometry. It finds more and more applications in different fields of science and technology.

\parpic[l][t]{%
  \begin{minipage}{40mm}
    \fbox{\includegraphics[width=110px,height=140px]{img/medaillons/markov.eps}}
  \end{minipage}
}
\textbf{Markov, Andrei Andreyevich} (1856-1922) was a Russian mathematician specializing in number theory, theory of probability and mathematical analysis born in Ryazan and died in Petrograd. Coming from a family of a small government official, he studied at the University of St. Petersburg and received a gold medal for his thesis\textit{ On the integration of differential equations by the method of continued fractions} ( 1878). Professor at the University of St. Petersburg in 1886, he became a member of the Academy of Sciences in 1896. Markov's researches continue the work of his predecessors of the St. Petersburg mathematical school: P. L. Chebyshev, E. I. Zolotarev and A. N. Korkin. His thesis \textit{Bilinear quadratic forms with positive determinant} (1880) inaugurated his works in the field of number theory. In Analysis, his research concerned continued fractions, limits of integrals, series convergence and approximation theory. We owe him a simple solution for determining the upper limit of the derivative of a polynomial (Markov's inequality). After 1910, he turned to the theory of probabilities, and proved rigorously, under fairly general conditions, the central limit theorem on the sum of independent random variables. Trying to generalize this theorem to dependent random variables, he comes to consider the important notion of events chains, known as Markov chains, and establishes a series of laws, the foundation of the theory of Markov processes. He extends several classical results concerning independent events to certain types of chains. His work is at the origin of the modern theory of stochastic processes. Markov was also interested in applications of probability theory, and he justified in a probabilistic way the least squares method.

\parpic[l][t]{%
  \begin{minipage}{40mm}
    \fbox{\includegraphics[width=110px,height=140px]{img/medaillons/markowitz.eps}}
  \end{minipage}
}
\textbf{Markowitz, Harry Maurice} (1927 -) is born in Chicago. He was professor at the University of New York. Markowitz is known for having developed a famous portfolio model in his article \textit{Choice of investment portfolios for a fortune}. Markowitz did not suspect that his article published in 1952 in the Journal of Finance, when he was young, then developed in a book published in 1959, \textit{Portofolio Selection: Efficient diversification}, will lay the foundation of modern portfolio theory and be used by a large number of practitioners. More precisely, Markowitz showed that the investor seeks to maximize his choice, taking into account not only the expected profitability of investments, but also the risk of the portfolio defined mathematically by the variance of profitability. Applying classical theorems of statistical computing and probabilistic techniques, he has demonstrated that a portfolio of several shares is always less risky than a portfolio consisting of one share, even though it would be the least risky. Implementation of Markowitz has quickly raised practical problems. While the volume of data required to calculate increased rapidly with the number of shares held (with 100 shares, the number of necessary statistics was 3,150, but he passed 20,300 for 200 shares and to 125,750 for 300 shares!), information gathering and processing became almost impossible with the available computers in the 1960s, resulting in additional prohibitive treatment costs. This is why William F. Sharpe look for a method for selecting efficient portfolios easier. Markowitz and Sharpe will be recognized as the founding fathers of portfolio management and the doctrinal body on which it is based. The Nobel Prize in Economics will them be awarded as well as to Merton Miller in 1990.

\parpic[l][t]{%
  \begin{minipage}{40mm}
    \fbox{\includegraphics[width=110px,height=140px]{img/medaillons/markx.eps}}
  \end{minipage}
}
\textbf{Marx, Karl} (1818-1883) was born in Trier and died in London. Karl Marx entered the University of Bonn and after at the University of Berlin, after finishing high school in Trier. He studied law in Berlin, but also  history and philosophy. Marx then helped to complete the three main schools of thought of the 19th century: classical German philosophy, classical English political economy and French socialism. Marx's social theory aims to reveal the economic law of capitalist society where the production of goods dominates by seeking the origin of the value of money. Thus, for Marx, money (as the supreme product of the development of exchange and commodity production) fades and hide the character and social ties of individual work. At a certain stage in the development of commodity production, money is also transformed into capital. Thus, the sequence of movements of goods was: $G$ (goods) - $M$ (Money) - $G$ (goods), that is to say, selling a commodity to purchase another. The general sequence of capital is against $M-G-M$, that is to say, the purchase for sale (at a profit). It is this increase in the primitive value of money, so its transformation into capital, which Marx called "capital gain" and that can't come from the movement of goods, because this can only be done by the exchange of the counterparts; it can't either come from an increase in prices, as the reciprocal profits and losses of buyers and sellers equilibrate at large scale. To obtain capital gain it must be according to Marx a commodity whose process of consumption was at the same time a process of value creation. However, this commodity is the human labour. The possessor of money buys labour power at its value, determined as of any other commodity, by the labour time socially necessary for its production. Having bought the labour force, the owner of money is entitled to consume it, that is to say to oblige him to work all day, say, 8 hours. However, in 5 hours (necessary labour time), the worker creates a product that covers the cost of his own subsistence, and for the remaining 3 hours (overtime), it creates an additional product, unpaid by the capitalist, which is the capital gain. Also to express the degree of exploitation of labour by capital, we should compare the capital gain not against the total cost of production, but only to the variable cost of human labour.

\parpic[l][t]{%
  \begin{minipage}{40mm}
    \fbox{\includegraphics[width=110px,height=140px]{img/medaillons/maxwell.eps}}
  \end{minipage}
}
\textbf{Maxwell, James Clerk} (1831-1879) was born in Edinburgh and died in Glenlair. Brilliant student at High school, James Clerk Maxwell continue his studies of mathematics at the University of Cambridge. He obtained a chair of natural philosophy at Aberdeen at the age of twenty five years. Then, from 1860 to 1865, he served as professor at the King's College of London. Following these five years of teaching, he decided to retire to his property of Glenair, Scotland. He will stay there during five more years to study. In 1871, Maxwell was appointed director of the Cavendish Laboratory founded by the Duke of Devonshire. Maxwell will then cease to make it grows so that it becomes the most famous scientific training center. From the beginning of his career, Maxwell focuses on the dynamics of gas. After proving mathematically that the rings of Saturn are composed of discrete particles, he studied the velocity distribution of gas molecules (according to Gauss's law). In 1860, he shows that the kinetic energy of these molecules depends only of their nature. But it was his research in electromagnetism that make Maxwell one of the most known scientific of the 19th century. Based on the work of Faraday, he introduced in 1862 the concept of field. Then, he shows that a magnetic field can be created by varying an electric field (Faraday had discovered induction phenomenon in which the variation of an electric field creates a magnetic field). His purely mathematical teaching will then enable him to prepare the famous differential equations describing the nature of the electromagnetic fields in space and time. He describes them in his treatise On electricity and magnetism published in 1873. Maxwell, by developing the theories of electromagnetism, also defined light as an electromagnetic wave, thus paving the way for further research for other physicist like Heinrich Rudolph Hertz.

\parpic[l][t]{%
  \begin{minipage}{40mm}
    \fbox{\includegraphics[width=110px,height=140px]{img/medaillons/mcfadden.eps}}
  \end{minipage}
}
\textbf{McFadden, Daniel} (1937 -) born in Raleigh is an econometrician who received in 2000, with James Heckman, the Nobel Prize in Economics for his contributions to the theory and methods of discrete choice analysis. He obtained a Bachelor of Science in Physics at the age of nineteen at the University of Minnesota and a PhD in behavioural sciences (economics) five years later in 1962. In 1964, he joined the University of Berkeley and focuses his research on the behaviour of choice, and the links between economic theory and economic measures. In 1975, he was awarded for the John Bates Clark Medal. In 1977, he went to the Massachusetts Institute of Technology, but returned to Berkeley in 1991, as the MIT had no statistics department. After his return, he founded the Laboratory of Econometrics, which is devoted to statistical computing and applied to economics. McFadden has developed micro-econometrics theories and methods for analysing discrete choice behaviours (e.g. data on occupations and places of residence of individuals) and is also famous for his pseudo-R coefficient for the probit logistic regression. From his economic theory on discrete choices, McFadden has developed new statistical methods that have had a decisive influence on the theoretical research, but are also widely used by marketing.

\parpic[l][t]{%
  \begin{minipage}{40mm}
    \fbox{\includegraphics[width=110px,height=140px]{img/medaillons/meitner.eps}}
  \end{minipage}
}
\textbf{Meitner, Lise} (1878-1960) was a physicist born in Vienna and died in Cambridge. In 1899, Lisa began a two-year entry exam accelerated preparation to enter at University. She was received and entered the University of Vienna in 1901, at the age of twenty-two years. After the first year, during which Lise followed many courses in physics, chemistry, mathematics and botany, she focused on physics. From the second year, she chose to take all the courses given by Ludwig Boltzmann ; this reflects the fascination that great theoretical physicist exercised over his students, with whom he developed intellectual but also personal relations. She obtained her PhD in 1905. Lise remained in Vienna during the years that followed his doctorate. As a woman, she could not expect an academic career, but nevertheless continued research. She met Paul Ehrenfest, a former student of Boltzmann, who directed his attention on the articles published by Lord Rayleigh. One described an optical effect that Rayleigh could not explain. Lise founded the theoretical explanation and derived observations. Lise went to Berlin in 1907 to follow the course of Max Planck. Otto Hahn and Lise studied radioactivity and they became famous for their work, including the discovery of protactinium in 1918. Regardless of its work with Hahn, Lise led pioneering research in nuclear physics. She first devoted to the study of spectra of beta and gamma radiation. In 1923, she discovered the non-radiative transition known today as the "Auger effect", named in honour of Pierre Auger, a French scientist who discovered the effect independently two years later. She also discovered the emission of electron-positron pairs in the beta decay more. She made various measurements of the mass of the neutron. In 1939 she played a major role in the discovery of nuclear fission, that she provides with her nephew Otto Frisch, the first theoretical explanation in 1939 using the liquid drop model of Niels Bohr. This is why she is considered as the "mother of the nuclear bomb" by the media of his time.

\parpic[l][t]{%
  \begin{minipage}{40mm}
    \fbox{\includegraphics[width=110px,height=140px]{img/medaillons/mendel.eps}}
  \end{minipage}
}
\textbf{Mendel, Gregor Johann} (1822-1884) was born in Brno and died in Heinzendorf. He was a monk in the monastery of Brno (Moravia). Mendel is widely recognized as a botanist and the father of genetics. He is at the origin of what is now called Mendel's laws, which define the way genes are passed from generation to generation. Mendel was born in a peasant family. Having aptitudes for studies, but having also a depressive tendency which earned him multiple troubles later in his career, the boy was quickly identify by the village priest who decided to send him away from home to continue his studies. Mendel attend in 1851 classes as an auditor of the Institute of Physics of Christian Doppler. He studied in addition to the obliged subjects: botany, plant physiology, entomology and palaeontology. In two years, he acquired the methodological basis which will later give him the possibility to realize his experiences. During his stay in Vienna, Mendel is brought to focus on the theories of Franz Unger, professor of plant physiology. Franz Unger propose the experimental study to understand the emergence of new characteristics in plants over successive generations. He hopes to solve the problem of hybridization in plants. Back to the monastery, Mendel installs an experimental garden in the courtyard and in the greenhouse, in agreement with his abbot, and set up a plan of experiments to understand the laws of the origin and formation of hybrids. He chose for his experiences the pea which has the advantage of being easily cultivated with many known varieties. In 1865, he exhibited at the Society of Natural Sciences Brno and publishes in 1866 the results of its studies after 10 years of painstaking work, Mendel also laid the theoretical foundations of modern genetics and heredity. His work does not generate enthusiasm among his contemporaries who are struggling to understand the mathematical experiences. Very few scientists of his time will speak about his work and Mendel gets only some answers from various correspondents. Of these, only Karl Wilhelm von Nägeli, professor of botany at Munich, wrote him doubting also some of his conclusions. In 1868, Mendel was elected superior of the convent after the death of the abbot.

\parpic[l][t]{%
  \begin{minipage}{40mm}
    \fbox{\includegraphics[width=110px,height=140px]{img/medaillons/mendeleiev.eps}}
  \end{minipage}
}
\textbf{Mendeleev, Dmitri Ivanovich} (1834-1907) was a Russian born in Tobolsk and died in St. Petersburg. He was a famous chemist best known for his periodic table of elements published in 1869. He showed indeed that the chemical properties of elements directly dependent on their atomic weight and that were a periodic functions of that weight. He entered at the age of fourteen years at the Tobolsk high school, after the death of his father. In 1849, the family who became poor moved to St. Petersburg and Mendeleev entered to university in 1850. After graduation, he contracted tuberculosis which forced him to move in the Crimean Peninsula near the Black Sea in 1855, where he became head of the local high school of science. He returns completely healed in St. Petersburg in 1856 where he also studied chemistry and became graduated in 1856. At the age of twenty-five, he works at Heidelberg with scientists like Robert Bunsen and Gustav Kirchhoff. At Heidelberg, he met the Italian chemist Stanislao Cannizzaro, whose ideas on the atomic weight influenced his thinking. Mendeleev returned to St. Petersburg and taught chemistry at the Technical Institute in 1863. He was appointed professor of general chemistry at the University of St. Petersburg in 1866.

\parpic[l][t]{%
  \begin{minipage}{40mm}
    \fbox{\includegraphics[width=110px,height=140px]{img/medaillons/merton.eps}}
  \end{minipage}
}
\textbf{Merton, Robert Cox} (1944-) received the Nobel Prize in Economics in 1997 along with his compatriot Myron Scholes for their development of the evaluation of financial derivatives. This method of evaluation has certainly accelerated the rapid growth of derivatives markets since the 1980s and led to improved management of risks associated with these new financial products. Merton has undoubtedly helped to open a new path in the field of economics and strongly influenced the other two winners. Born in 1944 in New York, he left the California Institute of Technology with a master's degree in Applied Mathematics. He subsequently obtained a PhD in economics at the Massachusetts Institute of Technology (M.I.T.) in Cambridge, under the direction of Paul Samuelson (Nobel Prize for Economics 1970) and specializes in problems of application of probabilistic methods to random evolution of financial asset prices. In 1988 he held the George Fischer Backer chair as professor in Business Administration at Harvard Business School in Cambridge. The pioneering work of Merton start from the early 1970, period during which he develops a new method of calculating the value of derivatives. The failure of his method applied to the management of an investment american fund risk (Long-Term Capital Management) in 1998, has somewhat tarnished its reputation as a specialist in international finance. But Merton himself had told a U.S. television network, following the award of the prize, that is a misunderstanding to think that we can eliminate the risk simply because we understood and can measures them.

\parpic[l][t]{%
  \begin{minipage}{40mm}
    \fbox{\includegraphics[width=110px,height=140px]{img/medaillons/minkowski.eps}}
  \end{minipage}
}
\textbf{Minkowski, Hermann} (1864-1909) was born in Alexotas (Russia) and died in Göttingen (Germany). He was a mathematical physicist who studied at the universities of Berlin and Königsberg. He studied at the high school of Königsberg where he was recognized for his performance in mathematics and he received his PhD in 1885 in the same city. He then taught at several universities in Bonn, Königsberg and Zurich. In Zurich, Einstein was a student in several of his lectures. Minkowski accepted a professorship in 1902 at the University of Göttingen, where he remained for the rest of his life. In Göttingen, he learned the physic-mathematics of Hilbert, he participated to a conference on the theory of the electron in 1905 and learned the latest results in the theory of electrodynamics. In 1907 Minkowski realized that the work of Lorentz and Einstein could be better understood in a non-Euclidean space. He considered space and time, which was previously thought to be independent, to be coupled together in a continuum four-dimensional space-time. Minkowski has established a four-dimensional treatment of electrodynamics. This space-time continuum has provided a framework for all later mathematical works in relativity. These ideas have been used by Albert Einstein in developing the general theory of relativity. Minkowski was mainly interested in pure mathematics and has spent much of his time studying quadratic forms and continued fractions. However, his most original work was his Geometry of numbers. This study has led to work on convex bodies and to questions about packing problems (ways in which the figures of a given form can be placed in another given figure).

\parpic[l][t]{%
  \begin{minipage}{40mm}
    \fbox{\includegraphics[width=110px,height=140px]{img/medaillons/mobius.eps}}
  \end{minipage}
}
\textbf{Möbius, August Ferdinand} (1790-1868) was a German mathematician and astronomer born in Schulpforta and died in Leipzig. Möbius was educated at Leipzig, Göttingen (under the direction of Gauss) and Halle. In 1815 he became professor of astronomy at Leipzig, then director of the observatory of the city, after having directed its construction. He has written several books of theoretical astronomy, including \textit{De computandis occultationibus fixarum per planetas} (1815). His mathematical works concerned mainly geometry and were, for the most part, published in the Journal of Pure and Applied Mathematics of Crelle, from 1828 to 1858, as a complement to his fundamental book \textit{Der barycentrische Calculation} (1827). By introducing a new coordinate system, Möbius studies geometric transformations, mainly projective transformations. His work had a great importance in the development of projective geometry. Studying the static in terms of geometry, Möbius also developed the theory of linear complexes of lines (Lehrbuch der Statik, 1837). Möbius can be considered as one of the pioneers of topology, with the discovery, published in a submission to the French Academy of Sciences, of the famous "Mobius Surface", with only one side.

\parpic[l][t]{%
  \begin{minipage}{40mm}
    \fbox{\includegraphics[width=110px,height=140px]{img/medaillons/monge.eps}}
  \end{minipage}
}
\textbf{Monge, Gaspard} (1746-1818) was born in Beaune and died in Paris. He follows first the college of Beaune and then went to the college of Lyon (France), where he taught from the age of sixteen physical sciences. An engineering officer, who had seen a map of the town of Beaune made by Monge using new methods of observation and construction graph, recommends Monge to the commander of the military school of Mézières. But he can't be accepted because of its common origin and is accepted only in a technical annex of the school. His scientific talents are recognized when one day he draws the plan of fortifications using a method much faster than previously known methods. He is then admitted to the military school as a mathematics teacher and continued his research, arriving at the general method of geometric representation known since as under the name of "Descriptive Geometry". But his discoveries, considered as valuable military secrets, can't be published. In 1780 he went to Paris to teach hydrodynamics. He immediately entered the Academy of Sciences, where he made a presentation on the lines of curvature drawn on a surface (problem already studied by Euler in 1760). In 1786, he published his famous \textit{Traité élémentaire de la statique} and soon after founded the École Polytechnique, where he had the opportunity to teach descriptive geometry and publish his works hitherto unknown. Chargé de Mission in Italy, Monge meets Bonaparte and is defined as responsible for recruiting scientists for the Egyptian expedition. Back in France, he resumed his education at the École Polytechnique became a senator and was knighted. But the Restoration deprive him of all titles, it will scratch Monge of the list of members of the Institute and will take him away his teaching position. In 1989, his ashes were transferred to the Panthéon. All his research closely intertwined pure geometry, infinitesimal analysis and analytical geometry, allowing, for example, to link each family of surfaces with a partial differential equation, and hence, to find solutions to differential equations using his theory of surfaces. The influence of Monge exerted trough his oral teaching, most of the 19th century French mathematicians that were his students.

\phantomsection
\addcontentsline{toc}{section}{N}
\label{sec:N}

\parpic[l][t]{%
  \begin{minipage}{40mm}
    \fbox{\includegraphics[width=110px,height=140px]{img/medaillons/napier.eps}}
  \end{minipage}
}
\textbf{Napier, John} (1550-1617) was a physicist, astronomer, mathematician and theologian born and died at Merchiston (U.K.). As it was the common practice for members of the nobility at that time, John Napier did not enter schools until he was thirteen. However he did not stay in school very long. It is believed that he dropped out of school in Scotland and perhaps travelled in mainland Europe to better continue his studies. In 1571 Napier, aged twenty-one, returned to Scotland, and bought a castle at Gartness in 1574. On the death of his father in 1608, Napier and his family moved into Merchiston Castle in Edinburgh, where he resided the remainder of his life. Mathematics were not his main activity but he had a lot of ideas to simplify calculations. He establishes some formulas of spherical trigonometry, popularized the use of the point to the English notation of decimal numbers and especially invented logarithms. His objective was to simplify trigonometric calculations needed in astronomy. He defined the logarithm of a sine based on mechanical considerations of moving points and the link between the arithmetic and geometric progressions.

\parpic[l][t]{%
  \begin{minipage}{40mm}
    \fbox{\includegraphics[width=110px,height=140px]{img/medaillons/navier.eps}}
  \end{minipage}
}
\textbf{Navier, Henri} (1785-1836) was born in Dijon and died in Paris. He was an engineer, mathematician and economist best known for his work on hydrodynamics. Henri was orphaned at age of nine, after the death of his father, renowned lawyer and former member during the Revolution. His uncle, engineer at the Corps des Ponts et Chaussées took in charge his education in Paris and consider him as his son before adopting him with his wife, also a close relative of the young Henri. His uncle force him to attend the École Polytechnique. Although one of the last received in 1802, he succeeded his schooling and its classification allows him to integrate the Corps des Ponts et Chaussées. He was appointed resident engineer of Ponts et Chaussées in 1808. Later, he became divisional inspector of this Corpse, and it seems that for a time, General Inspector like his uncle. From 1819 to 1835, he provides the course of Applied Mechanics of the École Nationales des Ponts et Routes (he is nominated professor in 1830 following the retirement of Eisenmann). In the early 1820s, he explores with Augustin-Louis Cauchy aspects of the mathematical theory of elasticity, which allows him to propose the motion equations of Newtonian fluids.

\parpic[l][t]{%
  \begin{minipage}{40mm}
    \fbox{\includegraphics[width=110px,height=140px]{img/medaillons/nash.eps}}
  \end{minipage}
}
\textbf{Nash, John} (1928-2015) was born in West Virginie (U.S.A.) and die in Monroe Township (New Jersey) at the same time as his wife (physicist) of a car accident. Son of John Nash Sr., an engineer, and Virginia Martin, a teacher he spent a lot of time reading and experimenting in his room that he had converted into a small laboratory. From 1945 to  1948, Nash studied at the Carnegie Institute of Technology in Pittsburgh, intending to become an engineer like his father. Instead, he developed an enduring passion for mathematics, and in particular the theory of numbers, Diophantine equations, quantum mechanics and relativity theory. He was admitted at the graduate level at the age of twenty years in all universities he had requested: Harvard, Princeton... He chose to go to Princeton. Having an interest in economics, Nash began to study game theory, an area that had been cleaned by John von Neumann, one of the great names of Princeton, a little over a decade ago. It is on this subject that he decided to make his thesis and he won the Nobel Prize for Economics in 1994. During the summer of 1950, Nash was employed as a consultant at RAND, top-secret institute that employed brainpower to develop various strategies of status quo of victory, in cases of conflict involving nuclear weapons. Nash began to study the compact smooth manifolds, which was the subject of a paper. He then became assistant at M.I.T. in 1951 to 1952, at only twenty-three years old. He really had the temperament of a problem-solver and raised the challenge of solving a question of Waren Ambros: Is it possible to dive any Riemannian manifolds in Euclidean space? Nash found a fundamental original method to achieve this problem. Nash became ill after some personal and professional problems, but he attributed his illness to his attempt to resolve the contradictions of quantum physics. Especially since shortly before he had completed work on non-linear elliptic PDE which earned him much admiration around him, but he finally had to share paternity with a young Italian who had set, independently and a few weeks before him, similar results: This earned them not getting the Fields Medal in 1958...

\parpic[l][t]{%
  \begin{minipage}{40mm}
    \fbox{\includegraphics[width=110px,height=140px]{img/medaillons/neyman.jpg}}
  \end{minipage}
}
\textbf{Jerzy, Neyman} (1894 – 1981) was a Polish mathematician and statistician who spent the first part of his professional career at various institutions in Warsaw, Poland and then at University College London, and the second part at the University of California, Berkeley. Neyman first introduced the modern concept of a confidence interval into statistical hypothesis testing and co-revised Ronald Fisher's null hypothesis testing (in collaboration with Egon Pearson).  He graduated from the Kamieniec Podolski gubernial gymnasium for boys in 1909. He began studies at Kharkov University in 1912, where he was taught by Russian probabilist Sergei Natanovich Bernstein. After he read 'Lessons on the integration and the research of the primitive functions' by Henri Lebesgue, he was fascinated with measure and integration. In 1921 he returned to Poland in a program of repatriation of POWs after the Polish-Soviet War. He earned his Doctor of Philosophy degree at University of Warsaw in 1924 for a dissertation titled \textit{On the Applications of the Theory of Probability to Agricultural Experiments}. He spent a couple of years in London and Paris on a fellowship to study statistics with Karl Pearson and Émile Borel. After his return to Poland he established the Biometric Laboratory at the Nencki Institute of Experimental Biology in Warsaw. He published many books dealing with experiments and statistics, and devised the way which the FDA tests medicines today. Neyman proposed and studied randomized experiments in 1923. He introduced the confidence interval in his paper in 1937. Another noted contribution is the Neyman–Pearson lemma, the basis of hypothesis testing. In 1938 he moved to Berkeley, where he worked for the rest of his life. Thirty-nine students received their PhDs under his advisorship. In 1966 he was awarded the Guy Medal of the Royal Statistical Society and three years later the U.S.'s National Medal of Science. He died in Oakland, California in 1981.

\parpic[l][t]{%
  \begin{minipage}{40mm}
    \fbox{\includegraphics[width=110px,height=140px]{img/medaillons/newton.eps}}
  \end{minipage}
}
\textbf{Newton, Isaac} (1642-1727) was a mathematician and physicist, considered as one of the greatest scientists in history. Newton was born in Lincolnshire (England), from peasant parents and died in London. At the age of five, he attended primary school at Skillington, then at the age of twelve that of Grantham. He will stay there four years until his mother order him to come back at Woolsthorpe to become a farmer and learn how to administer his domain. However, his mother, seeing that her son was better in mechanics than in livestock, allowed him to return to school to be able perhaps one day to enter at the university. At the age of seventeen, Newton falls in love with a classmate, miss Storey. He is authorized to have her as girlfriend and even got engaged with her, but he must first finish his studies before getting married. Finally, the marriage did not happen and Newton will be single all his life. At age of eighteen, he entered the Trinity College of Cambridge (he will stay there seven years), where he was noticed by his teacher, Isaac Barrow. He also have for professor Henry More who will have a great influence in his conception of absolute space. At Cambridge, he studied arithmetic, geometry in Euclid's Elements and trigonometry, but is particularly interested in astronomy, alchemy and theology. He receive at the age of twenty-five his bachelor of arts, but was forced to suspend his studies for two years following the emergence of the plague that struck the city in 1665. He then returned to his native region. It is during this period that Newton grew strongly in mathematics, physics and especially in optics. He gave important contributions to many areas of science. His discoveries and theories were the basis of much scientific progress after him. Newton was one of the inventors of the branch of mathematics named "infinitesimal calculus" (another inventor was the German mathematician Gottfried Wilhelm Leibniz). He also clarifies the mysteries of light and optics, formulate the three laws of motion and derived the law of universal gravitation based on Kepler's laws. He reached the argument that light is a mixture of different rays of different colors, and because of the phenomena of reflection and refraction, colors appear in separate components. Newton proved his theory of colors by passing light through a prism, which splits the light beam into separate colors. In 1696, he left Cambridge to become the first guardian of the Royal Mint and Master of the Mint in the following year. In 1699, he was promote member of the Royal Society and is elected president in 1703. He held this position until his death.

\parpic[l][t]{%
  \begin{minipage}{40mm}
    \fbox{\includegraphics[width=110px,height=140px]{img/medaillons/neumann.eps}}
  \end{minipage}
}
\textbf{Neumann Von, John} (1903-1957) was a mathematician born in Budapest and died in Washington. Von Neumann was a child prodigy: at the age of six, he converses with his father in ancient Greek and can mentally divide a 8-digit number. An anecdote relates that at the age of only eight years old, he has already read the 44 volumes of the universal history of the family library and he has completely memorized it: with an absolute memory, he is able to quote from memory entire pages of the books readen years ago. He entered the school in Budapest in 1911. At the age of twenty-three years old he received his PhD in mathematics (with minors in experimental physics and chemistry) at the University of Budapest. In parallel, he earned a degree in chemical engineering from the ETH Zurich in Switzerland (at the request of his father, wanting his son to invest in a more remunerative than mathematics). It is interesting to note that von Neumann never followed the courses and went in these two universities only for the exams. Between 1926 and 1930 he was Privatdozent in Berlin and Hamburg (Germany). He also worked with Robert Oppenheimer in Göttingen under the supervision of David Hilbert. During this period, one of the most fruitful of his life, he is also near of Werner Heisenberg and Kurt Gödel. In 1930, von Neumann was invited professor at Princeton University. Then, from 1933 to his death in 1957, he was professor of mathematics at the Faculty of the Institute for Advanced Study that has just been created. He joins there Albert Einstein and Kurt Gödel. Neumann emigrated to the United States in 1933 to join the Institute for Advanced Research in Princeton. He wrote an important book on Applied Mathematics and made a major work in the axiomatization of quantum physics (he founded that a quantum system can be considered as a point in a Hilbert space and introduced linear operators). He participated during the Second World War to the theoretical development of the atomic bomb and the study of shock waves. His mathematical work on ultra-fast simulations of the H-bomb, helped in the development of computers (he is also at the origin of Monte-Carlo method). He also contributed to the theory of games where some of these results had a great influence on the economy.

\parpic[l][t]{%
  \begin{minipage}{40mm}
    \fbox{\includegraphics[width=110px,height=140px]{img/medaillons/abel.eps}}
  \end{minipage}
}
\textbf{Niels, Abel} (1802-1829) was a Norwegian mathematician born in Frindoë and died in Froland. His father was a known Norwegian politician, but at the end of his life, he fell into disgrace, and when he died in 1820, it is Abel who had to bear the entire burden of the family. His father educated Abel himself until 1815, then sent him to the parochial school in Oslo. In this school Latin and Greek religion were taught with the old traditions, with corporal punishment. The situation changed in 1817 after the dismissal of a teacher following the death of a student: the school hired then a young teacher open to new ideas and knowing mathematics. This new teacher discovered that Niels was interested in mathematics, he founded him a scholarship for University. With the financial assistance from his teachers, he manages however his studies and make his first discoveries. But his works were lost by Cauchy and underestimated by Gauss. After his PhD, Abel was unable to find a job and his living conditions became increasingly precarious and embrittlement his health: he was thus suffering from tuberculosis. Despite trips to Paris and Berlin, his works are still not perceived at their true value. In his last weeks, he no longer has enough strength to leave it's bed. He died at only twenty-seven years old, while a friend just find him a job in Berlin. It is Jacobi who will understand the genius of the young mathematician. Abel had especially proved at the age of nineteen years, the impossibility of solving algebraic equations of the 5th degree by radicals, result that his contemporary Galois generalized to any degree. Posthumously, in 1830 Abel receive the grand prize of Mathematics of the France Institute.

\parpic[l][t]{%
  \begin{minipage}{40mm}
    \fbox{\includegraphics[width=110px,height=140px]{img/medaillons/noether.eps}}
  \end{minipage}
}
\textbf{Nöther, Emmy} (1882 -1935) was born in Erlangen and died in Princeton. Emmy considered first teaching French and English after passing the required examinations, but finally studied mathematics at the University of Erlangen, where her father was a lecturer. During the winter semester of 1903-1904, she studied at the University of Göttingen and attended the courses of the astronomer Karl Schwarzschild and mathematicians Hermann Minkowski, Felix Klein and David Hilbert. After completing his PhD in 1907 she worked for free at the Mathematics Institute in Erlangen during seven years. In 1915, she was invited by David Hilbert and Felix Klein to join the renowned Department of Mathematics at the University of Göttingen until 1933. In 1935, she was operated because of an ovarian cyst and, despite signs of recovery, died four days later at the age of fifty-three years old. She remains in the history of mathematics as the main founder of abstract algebra or modern algebra, which is one of the essential branches of contemporary mathematics. This abstract algebra takes importances compared to calculations performed in various sets, defined with various operations, and shows what these calculations have in common. In physics, Noether's theorem explains the fundamental connection between symmetry and conservation laws. His ideas have also contributed to the advancement of physics, in particular in the theory of relativity. Despite all her qualities, she had difficulties to lead a normal career as a university professor, because she was a woman in an exclusively male environment. However she enjoyed the esteem and support of David Hilbert, Albert Einstein and Felix Klein.

\phantomsection
\addcontentsline{toc}{section}{O}
\label{sec:O}

\parpic[l][t]{%
  \begin{minipage}{40mm}
    \fbox{\includegraphics[width=110px,height=140px]{img/medaillons/ohm.eps}}
  \end{minipage}
}
\textbf{Ohm, Georg Simon} (1789-1854) was a physicist born in Erlangen and died in Munich (Germany). Although his parents had not made higher education, Ohm's father was a respected man and an autodidact who himself gave his son an excellent education. From his earliest childhood Georg received from his father's very good teachings in physics, mathematics, chemistry and philosophy. Georg attended the school of Erlangen from eleven to fifteen years old and where he received a very limited scientific education, in contrast with the teachings of his father. In 1805, at the age of fifteen, Ohm entered the University of Erlangen. Ohm is dissipated as his father angry at the waste of its potential, sent him to Switzerland where, in 1806, he took up a post as a mathematics teacher in the school of Gottstadt bei Nydau. Ohm left his teaching position at Gottstadt bei Nydau in 1809 to become a private tutor in Neuchâtel (Switzerland) for two years. Then in 1811 he returned to the University of Erlangen. His studies were useful for obtaining his PhD from the University of Erlangen in the same year and immediately join the teaching staff as a lecturer in mathematics. The king Frederick William III of Prussia offered him a position at the Jesuit school of Cologne in 1817. Thanks to the reputation of this school in the teaching of science, Ohm is found to teach both mathematics and physics. The physics laboratory is well equipped, he devoted himself to experimentation. What is now known as Ohm's law appeared in 1827 in the book \textit{Die Kette galvanische, Mathematisch bearbeitet} in which he provides a complete theory of electricity. He entered the Polytechnic School of Nuremberg in 1833 and in 1852 became professor of experimental physics at the University of Munich, where he died later.

\parpic[l][t]{%
  \begin{minipage}{40mm}
    \fbox{\includegraphics[width=110px,height=140px]{img/medaillons/oppenheimer.eps}}
  \end{minipage}
}
\textbf{Oppenheimer, J. Robert} (1904-1967) was a physicist born in New York and died in Princeton. He was the scientific director of the Manhattan Project and also managed the development project of the first atomic bombs. He entered Harvard with a year's delay due to an attack of ulcerative colitis, he took advantage of this period to visit with his former English teacher in New Mexico. He became an amateur of horse riding as well as the mountains and plateaus of this region. Upon his return, he graduated in Chemistry. Percy Bridgman made him discovered experimental physics. It was during his studies at the Rutherford Ernest Cavendish Laboratory of Cambridge that he realizes that he master better the theory than experiments due to his clumsiness. In 1926, he continued his studies under the direction of Max Born at the University of Göttingen (Germandy) and obtained his PhD at the age of twenty-two. At Göttingen, he publishes articles on quantum theory. In 1927, he returned to Harvard and the following year at the Institute of Technology California. He is also known for his contribution to the quantum theory and the theory of relativity, and for studies of cosmic rays, positrons and neutron stars. He made important research in astrophysics, nuclear physics, and spectroscopy. He then discovered the Born-Oppenheimer approximation.

\parpic[l][t]{%
  \begin{minipage}{40mm}
    \fbox{\includegraphics[width=110px,height=140px]{img/medaillons/ostrogradsky.eps}}
  \end{minipage}
}
\textbf{Ostrogradsky, Mikhail Vasilyevich} (1801-1862) was an Ukrainian physicist and mathematician. He began his studies in mathematics at the University of Kharkov, and then went to Paris where he was in close contact with the famous French mathematicians Cauchy, Binet, Fourier and Poisson. Back in his homeland, he taught at the School of the Marine Cadet, at the Nicolas Academy of Engineering and at the Artillery School of St. Petersburg. He is famous in particular for establishing the flow divergence theorem, which allows to express the integral over a volume (or triple integral) of the divergence of a vector field as the surface integral (double integral extended to the area surrounding the volume) of the flow defined by this field. He was elected at the American Academy of Arts and Sciences in 1834, the Academy of Sciences of Turin in 1841, and the Academy of Sciences in Rome in 1853. Finally he was elected corresponding member of the Academy of Sciences of Paris in 1856. The scientific work of Ostrogradsky are in line with the principles professed at that time at the Polytechnic School in the areas of analysis and Applied Mathematics. In mathematical physics, he imagined a synthesis which would embrace the hydro mechanical theory of elasticity, the theory of heat and the theory of electricity under one uniform method.

\phantomsection
\addcontentsline{toc}{section}{P}
\label{sec:P}

\parpic[l][t]{%
  \begin{minipage}{40mm}
    \fbox{\includegraphics[width=110px,height=140px]{img/medaillons/pareto.eps}}
  \end{minipage}
}
\textbf{Pareto, Vilfredo} (1848-1923) was an Italian economist and sociologist, whose most famous contribution to economic theory is the definition of the concept of economic optimum. Born in Paris from an Italian father in exile and a French mother, he returned to Italy at the age of ten. He studied at the University of Turin and became an engineer. In 1893, he was appointed to the chair of political economy at the University of Lausanne (he died in Celigny Switzerland), where he succeeded Léon Walras. Among his works we found the analysis of expectations of economic agents. The fact that they are not independent of each other may give rise to movements of opinion that generate pessimistic crises. Pareto is also the father of the concept of optimum: The economy is optimum when the situation of an agent can not be improved without damaging at least one other agent. This concept is widely used in economics, because it allows to take into account the non-additivity of utilities of different agents. Competition achieves the Pareto optimum. Pareto has also integrated the indifference curves (formalized by Francis Edgeworth) to the Walrasian general equilibrium logic. The sociological work of Pareto was most discussed. In the \textit{Traité de sociologie générale}, published in 1916, he presented his theory of elites, giving that government power in all societies is the subject of a battle between the elites only. This thesis discredited democracies, and implicitly contributed to the development of fascism in Italy.

\parpic[l][t]{%
  \begin{minipage}{40mm}
    \fbox{\includegraphics[width=110px,height=140px]{img/medaillons/pascal.eps}}
  \end{minipage}
}
\textbf{Pascal, Blaise} (1623-1662) was a mathematician, physicist, theologian, mystic, philosopher, moralist and polemicist born in Clermont and died in Paris. Precocious child (at the age of eleven, he composed a short treatise on the sounds of vibrating bodies and proved the 32nd proposition of the first book of Euclid, at the age of sixteen he wrote a treatise on conics), he was educated by his father who was a mathematician. The earliest works of Pascal concern natural and applied sciences. He contributed significantly to the study of fluids. He clarified the concepts of pressure and vacuum by expanding the work of Torricelli. The extent of the areas of interest and genius of Pascal is impressive: inventor of the calculating machine, designer of the first transports in France, architect of the Poitevin marshes drying, he was also one of the finest prose writers of the French language and one of the greatest figures of the 17th century.

\parpic[l][t]{%
  \begin{minipage}{40mm}
    \fbox{\includegraphics[width=110px,height=140px]{img/medaillons/pauli.eps}}
  \end{minipage}
}
\textbf{Pauli, Wolfgang} (1900-1958) was an Austrian physicist, born in Vienna (Austria) and died in Zürich (Switzerland), known for his definition of the exclusion principle in quantum mechanics, for which he received the Nobel Prize in Physics in 1945. Pauli was born from a father who was a university professor and a mother who was journalist and lawyer. At High school in Vienna, Pauli was considered as a prodigy child in mathematics. In 1919, he began his studies in physics at the University of Munich with the Professor Arnold Sommerfeld. Since 1898, Sommerfeld was in charge of writing the 5th volume of the \textit{Enzyklopädie der Wissenschaften Mathematischen} (20,000 pages) devoted to physics. He requires at first the collaboration of Albert Einstein to write the article on relativity, but he refuses. Sommerfeld then asks Pauli, whose speciality was the relativity during registration to Sommerfeld courses. Thus, at the age of twenty-one, Pauli published his article summarizing the theories of relativity and General Relativity. In 1921, he obtained his PhD on the subject of the hydrogen atom, where he clearly showed the limits of the model of the Bohr's atom, on which he worked as an assistant with Max Born in Göttingen between 1921 and 1922. During the years 1922 and 1923, he worked alongside Niels Bohr in Copenhagen. Between 1923 and 1928, he taught at Hamburg before leaving at the Zürich ETH, where he obtained a professorship in theoretical physics. In 1935, he moved to the United States, where he held invited professor status, including at the Institute for Advanced Study at Princeton during the years 1935-1936, but also at the University of Michigan, in 1931 and 1941, and at Purdue University in 1942. In 1946, he obtained the U.S. citizenship, but returned the same year at the ETH Zurich, where a place as teacher had been kept. In 1949, he became a Swiss citizen. In the 1950s, he regularly returns to Princeton to teach as a visiting professor. In the last years of his life, he participated in the founding of CERN. He died of a peptic ulcer.

\parpic[l][t]{%
  \begin{minipage}{40mm}
    \fbox{\includegraphics[width=110px,height=140px]{img/medaillons/pearson.eps}}
  \end{minipage}
}
\textbf{Pearson, Karl} (1857-1936) was a British mathematician founder of modern statistics born in London and died in Surrey. Statistical analysis has been a great development in the late 19th century in the United Kingdom and Karl Pearson dominates his contemporaries by the extent and variety of his contributions instead having interests in statistics starting only at the age of thirty-three. He develops analytical methods for the study of natural selection and eugenics which he is an ardent promoter. His main contributions are the creation of the test of independence chi-square for judging whether differences in a set of variables with respect to the theoretical values can be assigned or not a random sample and the definition of the correlation coefficient. He received the Darwin medal (biology) in 1898. Pearson was also a business consultant. He also taught statistics at William S. Gosset who introduced the Student law in 1910. He is one of the founders of Biometrika which he was editor for 36 years and has grown to become the best review in mathematical statistics.

\parpic[l][t]{%
  \begin{minipage}{40mm}
    \fbox{\includegraphics[width=110px,height=140px]{img/medaillons/penrose.eps}}
  \end{minipage}
}
\textbf{Penrose, Roger} (1931 -) is a physicist and mathematician born in Colchester (United Kingdom). Penrose get graduated in mathematics from the London's College University and get his PhD from Cambridge University with a thesis on tensor methods in algebraic geometry. Between 1964 and 1973, he taught mathematics at the Birkbeck London's College and meets the famous physicist Stephen W. Hawking with whom he worked on a theory of the origin of the universe by contributing to the mathematical theory of General Relativity applied to cosmology and the study of black holes. In 1965, at Cambridge, he proves that gravitational singularities can be formed by gravitational collapse of massive stars at the end of their life. In 1971, Penrose discovers spin networks that would later form the geometry of space-time in loop quantum theory. Professor at Oxford, he received, with Hawking, the 1988 Wolf Prize for physics.

\parpic[l][t]{%
  \begin{minipage}{40mm}
    \fbox{\includegraphics[width=110px,height=140px]{img/medaillons/picard.eps}}
  \end{minipage}
}
\textbf{Picard, Charles-Emile} (1856-1941) born and died in Paris made his classical studies at the  École Supérieure de Vanves in 1864, then at the École Supérieure Napoléon (the future Henry IV High School) from 1868 to 1874 where he proved that he was an excellent student, but not really attracted by mathematics. He obtain in 1874 a Bachelor of Arts and the following year a B.Sc. He is received second at the École Polytechnique, and first at the École Normale Supérieure. Finally, passionate for science, he choose this subject to pass the aggregation in 1877. After various assistant positions in Paris and Toulouse, in 1881 he became professor at the École Normale Supérieure. His name is already famous in the circle of mathematicians, because he proved an important theorem on singularities of holomorphic functions which earned him a nomination for membership int the Academy of Sciences. But he is too young, and his election was postponed to 1889. In 1885, Picard was appointed professor at the Sorbonne, where he holds the Chair of differential calculus. Again, his age is a problem (must be at least 30 years for such a position) and it was used a clever procedure to circumvent the legislation. Later, Picard occupy the chair of analysis and algebra, and also exercise at the Central School of Arts and Manufacture (1894-1937): there he trained more than 10,000 mechanical engineers, and is according to Hadamard, a great teacher. Picard's work is difficult, and pave the way for further research. He is the first to use the fixed point theorem in a method of successive approximations, which permit to solve partial differential equations. We also own him works in algebraic geometry, as more applied research on the elasticity or heat. He is also an early defensor of the theories of Einstein. His \textit{Traité d'Analyse} was long considerate as a reference, and Picard was also a philosopher and historian of science. Among the distinctions that Picard has received, he was the president of the International Congress of mathematicians, he was elected to the Académie Française in 1924, and he received the Mittag-Leffler gold medal in 1937.

\parpic[l][t]{%
  \begin{minipage}{40mm}
    \fbox{\includegraphics[width=110px,height=140px]{img/medaillons/planck.eps}}
  \end{minipage}
}
\textbf{Planck, Max} (1858-1947) was a German physicist born in Kiel and died in Göttingen considerate as the founder of quantum physics. After receiving his bachelor's at the age of seventeen in Munich where his father taught, Max Planck went study physics in Berlin. Fascinated by thermodynamics, he supports a thesis on the second law of thermodynamics and the concept of entropy in 1879, which will remain the main concept explaining the majority of his researches. The following year, he became a lecturer at the University of Munich and then became professor of physics at the University of Kiel in 1885. Four years later, he is professor of physics at the University of Berlin, where he worked for nearly 40 years. In 1930 he became director of the Kaiser Wilhelm Institute for Scientific Research, which will soon have his name. Initiated by his doctoral thesis, the research of Planck in thermodynamics are quickly oriented on the black-body. Entity purely theoretical, the black-body absorbs all radiation it receives (the black carbon, absorbing 97\% of the radiation, is close to this ideal). To explain this phenomenon, Planck developed a new theory. He speculates that the energy of radiation can be emitted or absorbed by matter only in finite quantities, the quanta. He then shows that these "energy packets" are set to $h\nu$, where $\nu$ is the frequency of the radiation and $h$ is a universal constant (the "Planck constant"). Explaining his theory to the German Physical Society in 1900 in Berlin, Planck does not yet know that he has invented a new branch of physics: quantum physics. His discovery will then be at the origin of the creation of the atom model by Niels Bohr, the development of wave mechanics by Louis de Broglie, the explanation of the photoelectric phenomenon by Albert Einstein and the discovery of the uncertainty principle by Werner Heisenberg. Considered as one of the most famous physicists, Planck received the Nobel Prize in 1918.

\parpic[l][t]{%
  \begin{minipage}{40mm}
    \fbox{\includegraphics[width=110px,height=140px]{img/medaillons/poincare.eps}}
  \end{minipage}
}
\textbf{Poincaré, Henri} (1854-1912) was a French mathematician and physicist born in Nancy, died in Paris, who was said that he was the last scientist knowing all the mathematics of his time. Exceptional student at the Lycée Impérial de Nancy, he obtained in 1871 a Bachelor of Arts, with honours, and the same year his B.Sc. He ranks first in the entrance examination at the École Polytechnique in 1873, then at the École des Mines de Paris, as engineer at the Corps des Mines, in 1875. He obtained his PhD in 1876. Named 3rd class engineer in 1879 at Vesoul, he obtained the same year his PhD in mathematics at the Faculty of Sciences in Paris, and became a lecturer in analytical science at the faculty of Caen. The first works of Poincaré are on Fuchsian automorphic functions, the qualitative theory of differential equations and the theory of functions. In a series of 6 articles published from 1894, he is the creator of algebraic topology, expanding science in the 20th century and in which more conjectures due to Poincaré remains open. He was also deeply interested in celestial mechanics: \textit{Les Méthodes nouvelles de la mécanique céleste}, three volumes published between 1892 and 1899, announced modern research on dynamical systems and chaos. In mathematical physics, he founded the properties of the Poincaré-Lorenz group, who were a few months later lead to the fundamental article of Einstein's relativity.

\parpic[l][t]{%
  \begin{minipage}{40mm}
    \fbox{\includegraphics[width=110px,height=140px]{img/medaillons/poisson.eps}}
  \end{minipage}
}
\textbf{Poisson, Siméon Denis} (1781-1840) was a French mathematician whose works were focused on definite integrals, electromagnetic theory and the calculus of probabilities. His family forced him to study medicine that he abandoned in 1798 to study mathematics at the École Polytechnique, where he was a student of Laplace and Lagrange, who became his friends. He taught at the École Polytechnique from 1802 and in 1808, he was appointed astronomer at Bureau des Longitudes, and at its creation, in 1809, professor at the Faculty of Science. The most important work of Poisson focuses on applications of mathematics to physics and mechanics. His \textit{Traité de mécanique} was a mechanical reference for many years. A memoir, published in 1812, contains the most usual laws of electrostatics and the theory that electricity consists of two fluids with similar elements that repel, while different elements attract. In pure mathematics, he published a series of articles on definite integrals, and his research on the Fourier series announced those of Dirichlet and Riemann on this topic. It is in the book \textit{Recherches sur la probabilité des jugements...} (1837), which is an important book on the calculus of probabilities, that for the first time appears the Poisson distribution (or "Poisson law"). Initially obtained as an approximation to the binomial law of Bernoulli it will became fundamental in many problems. The other publications of Poisson include the \textit{Théorie mathématique de la chaleur} (1831) and the\textit{ Théorie mathématique de la chaleur} (1835). The name of Poisson is attached to many mathematical and physical concepts (Poisson integral and equation in potential theory, Poisson brackets in the theory of differential equations, Poisson's ratio in elasticity and Poisson's constant in electricity).

\parpic[l][t]{%
  \begin{minipage}{40mm}
    \fbox{\includegraphics[width=110px,height=140px]{img/medaillons/poynting.eps}}
  \end{minipage}
}
\textbf{Poynting, John Henry} (1852-1912) was a physicist born in Lancashire and died in Birmingham who has worked, among others, on electromagnetic waves. He defined what is named the Poynting vector that represents the power per unit area that carries an electromagnetic wave and the direction of the energy flow. Poynting followed elementary school in a school run by his father. From 1867 to 1872 he attended  Owen College (now Manchester University) where he had as a professor Osborne Reynolds. From 1872 to 1875 he was student at the University of Cambridge where he obtained the honours in mathematics. In the late 1870s he worked at the Cavendish Laboratory under the direction of James Clerk Maxwell. In 1903 he was the first to realize that solar radiation could attract small particles towards the Sun, effect later recognized as the Poynting-Robertson effect. During the year 1884, he analysed the prices of commodity exchanges, including wheat, silk, and cotton, using statistical methods. He was professor of physics at Mason Science College (which later became the University of Birmingham) until his death.

\phantomsection
\addcontentsline{toc}{section}{R}
\label{sec:R}

\parpic[l][t]{%
  \begin{minipage}{40mm}
    \fbox{\includegraphics[width=110px,height=140px]{img/medaillons/ramanujan.eps}}
  \end{minipage}
}
\textbf{Ramanujan, Srivanasa} (1887-1920) was born in Erode, a small village located 400 km south of Madras (India) in a poor family of the Brahmin caste. He spent his childhood in the town of Kumbakonam, where his father worked as an accountant by a draper. From the age of five, he attended different elementary schools before integrating the Town High School in 1898. In 1900, he began to develop his own mathematics based on his first book of mathematics, \textit{The plane Trigonometry}. He defines alone methods to solve the equations of the 3rd and 4th degree, then he also tries to solve those of the 5th degree, unaware that they can not be solved by radicals. We are then in 1902, it was at this time that Ramanujan buys his second (and last!) book that will draw his mathematical working methods, \textit{Synopsis of elementary results in pure mathematics}, compilation of about 6,000 theorems and other formulas by G.S. Carr. This book is essentially a book of results, mostly without proofs, that will influence the future style of Ramanujan, who also left very few details of his own mathematical proofs. At the age of seventeen his approach is already that of a researcher in mathematics. As his results are good, he received a scholarship enabling him to enter the Government College in Kumbakonam in 1904. However, he spends too much time on his research in mathematics and neglects other materials, which earned him the cancellation of the scholarship the following year. Without money, he goes away, without his parents authorization, to Vizagapatnam City where he continues his work on hyper-geometric series and relations between integrals and series. In 1906, he returned to High School again, at Madras this time, with the idea to pass an exam to enter the university. He attends classes a few months and then get sick. During the examination, he succeeded only in mathematics and fails everywhere else, which forbade him the entrance to the University of Madras. In the years that followed, he then goes on to develop his ideas alone, without any outside help and without knowledge of possible research topics, apart from those arising from the concepts presented in the Carr's book. Ramanujan also studied continued fractions and divergent series in 1908. He then falls very ill again and had to undergo, in 1909, an operation which it will be difficult to recover. He began to study and solve mathematical problems in the Journal of the Indian Society of Mathematics (SIM). In 1910, he developed relationships on modular elliptic equations. One year later, the publication of a brilliant article on Bernoulli numbers in the same newspaper earned him the recognition of his work by his peers. Although he has no university degree, he acquired the reputation of a mathematical genius in the area of Madras. The same year, he met the founder of the SIM, which allows him to get a temporary job as accountant in Madras and advises him to contact Ramachandra Rao, a donator member of the SIM. Thanks to this letter, Ramanujan gets the job and starts his work in 1912. He was then fortunate to be surrounded by people with a background in mathematics and interested by his works. The Chief Accountant of the Madras Port is a mathematician who published an article on the work of Ramanujan in 1913, \textit{On the distribution of primes}. On the other hand, a professor of the Madras Engineering College is interested in Ramanujan's abilities. Having himself studied in London, he wrote to one of his mathematics teachers, to whom he sends some results of Ramanujan. The University of Madras allocate later Ramanujan a scholarship in 1913. In 1914 Hardy brought Ramanujan at the Trinity College in Cambridge. This is the beginning of an extraordinary collaboration between the two men. In 1916, he obtained the title of Doctor of the University of Cambridge, even if he does not have the qualifications required to prepare a thesis. In 1918 Ramanujan was elected as a member of the Cambridge Philosophical Society. Three days later, probably the greatest honour of his career, his name appears on the election's list of members of the "Royal Society of London". He was proposed by an impressive list of well-known mathematicians. His election held on 1918 and he was also elected as a member of the Trinity College for six years. Ramanujan go back to India in 1919. However his health continues to deteriorate. He died the following year probably due to severe nutritional deficiencies. Ramanujan left behind a large number of unpublished notebooks (the famous Ramanujan Notebooks), filled with theorems that mathematicians are still studying. Today, his work has still applications in theoretical physics.

\parpic[l][t]{%
  \begin{minipage}{40mm}
    \fbox{\includegraphics[width=110px,height=140px]{img/medaillons/riccicurbastro.eps}}
  \end{minipage}
}
\textbf{Ricci-Curbastro, Gregorio} (1853-1925) born in Lugo and died in Boulogne was a mathematician specialised in differential geometry and one of the fathers of tensor calculus. After studying philosophy and mathematics, Ricci defended his doctoral thesis at the University of Pisa. In 1880, he was appointed professor of mathematical physics at the University of Padua. Levi-Civita was his student and helped to the development of Ricci's absolute differential calculus (1900) to explain mechanics, in abstract spaces (differentiable manifolds), relationships independent of the coordinate system used, inherent to studied the phenomenon (differential invariants). Associated with the differential geometry of Gauss and Riemann, the famous physicist Albert Einstein found in this new mechanics approach named "tensor calculus" (1916), the mathematical tools necessary for his theory of General Relativity.

\parpic[l][t]{%
  \begin{minipage}{40mm}
    \fbox{\includegraphics[width=110px,height=140px]{img/medaillons/riemann.eps}}
  \end{minipage}
}
\textbf{Riemann, Georg Friedrich Bernhard} (1826-1866) was a German mathematician. In high school, Riemann studied the Bible intensively, but he is distracted by mathematics. He even tries to prove mathematically the correctness of the Genesis. His teachers were amazed by his ability to solve complex problems in mathematics. In 1846, with the money from his family, he began studying philosophy and theology to become a priest in order to finance his family. In 1847, his father allows him to study mathematics. He first studied at the University of Göttingen where he met Carl Friedrich Gauss, then at the University of Berlin, where he had as teachers: Jacobi, Dirichlet and Steiner. In his thesis, presented in 1851 under the direction of Gauss, Riemann developed the theory of functions of a complex variable. In 1854 he gave a presentation which lays the foundations of differential geometry. He introduced the right way to extend to $n$-dimensional surfaces the results of Gauss himself. This presentation has changed the conception of geometry, opening the door to non-Euclidean geometry and to the theory of General Relativity. We also own him extensive works on integrals, following those of Cauchy, who gave in particular what we now name "Riemann integrals". Interested in gas dynamics, he lays the foundation for the analysis of partial differential equations of hyperbolic type. He will succeed to Dirichlet for the chair of Gauss in 1859. At 39, he died of tuberculosis.

\phantomsection
\addcontentsline{toc}{section}{S}
\label{sec:S}

\parpic[l][t]{%
  \begin{minipage}{40mm}
    \fbox{\includegraphics[width=110px,height=140px]{img/medaillons/salam.eps}}
  \end{minipage}
}
\textbf{Salam, Abdus} (1926-1996) was a Pakistani physicist who won the Nobel Prize for Physics in 1979 for his works on electroweak interactions and his synthesis of electromagnetism and the weak interactions. Born in Jhang Sadar, he studied at the Government College in Lahore. At the age of fourteen, Salam received the best results ever recorded for the entrance examination at the University of Punjab. Persecuted by the Muslim majority of his country because of his religious affiliation (ahmadiste), he must quit his country. He refugees in Britain, where he obtained in 1952 a PhD in mathematics and physics from the University of Cambridge. His doctoral thesis was a fundamental study on quantum electrodynamics. His work made him famous internationally. He returned to the Lahore Government College as a professor of mathematics, kept this place from 1951 to 1954 and then returned to Cambridge as a lecturer in Mathematics. He teaches in these schools, and in 1957 was appointed professor of theoretical physics at London's Imperial College. He remained there until his retirement. In 1959, he became the youngest member of the Royal Society at the age of thirty-three years. During the 1960s, Salam played an important role in establishing the nuclear research agency of Pakistan and the space research agency of Pakistan, where he was the founding Director. In 1964, he became director of the newly created International Center for Theoretical Physics in Trieste. That same year, he was awarded the Hughes Medal. In 1967, with the physicist Steven Weinberg, Salam proposed a theory to unify electromagnetism and weak interactions between elementary particles, theory that will be confirmed by experience. Salam will be the first Muslim to win the Nobel Prize for Physics in 1979, together with physicists Sheldon Lee Glashow and Weinberg.

\parpic[l][t]{%
  \begin{minipage}{40mm}
    \fbox{\includegraphics[width=110px,height=140px]{img/medaillons/samuelson.eps}}
  \end{minipage}
}
\textbf{Samuelson, Paul} (1915-2009) was an American economist, Nobel Prize in Economics in 1970 and leader of the school that he named "neoclassical synthesis", which meant endorse both Keynesian macroeconomics and neoclassical microeconomics lessons. Samuelson is considered one of the father of the current mainstream microeconomics and possibly one of the pioneer economists to generalize in an economic framework, the use of mathematical models developed for thermodynamic analysis. Thus, he would have helped to put the economic discipline primarily literary in some field of mathematics highly formalized and axiomatized.

\parpic[l][t]{%
  \begin{minipage}{40mm}
    \fbox{\includegraphics[width=110px,height=140px]{img/medaillons/savart.eps}}
  \end{minipage}
}
\textbf{Savart, Felix} (1791-1841) was a surgeon and physicist, born in Ardennes and died in Paris. Inventor of the sonometer and also of a gear that bears his name and the polariscope. He laid the foundations of molecular physics. With the physicist Jean-Baptiste Biot, he measured the magnetic field created by a current and formulated the Biot-Savart law. He also studied the properties of vibrating strings. He was a member of the Académie des Sciences, elected in 1827, and Chair of General and Experimental Physics of the Collège de France, appointed in 1836, succeeding André-Marie Ampère. He was elected as foreign member of the Royal Society in 1839. His name was given to a unit of measurement of musical intervals.

\parpic[l][t]{%
  \begin{minipage}{40mm}
    \fbox{\includegraphics[width=110px,height=140px]{img/medaillons/say.eps}}
  \end{minipage}
}
\textbf{Say, Jean-Baptiste} (1767-1832) was an economist, journalist and French industrialist born in Lyon and died in Paris. He comes from a family of merchants who emigrated to Amsterdam (Netherlands) and Geneva (Switzerland). It was during a trip to Britain, where the industrial revolution was underway, that he will adopt liberal ideas and especially the theories of Adam Smith, for which he will be a strong advocate when returning to France. In 1789, he published the brochure: \textit{Liberté de la presse}. In 1792, he participated in military campaigns of the French Revolution in Champagne. Initially working in a bank, he managed after a cotton mill at Auchy-lès-Hesdin at the Pas-de-Calais. His many books on political economy made that he was appointed professor at the Conservatoire National des Arts et Métiers in 1821, then at the Collège de France in 1830. The "Say's law" or "law of markets", states that more the producers are numerous and the productions multiple, more the opportunities are easy, varied and vast. In an economy where competition is free and perfect, crises of overproduction are impossible (...). There can't be an imbalance in global market economies and free enterprise (...), there is a spontaneous balancing economic flows (production, consumption, savings, investment). This law is sometimes wrongly reduced to the formula: any supply creates its own demand. The best summary of this approach would be: we spend only the money that we won. The supply-side economics, in the tradition of Say, opposes economic demand, which is that of Malthus and later Keynes.

\parpic[l][t]{%
  \begin{minipage}{40mm}
    \fbox{\includegraphics[width=110px,height=140px]{img/medaillons/schaefer.eps}}
  \end{minipage}
}
\textbf{Schaefer, Milner Baily} (1912-1970) was born in Wyoming and died in San Diego. Schaefer studied at the University of Washington where he received a Bachelor of Science in 1935. After his bachelor, he worked in the fisheries department of the state of Washington in Seattle. From 1937 to 1942 he worked at the Comission of the salmon fisheries of the Pacific Westminster, British-Columbia. He served in the Navy during the war and thereafter, he held various positions as a fisheries biologist. After completing his PhD at the University of Washington in 1950, Schaefer became Director of Investigations of the IATTC (Inter-American Tropical Tuna Commission), an international commission of fisheries. During the 10 years that followed, he worked on the theory of the dynamics of the fishery and developed a population model of marine species known under the name "Schaefer model". During the 1950s, Schaefer became increasingly involved in several committees, groups and organizations concerned with marine resources, particularly fishing and all aspects of oceanography. During this period, he lectured on the dynamics and exploitation of fish populations. In 1962, he resigned from his position as director of investigations at the IATTC for the position of Director of the Institute of Marine Resources of the University of California while serving as a scientific advisor to the IATTC.

\parpic[l][t]{%
  \begin{minipage}{40mm}
    \fbox{\includegraphics[width=110px,height=140px]{img/medaillons/scholes.eps}}
  \end{minipage}
}
\textbf{Scholes, Myron} (1941-) was born in Ontaria, he presented his PhD in 1969 at the University of Chicago. In 1988 he held the Frank E. Buck chair as Professor of Finance at the Graduate School of Business at Stanford University (California) where he also directs research for the Hoover Institution. He received the Nobel Prize in Economics in 1997 with Fischer Black, for the development of an evaluation method of financial derivative instruments (innovative mathematical results to estimate the risks associated with options) that have opened new horizons in the field of economic evaluations. The co-winner of Myron Scholes, Robert Merton, played a very important role in the development of this method of evaluation as well on the applications it has allowed to improve the management of risks related to new financial products. Already in 1900, Louis Bachelier, presented at the Sorbonne a visionary doctoral thesis: \textit{Théorie de la spéculation}. In the 1960s, authors like James Boness and Paul Samuelson (Nobel Prize in Economics in 1970) proposed models to determine the equilibrium price of options. Their assumptions have not proved to be sufficiently realistic for real applications, but improvements to these models in the early 1970s have yielded more satisfactory results. It was in 1973 that Black and Scholes put their skills together and propose the first version of the formula for option pricing which earned them the Nobel Prize. If Myron Scholes and Fischer Black had the fundamental intuition of the demonstration, they took for basis the research base equilibrium model of financial assets (or Capital Asset Pricing Model: CAPM) of their compatriot William Sharpe rewarded for this by the Nobel jury in 1990 (the other two winners were Harry Markowitz and Merton Miller).

\parpic[l][t]{%
  \begin{minipage}{40mm}
    \fbox{\includegraphics[width=110px,height=140px]{img/medaillons/schrodinger.eps}}
  \end{minipage}
}
\textbf{Schrödinger, Erwin} (1887-1961) was born and died in Vienna (Austria). He entered the Gymnasium of that city in 1898. Almost from the first day of class until he left school eight years later, Schrödinger was an excellent student. He was always first in his class thanks to his hard work between the four walls of his personal office. He continued his studies at the University of Jena. In 1920 he was appointed professor at the Stuttgart Technical High School (Germany) and the next year at the University of Breslau. In 1927, he succeeded Max Planck at the University of Berlin. Israelite, he left the country with the advent of National Socialism to go to Oxford where he obtained a professorship in 1933. Seven years later, he became professor of theoretical physics at the Dublin Institute for Advanced Studies of the Irish free State. He will return to Austria only in 1956. Schrödinger, like his contemporary Albert Einstein ,was horrified to learn by heart and be forced to memorize unnecessary facts. Schrödinger's early work focused on the study of color and quantum theory. But he is primarily known for his research in wave mechanics, discipline developed by the French Louis de Broglie. We own him the Schrödinger equation developed in 1926 to calculate the wave function of a particle moving in a field. By establishing this propagation equation, he gives an intuitive tool to quantum mechanics indispensable today (unlike the abstract Heisenberg matrix approach) that Einstein qualified as a Genius Idea. With that of Werner Heisenberg, Schrödinger's theory forms the basis of quantum mechanics. In 1933, Schrödinger shared the Nobel Prize in Physics with Paul Dirac for their contribution to the development of this new discipline. Schrödinger also attempt to apply his theory to biology and genetics in his books \textit{What is life} (1944) and \textit{Science and Humanism} (1951).

\parpic[l][t]{%
  \begin{minipage}{40mm}
    \fbox{\includegraphics[width=110px,height=140px]{img/medaillons/schwartz.eps}}
  \end{minipage}
}
\textbf{Schwartz, Lawrence} (1915-2002) was a French mathematician born and died in Paris. His work is mainly related to Analysis. Old student of the École Normale Supérieure, Laurent Schwartz taught from 1959 to 1960 and from 1963 to 1983 at the École Polytechnique. In 1975 he was elected member of the Académie Des Sciences. His thesis (1943) focuses on the study of approximation and sums of exponentials. Thanks to the theory of distributions, whose original idea came in 1945, he won the Fields Medal in 1950. The language and notation of Schwartz distributions have been naturally adopted by almost all mathematicians and are the natural framework of the theory of partial differential equations. From 1959 to 1962, Schwartz dedicated his time to theoretical physics: the use of distributions allows him to find a correct mathematical formulation for the theory of elementary particles. He has also conducted research on Radon measures on arbitrary topological spaces and has written various publications on cylindrical probabilities and disintegrations of measures.

\parpic[l][t]{%
  \begin{minipage}{40mm}
    \fbox{\includegraphics[width=110px,height=140px]{img/medaillons/schwarzschild.eps}}
  \end{minipage}
}
\textbf{Schwarzschild, Karl} (1873-1916) was a mathematician, astronomer and physicist born in Frankfurt and died in Potsdam who predicted the existence of black holes. His curiosity for the stars appeared from his early school years, when he built a small telescope. Because of to this interest, his father introduced him to a friend mathematician who had a private observatory. Schwarszchild learned to use a telescope and studied alone more advanced mathematics than at school. He became famous with his first two papers on the theory of orbits published at the age of sixteen when he was still in college. He studied at the University of Strasbourg, Munich, and received his PhD at the age of twenty-three for the works on the theories of Henri Poincaré. He was then hired as an assistant at the Kuffner Observatory in Ottakring. He devoted himself mainly to photometry: he performed pioneering works to improve photographic plates and implement their use in astronomy, and in the spectral study of the stars. From 1901 to 1909 he officiated as a professor at the prestigious Institute of Göttingen, where he had the opportunity to work with celebrities such as David Hilbert and Hermann Minkowski. He then held a position at the Astrophysical Observatory of Potsdam in 1909. Schwarzschild is best known for his contributions to theoretical physics, among in the Sun physics as in General Relativity, or stellar kinematics, as well as in various fields of astrophysics. In 1916, he founded a quantity named the "Schwarzschild radius" in the framework of the theory of relativity, stated shortly before by Albert Einstein. When a sufficiently massive star explodes in a supernova, the gravitational contraction produced what is called a "black hole": almost nothing, not even light, can escape this intense gravitational field. When the radius of a gaseous mass falls below the "Schwarzschild radius" for this mass, it collapses into a black hole.

\parpic[l][t]{%
  \begin{minipage}{40mm}
    \fbox{\includegraphics[width=110px,height=140px]{img/medaillons/shannon.eps}}
  \end{minipage}
}
\textbf{Shannon, Claude Elwood} (1916-2001) was born in Massachusetts and died in Migichan. He was a mathematician specializing in Applied Mathematics and electrical engineer, who developed the theory of communication, now known as the "Information Theory". Shannon took courses at the University of Michigan in 1940 and obtained his PhD (he wrote his thesis demonstrating that electrical applications of Boolean algebra could construct any logical, numerical relationship) from the Massachusetts Institute of Technology (MIT), Faculty of which he became a member in 1956, after working in the Bell Telephone laboratories. In 1949, Shannon published the \textit{A Mathematical Theory of Communication}, an article in which he presented his initial concept for a unification theory of the transmission and processing of information. Shannon contributed to the field of cryptanalysis for national defense during World War II, including his basic work on codebreaking and secure telecommunications. Information, according to this theory include all types of messages, including those sent along the nerve channels of living organisms. The information theory is now important in many areas.

\parpic[l][t]{%
  \begin{minipage}{40mm}
    \fbox{\includegraphics[width=110px,height=140px]{img/medaillons/sharpe.eps}}
  \end{minipage}
}
\textbf{Sharpe, William Forsyth} (1934-) is an economist born in Boston. The Sweden Royal Academy of Sciences has awarded in 1990 the Nobel Prize in economics to three American professors: Harry Markowitz, Merton Miller and William Sharpe. Even if the rewarded works were already old and are situated mostly between 1950 and 1970, the Academy decided that the winners were innovators in the field of the theory of financial economics and corporate finance. Indeed, they all contributed to emerge from the shadow of some American universities, a new discipline: quantitative finance. It was the first time that the Royal Swedish Academy rewarded work dealing with stock markets and portfolio management rather than economic equilibrium. William Sharpe, of Stanford University, was rewarded for his equilibrium model of financial assets and for his work on the theory of price formation for financial assets. He was also engaged in his research on the path opened by Harry Markowitz. This last had indeed developed a complicated procedure for selecting stocks to optimize an investment portfolio. But the implementation of his model was quickly raised by practical problems, at the point that the collection of information and treatment became almost impossible with the computers of 1960s. This is why William Sharpe began searching for an easier method of selecting efficient portfolios. He discovered that the variations in the profitability of each title are linked linearly to changes in the overall market, as measured by the concerned index market (e.g. Standard \& Poor 500 index in the United States, or CAC 40 in France). The number of necessary statistics was greatly reduced: 302 statistics instead of 3,150 in the Markowitz model for 100 titles, 602 instead of 20,300 for 200, 10,002 instead of 125,750 for 300 titles, the calculation was immediately easier. It is from this concept, simple in appearance, that Sharpe discovered his famous Beta $\beta$ coefficient linking the profitability of a security to the market index and also being a measure of the risk associated with market volatility. Beyond their practical contribution, the works of Sharpe have contributed decisively to the development of a pricing theory for financial assets more known as a "Capital Asset Pricing Model" (CAPM).

\parpic[l][t]{%
  \begin{minipage}{40mm}
    \fbox{\includegraphics[width=110px,height=140px]{img/medaillons/smith.eps}}
  \end{minipage}
}
\textbf{Smith, Adam} (1723-1790) was a scottish philosopher and economist, born in Kirkcaldy and died in Edinburgh (Scotland). He studied at the universities of Glasgow and Oxford. From 1748 to 1751, he taught rhetoric and literature in Edinburgh. During this time, he meets the philosopher David Hume, whose ideas had a great influence on the conceptions of Smith on ethics and economics. Smith was appointed professor of logic in 1751 and professor of moral philosophy in 1752 at the University of Glasgow. Later, he gathered the ethics courses that he conducted and published in his first masterpiece entitled \textit{Theory of Moral Sentiments}, in 1759. In 1763, he resigned his professorship to accompany the Duke of Buccleuch in a journey of eighteen months in France and Switzerland, as a tutor. From 1766 to 1776 he lived in Kirkcaldy where he worked on his main book: \textit{The Wealth of Nations}. Smith was later appointed commissioner of customs in Edinburgh in 1778, a position he held until his death. In 1787 he was also appointed rector of the University of Glasgow. His famous treatise\textit{ An Inquiry into the Nature and Causes of the Wealth of Nations} (1776), the first study attempting to describe the nature of capital and the historical development of industry and trade between European countries, caused him to be considered as the father of modern economics. \textit{The Wealth of Nations} is the first essay on the history of economic science which considers political economy as an autonomous discipline, distinct from political science, ethics and jurisprudence. Smith proposes a process analysis of production and distribution of wealth, and shows that the main source of any income, that is to say the basic forms in which wealth is distributed, are rents, wages and profits. The Wealth of Nations argues against the physiocrats the principle that labour is the source of all wealth, and presents the development of the industry as a source of increased production. For Smith, the theorist of liberal capitalism, the economics and moral comes from competition, production and trade of goods can only be stimulated, and consequently the general standard of living improved, when governments regulate and control a minimum industrial and commercial activities. To describe this situation, he speaks of a natural order set by the "invisible hand", which may naturally converge the sum of individual interests to the general interest. As a result, too much government intervention in the context of free competition could only be bad.

\parpic[l][t]{%
  \begin{minipage}{40mm}
    \fbox{\includegraphics[width=110px,height=140px]{img/medaillons/sommerfeld.eps}}
  \end{minipage}
}
\textbf{Sommerfeld, Arnold} (1868-1951) was a German physicist born in Königsberg, and died Münch. He studied mathematics and natural sciences at the Königsberg Universität where he received his PhD in 1891. He successively held the chairs of mathematics in Clausthal (1897), Applied Mathematics at Aix-la-Chapelle in France (1900) and theoretical physics in Munich (1906-1931). In 1897, he began with C. F. Klein, a treaty in four volumes of the gyroscope, that he needed thirteen years to complete and at the same time he also did research in other areas of applied physics and engineering, such as friction, lubrication and radio. We own him the improvement of Bohr's model (1916) introducing elliptical orbits and relativistic corrections. This new model, which implies a dependence of energy vis-à-vis the second quantum number, can explain the fine structure of spectral lines emitted by atoms. Sommerfeld also introduced the famous "fine structure constant". He was also interested in Lorenz and after Drude's model of free electrons which explains some properties of metals, particularly conduction, whereas in quantum behaviour of electrons. He participated to the development of band theory in solid state physics, presenting in 1928 the idea that electrons occupy quantified states in the material.

\parpic[l][t]{%
  \begin{minipage}{40mm}
    \fbox{\includegraphics[width=110px,height=140px]{img/medaillons/stokes.eps}}
  \end{minipage}
}
\textbf{Stokes, George Gabriel} (1819-1903) was a mathematician and physicist, born in Ireland and died in Cambridge. In 1841, he graduated with honours from the University of Cambridge and began a career as a researcher. Influenced by his former teacher, he devoted himself to the study of viscous fluids. He published in 1845 the results of his works on the movement of fluids in his thesis \textit{On the theories of internal friction of the fluids in motion}. His mathematical approach describing the flow of an incompressible Newtonian fluid in a three-dimensional space, adding a viscous force from the Euler equations (\textit{General principles of fluid motion}, 1755), is the origin of the Navier-Stokes equations. All his researches are synthesized by his treatise \textit{Report on recent research in Hydrodynamics}, published in 1846, the founding text of hydrodynamics. In 1849 he became a professor at the chair of mathematics at the same university. Elected in 1851 at the Royal Society, he will be the president from 1885 to 1890. The last three mentioned positions were occupied by Isaac Newton. He received the Smith Prize in 1841, the Rumford Medal in 1852 and the Copley Medal in 1893.

\parpic[l][t]{%
  \begin{minipage}{40mm}
    \fbox{\includegraphics[width=110px,height=140px]{img/medaillons/stefan.eps}}
  \end{minipage}
}
\textbf{Stefan, Josef} (1835-1893) was an Austrian physicist born in Sankt Peter near Klagenfurt and who died in Vienna. The research works of Stefan include kinetic theory of gases, especially hydrodynamics and radiation theory. After studying at the Wien Universität where he obtained his doctorate in 1858, appointed Privatdozent of mathematical physics, he became professor of physics in 1863, then director of the Institute of Physics (1866). Member of the Academy of Sciences in Vienna, he was the secretary from 1875. Before the work of Stefan, G. R. Kirchhoff had already described the properties of the "perfectly black-body", that can absorb all incident radiation and emit a broad spectrum of wavelengths. Stefan proofs empirically in 1879 that the intensity of the black-body radiation is proportional to the 4th power of its absolute temperature, relation known since as the "Stefan-Boltzmann law", Boltzmann having also deduced the same results from thermodynamic considerations . This law is one of the important first steps that led to the interpretation of the black body radiation and quantum theory of radiation.

\parpic[l][t]{%
  \begin{minipage}{40mm}
    \fbox{\includegraphics[width=110px,height=140px]{img/medaillons/sturm.eps}}
  \end{minipage}
}
\textbf{Sturm, Charles François} (1803-1855) was born in Geneva (Switzerland) and died in Paris. After studying at the University of Geneva, Sturm went to be tutor in the family of De Broglie in Paris where he attended the greatest scholars of his time and where he settled permanently starting 1825. In 1826 he determines the speed of sound in water, which earned him the following year, the grand prize of mathematics proposed for the best thesis on the compressibility of liquids. In 1829, he stated the famous theorem that bears his name, essential for the study of the properties of the roots of an algebraic equation which specifies the number of real roots of a numerical equation between two limits. He published the proof of this theorem in 1835. In 1830, in conjunction with his friend Liouville, he focused on the problem of the general theory of oscillations and studied differential equations of second order (Sturm-Liouville problems) in several articles, including \textit{Sur les équations différentielles linéaires du second ordre} (1836) and \textit{Sur une classe d'équations à différences partielles} (1836). The methods used will be at the origin of a lot of mathematical works and discoveries. He was elected in 1836 to the Académie des Sciences and work at the École Polytechnique. Succeeding to Poisson, he taught, from 1840, at the Faculté de Paris (mechanical chair). His C\textit{Cours d'analyse de l'École Polytechnique} (1857-1863) and his \textit{Cours de mécanique de l'École Polytechnique} (1861) will be published after his death in Paris.

\phantomsection
\addcontentsline{toc}{section}{T}
\label{sec:T}

\parpic[l][t]{%
  \begin{minipage}{40mm}
    \fbox{\includegraphics[width=110px,height=140px]{img/medaillons/taylor.eps}}
  \end{minipage}
}
\textbf{Taylor, Brook} (1685-1731) was an English mathematician born in Edmonton and died in London. He is famous for his contributions to the development of infinitesimal calculus. Taylor was educated at Saint John College, Cambridge. He obtained in 1708 a remarkable solution to the problem of the center of oscillation, which however remained unpublished until 1714 when his priority right was disputed by John Bernoulli. Taylor's book, \textit{Methodus incrementorum directa} (1715), added to higher mathematics a new chapter, named nowadays the "calculus of finite differences". Among other ingenious applications, he used it to determine the movement pattern of a vibrating string with success by reducing the problem to the principles of mechanics. The same book contains the famous formula known as "Taylor's theorem", who's importance appeared only in 1772, when Louis Lagrange realized its power and made it the fundamental principle of differential calculus. In his essay \textit{Linear Perspective}, Taylor sets out the principles of art in an original and more general form than any of his predecessors, but the work suffered from the confusion and lack of clarity that affected most of his writings. Taylor was elected to the Royal Society in 1712. He sat in the same year at the committee to settle priorities disputes between Newton and Leibniz and was secretary of the society from 1714 to 1718. From 1715, his research took a philosophical and religious orientation.

\parpic[l][t]{%
  \begin{minipage}{40mm}
    \fbox{\includegraphics[width=110px,height=140px]{img/medaillons/teller.eps}}
  \end{minipage}
}
\textbf{Teller, Edward} (1908-2003) was a nuclear physicist born in Budapest and died at Stanford. He left Budapest in 1926 to go to Karlsruhe (Germany) to study chemistry, but soon he will develop an affinity with the new theory of quantum physics which led him to study at the Leipzig Universität where he obtain his PhD at the age of twenty-two. Teller won this title under the direction of Werner Heisenberg who participated actively in the later German nationalists camp during World War II. In 1935, Teller expat to the United States and its expertise in advanced physics led him to make a lot of relationships and a good reputation in the scientific community. He was named professor in many American universities and worked on the Manhattan Project in 1942 where he led the very important work that helped to create the first nuclear fission bomb. The work done, Teller argued for the continuation of work looking for a thermonuclear bomb by fear of the Russian advance in this field (Teller was anticommunist and very good friend of Landau who was arrested by the communist police). Teller persuaded the U.S. government to finance research for a hydrogen bomb and led the successful works which make him considered today as the father of the H-bomb.

\parpic[l][t]{%
  \begin{minipage}{40mm}
    \fbox{\includegraphics[width=110px,height=140px]{img/medaillons/tesla.eps}}
  \end{minipage}
}
\textbf{Tesla, Nikola} (1856-1943) was a genius Serbian engineer and inventor in the field of electricity who died in New York. He is often considered as one of the greatest scientists in the history of technology, having for over 300 patents (which are mostly affected to Thomas Edison) dealing with new methods to address the conversion of energy. In 1875, he entered the Polytechnic in Graz (Austria), where he studied mathematics, physics and mechanics. A scholarship given by the administration of the Military Frontier (Vojna Krajina) avoiding him money problems. This did not however prevent him to work hard to assimilate the program for the first two years of study in one year. The following year, the removal of the Military Frontier removes any financial assistance to Tesla, apart from that, very small, that can bring his father, which does not allow him to complete his second year of study. Tesla gained experience in telephony and electrical engineering before immigrating to the United States in 1884 to work for Thomas Edison in New York City. We owe him contributions to the design of the modern alternating current (AC) electricity supply system, the asynchronous electric motor, polyphase alternator, mounting three-phase star, the rotary converter. Tesla discovered the principle of wave reflection on objects in 1900, he studied and published, despite financial problems, the foundations of what would become almost three decades later the radar.

\parpic[l][t]{%
  \begin{minipage}{40mm}
    \fbox{\includegraphics[width=110px,height=140px]{img/medaillons/thom.eps}}
  \end{minipage}
}
\textbf{Thom, René} (1923-2002) was a French mathematician author of important works in differential topology. Born in Montbéliard and died in Bures-sur-Yvette, Thom was a student at the École Normale Supérieure. In 1958, he received the Fields Medal for his theory of cobordism (equivalence relation between compact differential manifolds). In a communication at the conference of Strasbourg (1951), Thom establishes that if the zeros of a polynomial ideal form a variety, it is a border variety, and his thesis, \textit{Espaces fibrés en sphères et carrés de Steenrod} (1951), already contains the germ of the main cobordistes methods. It is in the last chapter of a dissertation of 1954 (\textit{Quelques Propriétés globales des variétés différentiables}) that the theory of cobordism is exposed for the first time. After 1955, Thom has studied especially laminated spaces and stratified sets and morphisms. We owe him results on the approximations of differentiable transformations and their singularities, comparisons of differentiable structures on a triangulated manifold and a Morse theory for laminated varieties. He is also one of the first to use techniques of "surgery" varieties. Since 1969, Thom is devoted to the applications of topology to the phenomena of life. To describe the birth and evolution of forms, he has developed a specific mathematics: his catastrophe theory is a theory of singularities of certain differential equations. Specifically, it allows, from observed phenomena, to trace their unknown causes, at least partially. Thom gave a presentation of his work in the book \textit{Stabilité structurelle et morphogenèse} (1973).

\parpic[l][t]{%
  \begin{minipage}{40mm}
    \fbox{\includegraphics[width=110px,height=140px]{img/medaillons/thales.eps}}
  \end{minipage}
}
\textbf{Thales of Miletus } ($\sim$624 BC. - $\sim$524 BC.) Is one of the first mathematician whose history has retained the name. He was born in Miletus in Minor Asia on the Mediterranean coast of modern Turkey. More than just a mathematician Thales was a universal scholar, curious about everything, astronomer and philosopher, very observant. We did not prove what we say at the time of Thales, we only noticed properties. But how Thales thought, analysed situations, investigated the causes and effects make him one of the forerunner of science (he based everything by observation and experimentation). One of the big questions for Thales was water, and the causes of the rain. He had noticed that the air turned into rain, and he searched desperately answers. Thales has formulated several geometric properties that he learned perhaps from the Egyptians, instead some elements of this properties were already known long age, he laid the foundations of reasoning with ideal figures through which he obtained several results known today as "Thales' theorem". But the must know fact of Thales is undoubtedly the prediction of a solar eclipse, probably that of 8 May 585 BC. We also owe him the first discovery of electricity through two experiments. First he noticed that amber had the property of attracting light materials . Another experiment realized in Magnesia..., in -600, allows him to highlight the properties of magnetization of iron oxide.

\parpic[l][t]{%
  \begin{minipage}{40mm}
    \fbox{\includegraphics[width=110px,height=140px]{img/medaillons/tukey.jpg}}
  \end{minipage}
}
\textbf{Tukey, John Wilder} (1915-2000) was an American mathematician best known for development of the FFT (Fast Fourier Transform) algorithm and box plot. The Tukey range test, the Tukey lambda distribution, the Tukey test of additivity, and the Teichmüller–Tukey lemma all bear his name. He is also credited with coining the term "bit". Tukey was born in New Bedford, Massachusetts in 1915, and obtained a B.A. in 1936 and M.Sc. in 1937, in chemistry, from Brown University, before moving to Princeton University where he received a PhD in mathematics. During World War II, Tukey worked at the Fire Control Research Office and collaborated with Samuel Wilks and William Cochran. After the war, he returned to Princeton, dividing his time between the university and AT\&T Bell Laboratories. He became a full professor at 35 and founding chairman of the Princeton statistics department in 1965. He was awarded the National Medal of Science by President Nixon in 1973.[6] He was awarded the IEEE Medal of Honor in 1982 "For his contributions to the spectral analysis of random processes and the fast Fourier transform (FFT) algorithm." Tukey retired in 1985. He died in New Brunswick, New Jersey on July 26, 2000.

\parpic[l][t]{%
  \begin{minipage}{40mm}
    \fbox{\includegraphics[width=110px,height=140px]{img/medaillons/turing.eps}}
  \end{minipage}
}
\textbf{Turing, Alan} (1912-1954) In his theoretical work in the fields of logic and probabilities, Turing is considered, if not the founder of computers, in any case, as one of the spiritual fathers of artificial intelligence. Born in Paddington (United-Kingdom) Turing does a normal education despite a brilliant mind and net predispositions for sciences. In 1926 he went at Sherborne School. From 1931 to 1934, Turing studied Mathematics at the King's College of the University of Cambridge. During this period, he discovered the work of John von Neumann on quantum mechanics. Stimulated by his researches, he began the study of problems of probability and logic. After his graduation, he learned in the summer of 1936 the developments of Max Newman on a mathematical theory of Gödel incompleteness and the question of Hilbert's decidability. If for many proposals, it is easy to find an algorithm, what about those for which the algorithm, not rigorous enough, is not enough to validate the proposal? Should we infer that they can not be validated? It is now in this direction that the researches of Turing will focus. In 1936 he was awarded the Smith price for his work on probabilities and the concept of "Turing machine". This concept is the basis of all theories of automata and more generally for the theory of computability. The purpose is to formalize the principle of algorithm, represented by a sequence of instructions - acting in sequence on input data - that might provide a result. This formalization requires Turing to develop the notion of computability and identify a class of "decidable" problems. This led him to introduce a new class of functions: "computable functions in the sense of Turing". During his PhD at Princeton University from 1936 to 1938, Turing conceived the idea of building a computer. Returning to Cambridge, he studied mathematics and focuses on the Riemann zeta function. World War II soon offers him the opportunity to put into practice his theories. It is in the British Communications Department of the Ministry of Foreign Affairs that he is confronted to the Enigma secret code, name of the machine used by the German Navy to communicate with submarines. The encryption used by the Nazis always escaped the traditional methods of investigation. With the collaboration of W. G. Welchman, Turing was able to break the code by applying his new method and. Once war finished, Turing joined the National Physical Laboratory, where he began, in competition with U.S. projects to create the first computer. Technological advances suggest him to achieve this goal in the near future. In 1948, thanks to Newman, he obtained a position as a lecturer in mathematics at the University of Manchester that he held until the end of his life. Two years later, he participated with Frederic Williams and Tom Kilburn at the realization of an electronic computer, the Mark I, and wrote on this occasion a programming manual. At the same time, he publishes \textit{Can a machine think?} in which he summarizes the conceptual and mathematical basis of programmable electronic computer and sums up his philosophy of "intelligent machine". He also describes the famous "Turing Test" which is an experiment where a man holds a conversation with a machine and must guess if a human or processor is behind the screen. Turing was convinced that everything was a only problem of information and that the development of technologies will allow in the next fifty years machines able to defeat the human being at least five minutes. Turing committed suicide by cyanide poisoning because of homophobic persecutions in the United-Kingdom.

\phantomsection
\addcontentsline{toc}{section}{V}
\label{sec:V}

\parpic[l][t]{%
  \begin{minipage}{40mm}
    \fbox{\includegraphics[width=110px,height=140px]{img/medaillons/vanderwaals.eps}}
  \end{minipage}
}
\textbf{Van Der Waals, Johannes Diderik} (1837-1923) was a Dutch physicist born in Leiden (Netherland) and died in Amsterdam. Van Der Waals was first teacher at the age of twenty before becoming, after a lot of solitary efforts, teacher in middle school (1863). He attended classes at the University of Leiden from 1862 to 1865 and teaches as professor of physics at Deventer and The Hague (1866). In 1873, he was received his PhD at the University of Leiden after defending a dissertation entitled: \textit{Over de continuiteit van den gas en vloeistoftoestand} that contains the presentation of the state equation that bears his name and led to much more positive results than the classical equation of ideal gases near the liquefaction zone. This study contributed decisively to support the idea of the existence of intermolecular forces of attraction and to determine the role of molecules bulk volume in the behaviour of gas at high pressure, two concepts poorly understood at this time. The rapid success of this new theory is illustrated by the many translations of the original paper that followed its presentation. It is now known that the van der Waals equation is still imperfect and it would be foolhardy to try to preserve the name of "real gas equation" which was once awarded. Indeed, nowadays state equations even more appropriate can achieve an approximation much more accurate which are generally derived from kinetic considerations based on molecular virial theorem forces. From 1877 to 1907, the date of his retirement, Van der Waals was appointed professor of physics at the University of Amsterdam. It was during this period that he made known his law named "the theorem of corresponding states" (1880). This equation of state for all pure bodies greatly contributed, too, to his reputation because it was later used as a guide for prior tests to the liquefaction of hydrogen and helium. From another point of view, the van der Waals contribution is also considered one of the first attempts to express the laws of physics in terms of reduced variables. Among other works of Van der Waals, we can found a major contribution to the theory of binary mixtures and molecular study of capillarity. He received the Nobel Prize in Physics in 1910 for his work on the state equation of gases and liquids aggregation.

\parpic[l][t]{
  \begin{minipage}{40mm}
    \fbox{\includegraphics[width=110px,height=140px]{img/medaillons/viete.eps}}
  \end{minipage}
}
\textbf{Viète, François} (1540-1603) was born in Fontenay-le-Comte (France) and died in Paris. Viète is known today as the inventor of modern algebra. However, at his time, he was best known as a master of requests and Privay Councillor of Henry IV than as a mathematician. His whole life is marked by the duality of a brilliant political career and a strong work practice on the highest problems of mathematics of his century. His scientific work has suffered from numerous political concerns and the limited time they left him. The fact remains that the contribution of Viète to the development of mathematics in the late 16th century is very important. It is characterized by the systematic introduction of the literal representation in algebraic problems for both the unknown and the known quantities, which presents the main advantage of treating the general case and the special cases and not to focus on the structure of problems rather than their expression. Viète in his youth was a student of the Franciscan, at the Collège de Cordeliers. He continued his studies at the Faculté de Droit de Poitiers and entered in the active life as a lawyer. He was appointed council of the Brittany Parliament in 1573, staying there only a few, quite occupied he is by his mathematical work and confidential missions assigned by the king. We found then his trace in Paris in 1579 where he published the \textit{Canon mathematicus}, accompanied by the \textit{Liber singularis}. Appointed master of requests of the king's household in 1580, he resigned from his position in 1585, as a result of people conflicts. In 1589, he is at Tours and prepares the publication of his scientific work. He is also responsible of statistical cryptography for the King. He returned to Paris with the King and was appointed as Privy Councillor. Viète will die after a long period of decline because of disease.

\phantomsection
\addcontentsline{toc}{section}{W}
\label{sec:W}

\parpic[l][t]{
  \begin{minipage}{40mm}
    \fbox{\includegraphics[width=110px,height=140px]{img/medaillons/walras.eps}}
  \end{minipage}
}
\textbf{Walras, Leon} (1834-1910) was a French economist born in Evreux (France) and died in Clarens (Switzerland). He is the son of Auguste Walras, a French economist whose ideas greatly influence her son in the field of social and financial reform in general. He studied at the Collège de Caen in 1844, and the Collège de Douai in 1850. He graduated bachelor-ès-Letters in 1851 and bachelor-ès-Sciences in 1853. The same year, he is not declared eligible for Polytechnic and also after a second trial. In 1854, he is received as external student at the École des Mines de Paris, but he has no interest in engineering and he left the school. Appointed professor at the University of Lausanne (Switzerland), Walras denounced, from the 1870s, the liberal economic theories taught in universities, that he felt unable to explain the economic problems of his time. In his \textit{Éléments d'économie politique pure} (1874), his critics focus especially on the theories of labour value and rent but through it, it is all the classical heritage that he challenges (including that of Adam Smith). Influenced by the mathematician Antoine Cournot, he is one of the first to introduce systematically mathematics in economics. Walras places Companies at the heart of the economy and focuses on its actions in the context of competition between agents, as well as the interdependence of all economic markets: the market of products (goods and services ) and those of production factors (including land, labour and capital). He wonders how to set prices and quantities simultaneously, and defines the problem of general equilibrium, that is to say, the stability of equilibria in all markets. Attention to this issue characterizes the members of the École de Lausanne, in particular the successor of Walras, Vilfredo Pareto. With the Austrian Carl Menger and the Britain Stanley Jevons, who he did not know when he undertook this path, he is considered as one of the founders of neoclassical marginalism.

\parpic[l][t]{
  \begin{minipage}{40mm}
    \fbox{\includegraphics[width=110px,height=140px]{img/medaillons/weber.eps}}
  \end{minipage}
}
\textbf{Weber, Wilhelm} (1804-1891) was a German physicist born in  Wittenberg  and died in Göttingen specialized in electrodynamics. Weber wrote in 1824 a treatise on the wave motion with his older brother, Ernst Heinrich Weber, well known anatomist, and studied with his brother Eduard Friedrich Weber the walking mechanism (1836). In 1831, on the recommendation of Carl Friedrich Gauss, he was hired by the university of Göttingen as professor of physics, at the age of twenty-seven. As a teacher he did what a lot of students would like: a free of charge access to the college laboratory. At Göttingen he collaborated with Carl Friedrich Gauss on the study of geomagnetism, and he connected their laboratories by an electric telegraph: it was one of the first telegraph transmissions that we know. His greatest achievement was that he brought to Leipzig, with F.W.G. Kohlrausch: he determined the ratio of electrostatic and electrodynamic units (Weber's constant) which proved to be the equivalent of a speed, and was later used by James Clerk Maxwell to strengthen his theory of electromagnetism.

\parpic[l][t]{
  \begin{minipage}{40mm}
    \fbox{\includegraphics[width=110px,height=140px]{img/medaillons/weierstrass.eps}}
  \end{minipage}
}
\textbf{Weierstrass, Karl Theodor Wilhelm} (1815-1897) was a mathematician born in Ostenfelde (Prussia) and died in Berlin, who gave to the theory of functions its modern form by specifying in particular the formalism of limits and is thus considered to be the father of modern analysis. His interest in mathematics began while he was a Gymnasium student at Theodorianum in Paderborn. He was sent to the University of Bonn upon graduation to prepare for a government position. Because his studies were to be in the fields of law, economics, and finance, he was immediately in conflict with his hopes to study mathematics. He resolved the conflict by paying little heed to his planned course of study, but continued private study in mathematics. The outcome was to leave the university without a degree. For many years, Weierstrass worked behind the scenes to establish his theory of functions of complex variable, based on entire series developments. After that he studied mathematics at the University of Münster (which was even at this time very famous for mathematics) and his father was able to obtain a place for him in a teacher training school in Münster. Later he was certified as a teacher in that city. In 1854 he published a memoir on \textit{Abelian integrals and hyperelliptic integrals inversion}, which established his reputation as a mathematician and earned him an honorary doctorate from Königsberg Universität. Appointed professor at the Berlin Universität, he taught from 1864 to his death. He published only a little during his lifetime and his reputation came mainly from the influence of his lectures in Berlin. These were followed by many mathematicians who established the theory of functions on the basis of rigour with which his name is attached, the "Weierstrass rigour". He is also known to have published an example of a continuous function differentiable nowhere (Weierstrass function).

\parpic[l][t]{
  \begin{minipage}{40mm}
    \fbox{\includegraphics[width=110px,height=140px]{img/medaillons/weyl.eps}}
  \end{minipage}
}
\textbf{Weyl, Hermann} (1885-1955) is one of the most influential mathematicians of the 20th century, one of the first to combine General Relativity with the laws of electromagnetism. His research mainly concentrated on mathematical topology and geometry. He conducted research in quantum mechanics and number theory. Born in Elmshorn near Hamburg (Germany), Weyl studied from 1904 to 1908 in Göttingen and Munich, mainly interested in mathematics and physics. His doctorate in Göttingen was supported under the direction of Hilbert and Minkowski. In 1910, he obtained a teaching position at Göttingen as a private lecturer. He taught mathematics at the ETH of Zürich in Switzerland in 1913. It is at Princeton that he worked with Einstein. Weyl searched the unification of gravitation and electromagnetism. This research gave an explanation of the violation of the non-conservation of parity, a characteristic of weak interactions. In 1918, he introduced the concept of gauge, the first step in what will become the gauge theory. He laid the foundation, giving rise to spinors, that become familiar around 1930. Weyl continued to work at the Institute for Advanced Studies until his retirement in 1952. In reality, his vision was an unsuccessful attempt to model the electromagnetic and gravitational fields as space-time geometric properties. Those works are fundamental to understand the symmetry of the laws of quantum mechanics. He died in Zürich.

\parpic[l][t]{
  \begin{minipage}{40mm}
    \fbox{\includegraphics[width=110px,height=140px]{img/medaillons/weinberg.eps}}
  \end{minipage}
}
\textbf{Weinberg, Steven} (1933-) is born in New York, he began his studies at New York and then at Cornell University (also in New York) and supported in 1957 at Princeton, his thesis on the effects of strong interaction processes dominated by the weak interaction. Researcher at the University of California at Berkeley from 1959 to 1966, he was interested at many problems in quantum field theory, particle physics and astrophysics. Professor at Harvard in 1973, he contributed decisively to the modern understanding of the fundamental interactions. He joined the University of Texas at Austin in 1982. The unification of the fundamental forces used the efforts of modern physicists since Newton, Maxwell and Einstein who, after having united space and time, tried in vain to unify in a single theory gravitation and electromagnetism. The discovery in the early 20th century, of the two nuclear forces, weak and strong interactions, gave a new impulsion to these efforts. In 1967, Weinberg and the Pakistani physicist Abdus Salam proposed independently that electromagnetism and the weak nuclear interaction are derived from a single electroweak interaction, whose gauge symmetry is spontaneously broken and whose vector is a triplet of bosons massive photon. A few years later, experiments at CERN in Geneva brought the first confirmations of the Weinberg-Salam model. The 1979 Nobel Prize in Physics (shared with American Sheldon Lee Glashow to the importance of his pioneering work) rewarded the two authors of what is now named the "standard model" of electroweak interactions. Excellent teacher, Weinberg is the author of several physics course level, both on the gravitational field theory. Popularizer of talent, his book The \textit{First Three Minutes of the Universe} was a worldwide success.

\parpic[l][t]{
  \begin{minipage}{40mm}
    \fbox{\includegraphics[width=110px,height=140px]{img/medaillons/wilcoxon.eps}}
  \end{minipage}
}
\textbf{Wilcoxon, Frank} (1892-1965) was a chemist and statistician known for the development of famous statistical tests. Frank Wilcoxon was born from american parents in County Cork (Ireland). He grew up in Catskill, New York but received part of his education in England. In 1917, he graduated from Pennsylvania Military College with a B.Sc. After the First World War he entered graduate studies, first at Rutgers University, where he was awarded an M.S. in chemistry in 1921, and then at Cornell University, gaining a PhD in physical chemistry in 1924. Wilcoxon entered a research career, working at the Boyce Thompson Institute for Plant Research from 1925 to 1941. He then moved to the Atlas Powder Company, where he designed and directed the Control Laboratory, before joining the American Cyanamid Company in 1943. During this time he developed an interest in inferential statistics through the study of R.A. Fisher's 1925 text, \textit{Statistical Methods for Research Workers}. He retired in 1957. Over his career Wilcoxon published over 70 papers. His most well-known paper contained the two new statistical tests that still bear his name, the Wilcoxon rank-sum test and the Wilcoxon signed-rank test. These are nonparametric alternatives to the unpaired and paired Student's $T$-tests respectively. Wilcoxon died after a brief illness in Tallahassee (Florida - USA).

\parpic[l][t]{
  \begin{minipage}{40mm}
    \fbox{\includegraphics[width=110px,height=140px]{img/medaillons/witten.eps}}
  \end{minipage}
}
\textbf{Witten, Edward} (1951-) is a mathematician and physicist, winner of the Fields Medal in 1990 and born in Baltimore (Maryland). Witten completed his graduate studies at Brandeis University in Waltham (Massachusetts), then Princeton University (New Jersey), where he defended his doctoral thesis in physics in 1974. Researcher at Harvard University from 1976 to 1980, he taught at Princeton University, then became a member of the Institute for Advanced Study (IAS) at Princeton in 1987. After work in theoretical physics of elementary particles, Witten focuses his research on mathematical physics and in particular contributes significantly to the development of superstring theories in the hope that they might emerge to an understanding of the gravitational interaction at the quantum level. In mathematics, he has contributed to the study of Morse theory, proving classical Morse inequalities connecting the critical points to homology. In 1987, he proved an infinite sequence of rigidity theorems on the space of solutions of differential equations, such as the Rarita-Schwinger equation, encountered in physics. In knot theory, he showed in 1989 that we can interpret the Vaughan Jones' invariants of knots as Feynman integrals for 3-dimensional gauge theory. He has, furthermore, explored the relationship between quantum field theory and differential topology of 2 or 3-dimensional varieties. Recent advances in the understanding of 2-dimensional models of gravity are largely due to the influence of the innovative ideas of Witten.

\phantomsection
\addcontentsline{toc}{section}{Y}
\label{sec:Y}

\parpic[l][t]{
  \begin{minipage}{40mm}
    \fbox{\includegraphics[width=110px,height=140px]{img/medaillons/yang.eps}}
  \end{minipage}
}
\textbf{Yang, Chen-Ning} (1922-) is one of the greatest physicists theorists of the second half of the 20th century. He is professor at the Chinese University of Hong Kong and at the Tsinghua University in Beijing, Professor Emeritus of the University of New York at Stony Brook, Yang. Yang obtained his Master of Science from Tsinghua University in 1944. He enrolled in 1946 at the University of Chicago, that Fermi had just joined. Later, he decided to devote himself to theoretical physics, and in 1949 he defended his thesis work on the phenomenology of nuclear reactions. His career began at the Institute for Advanced Studies (IAS) in Princeton in 1949. In 1965, he refused to succeed to Oppenheimer as director, but he decided in 1966 he finally accepted the Einstein Chair and the position of Director of the Institute of Theoretical Physics of the new University of New York at Stony Brook. From 1971 he actively engages in restoring scientific relations between China and the United States and is involved in the creation of new research institutes, especially in Nanjing. Yang's contributions are characterized by their depth, their amplitude and their variety, from the phenomenology of particle quantum field theory, through the statistical mechanics as well as various forays into physics of condensed matter. His great merit are related to two points: firstly, he showed that the hypothesis of space symmetry had not been tested for weak interactions, and secondly, he devised a whole series of new tests for the space reflection invariance. These advanced in the theory of weak interactions have lead, with the introduction of the Yang-Mills fields, to the electroweak standard model. The idea of Yang was to generalize gauge invariance to groups of rotations in 3-dimensional abstract space intended to describe the internal degrees of freedom of matter fields. The Yang-Mills fields imposed themselves as a fundamental tool for the construction of a predictive theory of all weak, strong and electromagnetic interactions, decisive event that engaged the revolution in physics in the 1970s. All of his work have had a considerable impact in theoretical physics. Nearly 20 years after the publication of his article with Mills, Yang gave a precise reformulation of the theory of Yang-Mills fields under strict fiber spaces. The analogy with the theory of gravitation becomes also apparent and the notions of curvature and parallel transport are introduced naturally. Particular solutions of the Yang-Mills equations, such as this discovered by Gerard't Hooft, are used by mathematicians to explore the properties of differential manifolds in 4 dimensions. Yang has received numerous scientific awards including the Nobel Prize of Physics in 1957 that he shared with Tsung-Dao Lee. This prestigious award was granted for their work on parity laws in the field of elementary particles. These fundamental studies are particularly important because they showed that the left-right symmetry of elementary particles, universally accepted at the time, was simply incorrect, which was later proven experimentally.

\parpic[l][t]{
  \begin{minipage}{40mm}
    \fbox{\includegraphics[width=110px,height=140px]{img/medaillons/yates.jpg}}
  \end{minipage}
}
\textbf{Yates, Frank} (1902-1994) was one of the pioneers of 20th century statistics. Yates was born in Manchester, the eldest of five children (and only son) of seed merchant Percy Yates and his wife Edith. He attended Wadham House, a private school, before gaining a scholarship to Clifton College in 1916. In 1920 he obtained a scholarship at St John's College, Cambridge and four years later graduated with a First Class Honours degree. He spent two years teaching mathematics to secondary school pupils at Malvern College before heading to Africa where he was mathematical advisor on the Gold Coast Survey. He returned to England due to ill health and met and married a chemist, Margaret Forsythe Marsden, the daughter of a civil servant. This marriage was dissolved in 1933 and he later married Prascovie (Pauline) Tchitchkine, previously the partner of Alexis Tchitchkine. After her death in 1976 he married Ruth Hunt, his long-time secretary. In 1931 Yates was appointed assistant statistician at Rothamsted Experimental Station by R.A. Fisher. In 1933 he became head of statistics when Fisher went to University College London. At Rothamsted he worked on the design of experiments, including contributions to the theory of analysis of variance and originating Yates's algorithm and the balanced incomplete block design. During World War II he worked on what would later be called operations research. After the war he worked on sample survey design and analysis. He became an enthusiast of electronic computers, in 1954 obtaining an Elliott 401 for Rothamsted and contributing to the initial development of statistical computing. In 1960 he was awarded the Guy Medal in Gold of the Royal Statistical Society, and in 1966 he was awarded the Royal Medal of the Royal Society. He retired from Rothamsted to become a senior research fellow at Imperial College London. He died in 1994, aged 92, in Harpenden.

\parpic[l][t]{
  \begin{minipage}{40mm}
    \fbox{\includegraphics[width=110px,height=140px]{img/medaillons/young.eps}}
  \end{minipage}
}
\textbf{Young, Thomas} (1773-1829) was physicist, physician and British egyptologist born in Milverton and died in London, best known for his discoveries in optics (interference phenomena), elasticity of materials and medicine (explanation of color vision). At the age of fourteen he knew already the basics of more than a dozen languages. Young he began studying medicine in 1792 in London, then went to Edinburgh in 1794 and a year later to Göttingen (Germany), where he received his doctorate in physics in 1796. In 1799, he began practising medicine in London. From 1802 until his death, he served as secretary of the Royal Society. In 1811, Young was appointed to St. George's Hospital in London. He was part of several official scientific committees and, from 1818, he was appointed secretary of the Greenwich Office and editor of the \textit{Nautical Almanac}. In optics, Young discovered the phenomenon of interference, and thus contributed to establish the wave nature of light. He was the first to describe and measure astigmatism and find a physiological explanation for the sensation of color. Young is also known for his work on the theory of capillarity and elasticity. He also contributed to the deciphering of hieroglyphics inscribed on the Rosetta Stone. His writings include extensive work in medicine, physics and Egyptology.

\parpic[l][t]{
  \begin{minipage}{40mm}
    \fbox{\includegraphics[width=110px,height=140px]{img/medaillons/yukawa.eps}}
  \end{minipage}
}
\textbf{Yukawa Hideki} (1907-1981) was a physicist Japanese, born and died in Tokyo, he was the 5th of 7 children who became, for the most of them distinguished scholars. He was quickly interested to mathematics and philosophy. He was admitted at the Department of Physics at Kyoto University in 1926. Great reader, Yukawa became fascinated by the new philosophy accompanying relativity and quantum theory, concepts he had discovered especially in the works of Max Planck. In parallel to his studies, he became aware of the contemporary developments in quantum physics that led to its formulation established in the late 1920s. He graduated from the University of Kyoto in 1929 and began therefore personal research in the double direction of relativistic quantum physics and nuclear physics which only started to emerge. He first focused on the problem of the electron-proton nuclear binding then on the quantum field theory. While teaching quantum physics, Yukawa continued his research on the problems of the physics of nuclei. In 1934, he attacked the problem of the nuclear force that the theory of Fermi was unable to solve. He took an idea he had considerate in his early work, that of a force exchange, passed between the neutron and the proton by a new particle associated with a new field, which he proposed to deduce the properties from the nuclear interaction. It is in  1934 that he discovered the solution, obtaining a relation between the mass of the hypothetical exchange particle and scope of action of nuclear forces. The Yukawa particle, the meson, must had a mass 200 times that of the electron. It was assumed that these mesons had integer spin or none, that they were obeying the Bose-Einstein statistics and that they were provided with positive and negative charges. This work did not attract attention until the day when other researchers announced the discovery of a new particle in cosmic rays, with the mass predicted by Yukawa. It appeared, however, that the interaction of the meson with the material was too weak to be the particle of nuclear forces exchange. The theory of the two mesons solved the difficulty. He had discovered in the meantime the mechanism of disintegration of the nucleus by orbital electron capture by applying the Fermi's theory. He was the first Japanese to receive the Nobel Prize for Physics in 1949 for his mesic theory of nuclear forces. Yukawa founded the Research Institute for Fundamental Physics at Kyoto University and directed it until his retirement in 1970. He was not limited to the activity of physicist, he wrote essays on scientific creativity and militated for campaigned for peace, signing the appeal of Albert Einstein and Bertrand Russell against the use of nuclear weapons.



\phantomsection
\addcontentsline{toc}{section}{Z}
\label{sec:Z}

\parpic[l][t]{
  \begin{minipage}{40mm}
    \fbox{\includegraphics[width=110px,height=140px]{img/medaillons/zeeman.eps}}
  \end{minipage}
}
\textbf{Zeeman, Pieter} (1865-1943) was a physicist born at Zonnemaire (Netherlands) and died at Amsterdam. He became interested in physics at an early age. In 1883 the aurora borealis happened to be visible in the Netherlands. Zeeman, then a student at the high school in Zierikzee, made a drawing and description of the phenomenon and submitted it to \textit{Nature}, where it was published. After Zeeman passed the qualification exams in 1885, he studied physics at the University of Leiden under Hendrik Lorentz. In 1890, even before finishing his thesis, he became Lorentz's assistant. This allowed him to participate in a research program on the Kerr effect (the reflection of polarized light on a magnetized surface). In 1893 he submitted his doctoral thesis on this effect. After obtaining his doctorate he went for half a year to F. Kohlrausch's Institute in Strasbourg. In 1895, after returning from Strasbourg, Zeeman became Privatdozent in mathematics and physics in Leiden. In 1896, three years after submitting his thesis on the Kerr effect, he disobeyed the direct orders of his supervisor and used laboratory equipment to measure the splitting of spectral lines by a strong magnetic field. He was fired for his efforts, but he was later vindicated: he won the 1902 Nobel Prize in Physics for the discovery of what has now become known as the Zeeman effect. As an extension of his thesis research, he began investigating the effect of magnetic fields on a light source. Because of his discovery, Zeeman was offered a position as lecturer in Amsterdam in 1897. In 1900 this was followed by his promotion to professor of physics at the University of Amsterdam. In 1902, together with his former mentor Lorentz, he received the Nobel Prize for Physics for the discovery of the Zeeman effect. Five years later, in 1908, he succeeded Van der Waals as full professor and Director of the Physics Institute in Amsterdam. He retired as a professor in 1935.

	\begin{fquote}[Mark Twain]It takes a thousand men to invent a telegraph, or a steam engine, or a phonograph, or a photograph, or a telephone or any other important thing—and the last man gets the credit and we forget the others. He added his little mite — that is all he did. These object lessons should teach us that ninety-nine parts of all things that proceed from the intellect are plagiarisms, pure and simple; and the lesson ought to make us modest. But nothing can do that.
 	\end{fquote}
 	
 	