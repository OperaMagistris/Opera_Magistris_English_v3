When arriving at the three thousandth A4 page of writing this book and at the occasion of the 3rd edition, it seemed appropriate to us to try to give a rough timeline of the majority of subjects mentioned in this book\footnote{Even if for some dates it is not possible to check if they are legends or real facts...}. This gives a better perspective on the tools used and also to pay tribute to our illustrious predecessors whom we owe our quality of life, our longevity, our mastery of the environment (not necessarily its respect...) and of its understanding.

If there are important dates missing (but only on subjects near of those presented in the various sections of this book!) or that you identify errors, do not hesitate to let us know, it is a first draft, and therefore the chronology can only be improved.

For more information the reader may refer to \href{http://www.wikipedia.com}{{\color{blue} Wikipedia}}, which has reached the top level in the number of available historical dates and in quality (with verification of sources!).

\begin{center}
\textit{This is the story of how history made science and how science entered in history, and how the ideas which emerged made the modern World.}
\end{center}

\includegraphics[width=\textwidth]{img/raphael_school_of_athens.jpg}

\textbf{+2016}\\
The first observation of gravitational waves was made on 14 September 2015 and was announced by the LIGO and Virgo collaborations on 11 February 2016.

\textbf{+2013}
On 14 March 2013 CERN confirmed that CMS and ATLAS have compared a number of options for the spin-parity of this particle, and these all prefer no spin and even parity. This, coupled with the measured interactions of the new particle with other particles, strongly indicates that it is a Higgs boson. This also makes this particle the first elementary scalar particle to be discovered in nature. In July 2017, CERN confirmed that all measurements still agree with the predictions of the Standard Model.

\textbf{+2006}
The  cognitive psychologist and computer scientist Geoffrey Everest Hinton publish \textit{A fast learning algorithm for deep belief nets}, which rejuvinates interest in Deep Learning. 

\textbf{+2001}\\
First release of Opera Magistris but under the name "Sciences.ch". A compendium on Applied Mathematics that has for purpose to merge all the modern knowledge on STEM (science, technology, engineering, and mathematics) and beyond at the Bachelor/Master level with a maximum of details in mathematical developments.

\textbf{+1995}\\
Michel Mayor and Didier Queloz definitively observe the first extrasolar planet around a main sequence star. The same year the first gaseous Bose-Einstein condensate is produced by the physicists Eric Cornell and physicists Carl Wieman at the University of Colorado at Boulder NIST–JILA lab, in a gas of rubidium atoms cooled to $170$ nanokelvins.

\textbf{+1994}\\
The works of the mathematician Andrew Wiles (more than 10 years of research!) give a solution to the Fermat's last theorem. First algorithm using Quantum computer for Prime factorization by Peter Shor.

\textbf{+1992}\\
Support vector machines (SVMs) are invented.

\textbf{+1986}\\
David E. Rumelhart, Geoffrey E. Hinton and Ronald J. Williams invent backpropagation, a new learning procedure for networks of neurone-like units. The engineers Bill Smith and Mikel J. Harry while working at Motorola introduces Six Sigma ($6\sigma$), a set of techniques and tools for scientific process improvement.

\textbf{+1983}\\
The physicists Carlo Rubbia, Simon van der Meer, and the CERN UA-1 team discovered the W and Z bosons which confirm the unification of the weak nuclear and the electromagnetic forces.

\textbf{+1982}\\
The astrophyicist Werner Becker discover the first millisecond pulsar. The same year, Alain Aspect team observe the violation of Bell inequalities.

\textbf{+1978}\\
The cryptologists mathematicians Ronald Rivest, Adi Shamir and Leonard Adelman propose a public key encryption procedure named "RSA" based on the difficulty of factorization into primes numbers.

\textbf{+1976}\\
Development of a method by a Peter Mansfield and Andrew Maudsley to makes possible nuclear magnetic resonance scanners (NMR).

\textbf{+1975}\\
The computer scientist John Holl invents genetic algorithms.

\textbf{+1973}\\
The mathematician Fischer Black and economist Myron S. Scholes publish a financial asset pricing model.

\textbf{+1969}\\
In his \textit{Control charts for measurements with varying sample sizes} the quality engineer Irving Wingate Burr introduce the famous unbiased constants $c_4$ and $d_2$ that bear his name.

\textbf{+1968}\\
Hidden Markov Model (HMM) invented

\textbf{+1967}\\
Jocelyn Bell Burnell and Antony Hewish discover the first pulsar.

\textbf{+1965}\\
Detection by the astrophysicists Arno Penzias and Robert Wilson of the cosmic microwave due to sky background radiation predicted by the theory of the physicist Robert Dicke. Development of the Fast Fourier Transform algorithm by James W. Cooley and John W. Tukey that has huge application in sciences. The physicist John Stewart Bell discover the Bell inequalities.

\textbf{+1964}\\
The astrophysicist Irwin Shapiro predicts a gravitational time delay of radiation travel as a test of General Relativity. The same year, the physicist John Stewart Bell shows that all local hidden variable theories must satisfy Bell's inequality.

\textbf{+1963}\\
The mathematician and meteorologist Edward Lorenz found what is probably the first strange attractor and opens the way to chaos theory.

\textbf{+1962}\\
The mathematician Benoît Mandelbrot discovered fractals by chance in the analysis of signals located at Bell Laboratories in the United States where he will use computers for repeat graphic patterns endlessly and whose principle is the basis of the theory of fractals. The same year, the economist William Forsyth Sharpe publishes the CAPM (Capital Asset Pricing Model). 

\textbf{+1960}\\
The physicist Abdus Salam postulates the existence of W and Z bosons to explain beta decay and the emergence of a new Z boson, which had never been seen before. The same year, the pysicists Ali Javan and Theodore Maiman invented each particular type of LASER. The mathematicians and engineers Irving Reed and Gustave Solomon present the Reed-Solomon error-correcting code.

\textbf{+1959}\\
The physicists Yakir Aharonov and David Bohm predict the Aharonov-Bohm effect (particle turning around a magnetic field but in a region where the magnetic field is zero is sensible to the field through the vector potential) and the following year the physicist Robert G. Chambers confirmed the effect experimentally. 

\textbf{+1958}\\
The perceptron is developed at Cornell University by Frank Rosenblatt. It is the first program able to learn by trial and errors.

\textbf{+1957}\\
The physicists John Bardeen, Leon Neil Cooper and John Robert Schrieffer propose and theory of supraconductivity.

\textbf{+1956}\\
The physicists Clyde Cowan and Fred Reines observe the neutrino hypothesized 25 years ago by the physicist Wolfgang Pauli. The same year, the  linguist, philosopher, cognitive scientist, historian, social critic, and political activist Avram Noam Chomsky publish Three Models for the Description of Language where he introduces the classification of formal grammars named today the "Chomsky hierarchy", which contains the class of out-of-context grammars, playing an important role in computer science to create programming languages. In the Logic Theorist by the cognitive computer scientist Allen Newell and the political scientist, economist, sociologist, psychologist, and computer scientist Herbert Simon, the first artificial intelligence computer program is proposed, which produced evidence in Russell and Whitehead's Principia Mathematica system. For one of the theorems, the program produces a simpler proof than that presented in the Principia!

\textbf{+1955}\\
The physicists Owen Chamberlain, Emilio Gino Segrè, Clyde Wiegand and Thomas Ypsilantis discover the antiproton.

\textbf{+1954}\\
The economist Harry Markowitz published his thesis on the efficient diversification model of financial assets portfolios. The same year, the physicists John Bell and the duo Wolfgang Pauli and Gerhart Lüders develop the CPT theory analyzing the symmetry of physical laws for transformations involving simultaneously the charge, parity and time. The physicist Charles Hard Townes developed the MASER. The biologist Milner Baily Schaefer publishes his equilibrium populations model.

\textbf{+1953}\\
Monte Carlo Markov Chain (MCMC) invented. Bayesian inference finally becomes tractable on real problems.


\textbf{+1952}\\
The mathematician George Bernard Dantzig developed the simplex algorithm for operational research.

\textbf{+1951}\\
The CPT theorem appears for the first time implicitly in the work of the physicist Julian Schwinger to prove the correlation between spin and statistics.

\textbf{+1950}\\
The physicists Johannes Hans Daniel Jensen and Maria Goeppert-Mayer developed the shell model of the nuclear core. The same year, the economist and mathematician John Forbes Nash developed the concept of non-cooperative games and generalizes the notion of minimax for zero-sum games; the engineer David Huffman found the algorithm used to compress any type of series symbols. In his \textit{Error-detecting and error-correcting codes} the mathematician Richard Wesley Hamming presents the Hamming family of code, linear codes used to o detect and correct transmission errors.

\textbf{+1949}\\
The mathematician and electrical engineer Claude Shannon published an article containing the information theory which will became the foundation of a number of physical theories, statistics and numerical methods. The physicist Richeard Feynman proposed the interpretation of the positron as an electron moving backward in time in his paper \textit{The Theory of Positrons}.

\textbf{+1948}\\
The physicist Maria Göpper-Meyer develops with success a theoretical model for the structure of the atomic nucleus and textile engineer and statistician Genichi Taguchi developed the experimental designs (DOE) that bear his name. The physicist Richard Feynman introduce the diagrams that bear his name and also the path integral formulation. The physicist Polykarp Kusch measure the anomalous magnetic moment of the electron (deviation of theoretical prediction of Dirac theory) thus leading to reconsideration of and innovations in quantum electrodynamics.

\textbf{+1947}\\
The physicists Cecil Powell, Cesare Mansueto Giulio Latte, and Giuseppe Occhialini discover the pion in the study of cosmic rays. The same year the physicists John Bardeen, Walter H. Brattain and William Schockley invent semiconductor transistors in the laboratories of the Bell telephone company in the United States that will cause the computer revolution. Measurement of the Lamb shift by Willis Eugene Lamb (shift of energy spectrum non-predicted by Dirac theory but explain in the framework of Quantum Electrodynamics).

\textbf{+1946}\\
The physicist and chemist Willard Frank Libby develops and discovers the possibility of Carbon 14 dating. The same year, the physicists Walter Houser Brattain, John Bardeen and William Bradford Shockley discover the transistor effect.

\textbf{+1945}\\
The Trinity Test, the first successful detonation of a nuclear weapon by the physicist Robert Oppenheimer and his team in New Mexico.

\textbf{+1944}\\
The mathematician John von Neumann developed the foundations of the mathematical theory of games.

\textbf{+1943}\\
The physicist Tomonaga Sin-Itiro published an article posing the physical basis of quantum electrodynamics.

\textbf{+1942}\\
The physicist Enrico Fermi and his team conducted the first controlled chain reaction in the purpose to build the first atomic bomb.

\textbf{+1941}\\
The physicist Ernst Stueckelberg interprets positrons as electrons with positive energy traveling back in time. The physicist Lev Davidovich Landau publish a theory of superfluidity.

\textbf{+1940}\\
The physicist Edward Teller sees the possibility of using the enormous amount of heat generated by the explosion of a fission bomb to trigger the nuclear fusion process. This is the approach considered as the discovery of nuclear fusion. The same year, the physicist William Donald Kerst developed the first betatron. The physicist John Wheeler in a telephone call to Richard Feynman hypothesizes that all electrons and positrons are actually manifestations of a single entity moving backwards and forwards in time (the "one-electron universe postulate").

\textbf{+1939}\\
The chemists Otto Hahn and Fritz Strassmann bombard uranium with neutrons and discovered that barium is produced by the experience (discovery of nuclear fission). The same year, the physicists Lise Meitner and Otto Robert Frisch determine that nuclear fission occurred during the Hahn-Strassman experiment. The physicists Wolfgang Pauli, Markus Fierz and Frederik Jozef Belinfante prove that the properties of permutation of identical particles, bosons or fermions are controlled by their spin.

\textbf{+1938}\\
The chemist and physicist Isidor Isaac Rabi and his colleagues studied the effects of placing beams of molecules in strong external magnetic fields, leading to the development of nuclear magnetic resonance (NMR). The same year, the physicists Hans Bethe and Carl von Weizsäcker propose a nuclear theory of stars and the mathematician and electrical engineer Claude Shannon publish what is probably the most famous master's thesis of the 20th century (\textit{A symbolic analysis of relay and switching circuits}), and prove that it is possible to simplify the design of logic circuits by using the Boolean algebra. This master's thesis has played an important role in the design of electronic computers. The physicist Piotr Leonidovich Kapitza observed the phenomenon of superfluidity.

\textbf{+1937}\\
The physicists Seth Neddermeyer, Carl Anderson, Jabez Curry Street and E.C. Stevenson discover muons in the traces left by cosmic rays in a bubble chamber. The same year the mathematician John von Neumann developed the Monte Carlo methods for various numerical methods and the physicist Niels Bohr developed the liquid drop model of the nucleus.

\textbf{+1936}\\
The physicists George Gamow and Edward Teller work together to formulate the theory of beta radioactive emissions. The same year, in his \textit{On computable numbers}, the computer scientist, mathematician, logician, cryptanalyst, philosopher and theoretical biologist Alan Mathison Turing analyzes the concept of computability using the Turing machine concept, one of the foundations of theoretical computing. The engineer, statistician, professor, author, lecturer, and management consultant publish his works on the CUSUM, UWMA and EWMA control charts.

\textbf{+1935}\\
The physicist Hideki Yukawa present the strong interaction theory and predicted the existence of mesons. The same year, the astrophysicist and mathematician Subrahmanyan Chandrasekhar reports the results of his researches on the collapse of stars into white dwarfs and beyond 1.44 solar masses into neutron stars. The article by Albert Einstein, Boris Podolsky and Nathan Rosen on the EPR paradox is published in Physical Review and questioned the nonlocality of the Copenhagen interpretation.

\textbf{+1934}\\
The physicist Pavel Cherenkov Alekseyevich study the emission of light when relativistic particles pass through an amorphous medium. The same year the physicist Enrico Fermi suggested to bombard uranium atoms with neutrons to obtain an element with 93 protons, formulated the theory of beta decay and the physicist Leó Szilárd realizes that a nuclear chain reaction is possible. The physicists Irène Joliot-Curie and Frédéric Joliot bombard aluminum atoms with alpha particles and create artificially radioactive phosphorus-30.

\textbf{+1933}\\
The mathematician Andrei Nikolaevich Kolmogorov published a book containing a solid base of axioms of probability. The physicist Ernst August Friedrich Ruska realize the first electronic microscope by transmission using electrons instead than photons.

\textbf{+1932}\\
The physicist Carl David Anderson discovered the positron. The same year the physicist Werner Heisenberg present the theoretical model of the proton-neutron nuclear core and uses it to explain isotopes. The physicist James Chadwick discovered the neutron and the physicists John Cockcroft and Ernest Walton break the nuclear core of lithium and boron by proton bombardment.

\textbf{+1931}\\
The physicist Wolfgang Pauli puts forward the hypothesis of neutrino to explain the apparent violation of the principle of conservation of energy in beta decay. The same year the mathematician and logician Kurt Gödel showed that a system can be both consistent and complete (incompleteness theorem) and that if the system is coherent, then the coherence of the axioms can not be proved within the system. The physicist Ernest Lawrence invents the first cyclotron and the physicist, engineer and statistician Walter Andrew Shewhart publish his book \textit{Economic Control of Quality of Manufactured Product} where he introduces the main control charts.

\textbf{+1930}\\
The physicist Fritz London explains that Van der Waals forces are due to the interaction of the dipole moments of molecules. The same year the physicist Paul Dirac present his electron-hole theory and the economist John Maynard Keynes publish his A Treatise on Money.

\textbf{+1929}\\
The astronomer Edwin Hubble by studying the redshift hypothesizes that the Universe is not static. The same year, the physicist Robert Van de Graaff invents the first particle accelerator, known today as the "Van de Graaff accelerator".

\textbf{+1928}\\
The physicist Paul Dirac established his relativistic wave equation for the electron, which generalizes and improves the spinless relativistic equation of Klein-Gordon. The same year, the physicists Friedrich Hund and Robert S. Mulliken introduces the concept of molecular orbital and the physicist and cosmologist George Gamow develops the theoretical model of quantum alpha decay by tunneling. The physicist Félix Bloch studies the problem of a particle submitted to a periodic potential and develops the band theory that will become a main foundation of solid state theory, particularly to understand their transport or optical properties.

\textbf{+1927}\\
The physicist Werner Heisenberg establish the uncertainty principle, by which the position and momentum of a particle can not be simultaneously known with accuracy, indirectly by developing a new theoretical basis for quantum mechanics. The same year, the physicists Walter Heitler and Fritz London present the quantum theory of the chemical bond established from the hydrogen molecule and the physicist Max Born interprets the wave function of Schrödinger as probabilities and with the help of the physicist Robert Oppenheimer presents the Born-Oppenheimer approximation. The physicists Clinton Joseph Davisson, Lester Germer and George Paget Thomson confirm the wavelike nature of the electrons by diffraction. The physicist Paul Ehrenfest prove the famous quantum physics theorem that bears his name. The physicist Paul Dirac introduces the quantification of the electromagnetic field.

\textbf{+1926}\\
The physicist Erwin Schrödinger establish his wave equation that defines quantum mechanics in an analytical form by developing the ideas of De Broglie on theory of wave mechanics and he proves that the wave and matrix formulations of quantum theory are mathematically equivalent. The same year the physicists Oskar Klein and Walter Gordon establish the equation of relativistic quantum mechanics for spinless particles and Paul Dirac defines the Fermi-Dirac statistics. In the field of population dynamics the physicist and mathematician Vito Volterra published the nonlinear differential equation modeling the predator/prey equilibrium.

\textbf{+1925}\\
The physicist Pierre Auger discovered the Auger effect (2 years after Lise Meitner) and the same year the physicists George Uhlenbeck and Samuel Goudsmit postulate and reveal the existence of electron spin. Also the same year, the physicist Wolfgang Pauli established by necessity the principle of quantum exclusion. The physicists Werner Heisenberg, Max Born and Pascual Jordan formulate quantum matrix mechanics.

\textbf{+1924}\\
The physicist John Lennard-Jones proposed a semi-empirical description of inter-atomic interaction forces and the same year, the physicists Satyendranath Bose and Albert Einstein define the Bose-Einstein statistics. In the field of statistics, the statistician Ronald Fisher defines major modern concepts in statistics.

\textbf{+1923}\\
The astronomer Edwin Hubble estimates the distance between the Earth and the spiral galaxies, showing that they are far from the Milky Way and in the same year the physicist Louis de Broglie suggested the wave-particle duality from quantum theory and from the mass-energy equivalence and the physicist Lise Meitner discovers the Auger effect. The mathematician Norbert Wiener introduces Brownian movement theory.

\textbf{+1922}\\
The physicist Arthur Compton studied the scattering of X photons by electrons and the astrophysicist Alexander Friedmann develops non static universe models.

\textbf{+1921}\\
The physicist Alfred Landé defines the gyromagnetic ratio and introduces also half integer quantum numbers. The same year physicists Otto Stern and Walter Gerlach show experimentally that the intrinsic moment of the electron is quantized. The mathematician and physicist Theodor Franz Eduard Kaluza proved that a five-dimensional version of Einstein's equations unifies gravitation and electromagnetism.

\textbf{+1920}\\
The astronomer Vesto Melvin Slipher highlights the phenomenon of red shift in the spectrum of galaxies. The same year, the physicist Arnold Sommerfeld introduces a fourth quantum number to the original model of the Bohr's atom. the physicist Niels Bohr introduce the correspondance principle assuring the transition quantum physics $\mapsto$ classical physics when $\hbar\rightarrow 0$. The astronomer and astrophysicist Ernst Julius Öpik confirms that the "Andromeda nebula" is located outside the Milky Way-this confirms that our universe is much larger than we thought because it is not limited to the Milky Way.

\textbf{+1919}\\
The physicist Ernest Rutherford performed the first artificial disintegration of an atom by bombarding nitrogen with alpha particles. The same year, the physicist and mathematician Amali Emmy Noether develops his theorem on differential invariants in the calculus of variations, one of the most important mathematical theorems ever proved in guiding the development of modern physics.

\textbf{+1918}\\
The astronomer Harlow Shapley made the first accurate estimate of the size of our galaxy and the Sun's position in it. The same year, the physicist Hermann Weyl introduces the notion of gauge, the first step in what will become the gauge theory.

\textbf{+1917}\\
The physicist Albert Einstein introduces the idea of stimulated emission of radiation used in the manufacturing base of LASER. The same year, the physicist Arnold Sommerfeld introduces a third quantum number to the original model of the Bohr's atom.

\textbf{+1916}\\
The physicist Albert Einstein developed his theory of General Relativity and how matter plays on the space-time to produce gravitational effects. This is the first theory named "background independent". The same year, the physicists Gilbert Lewis and Irving Langmuir present the electronic shell model to explain chemical bonds and the physicist Arnold Sommerfeld introduces relativity in his model of 1915 and this relativistic correction explained the observed values by high resolution spectrographs and therefore the splitting of spectral lines named "fine structure" and he introduces at the same time a second quantum number describing elliptical orbits. The physicist Karl Schwarzschild found a mathematical solution of Einstein's equations, that he applies to neutron stars and black holes.

\textbf{+1915}\\
The physicist Arnold Sommerfeld refines the atomic model of the physicist Niels Bohr by introducing elliptical orbits to explain the fine structure lines of the hydrogen atom. This new model does not, however, explain the range of the observed spectra of the hydrogen atom. The same year the physicist Albert Einstein calculates the trajectory of Mercury with General Relativity. He finds that his theory to calculate the perihelion advance of Mercury with high accuracy and the bending of light rays in the gravitational field of the Sun. The end of that year he submitted the article that describes the field equations of gravitation, these equations will be the basis of the theory of General Relativity.

\textbf{+1914}\\
The physicist Ernest Rutherford showed that the positively charged atomic nuclei contain protons. The same year the physicist Albert Einstein and the mathematician Marcel Grossmann published an article on tensor calculus, and more particularly on the Riemann-Christoffel and Ricci tensor (more generally on tensor analysis and differential geometry) and the physicist Peter Debye develops a model of the behavior of the thermal capacity of the solids as a function of temperature. The mathematician Felix Hausdorff introduces the concepts of Hausdorff distance and Hausdorff dimension.

\textbf{+1913}\\
The physicist Niels Bohr present the quantum model in circular layers of the atom and the same year the physicist Robert Millikan measures the fundamental electric charge. The same year, the physicists William Henry Bragg and William Lawrence Bragg find the Bragg condition for strengths X-ray reflection and the physicist Henry Moseley showed that the atomic number is the true criterion of elements discrimination. In mathematics, the mathematician Elie Cartan announces his discovery of spinors. The pysicist Johannes Stark demonstrates that strong electric fields will split the Balmer spectral line series of hydrogen.

\textbf{+1912}\\
The physicist Max von Laue proposes the use of crystal lattices to diffract X-rays and in the same year the physicists Walter Friedrich and Paul Knipping diffract X-rays using zinc sulfide. The same year, the physicist Ernest Rutherford proposes the use of radioactivity as a means of dating. The chemists Otto Sackur and physicist Hugo Tetrode prove a formula to calculate the entropy of a mono-atomic gas (first apparition of Planck constant in thermodynamics).

\textbf{+1911}\\
The physicist Ernest Rutherford discovered the atomic nucleus by bombarding a thin gold foil with alpha particles. Some particles bounce on the core of gold atoms. The same year, the physicist and chemist Jean Perrin proves the existence of atoms and molecules and the physicist Heike Kammerlingh Onnes discovers superconductivity.

\textbf{+1911}\\
The physicist Robert Andrews Millikan determine the electric charge carried by a single electron with his famous oil-drop experiment in which he replaced water (which tended to evaporate too quickly) with oil.

\textbf{+1909}\\
The physicists Hans Geiger and Ernest Marsden discover that alpha particles can be strongly deflected by thin metal foils and in the same year the physicists Ernest Rutherford and Thomas Royds demonstrated that alpha particles are helium atoms ionized twice. In the field of Applied Mathematics, the mathematician Agner Krarup Erlang published the first paper on the theory of queues. 

\textbf{+1908}\\
The statistician William Sealy Gosset published an article proposing a new statistical distribution and a new statistical test named respectively the" Student law" and "Student's t-test". The same year, the mathematician Ernst Friedrich Ferdinand Zermelo proposed an improvement of the axioms of set theory. In his \textit{Mendelian proportions in a mixed population}, the mathematician Godfrey Harold Hardy exposes what is now known as the "Hardy-Weinberg principle" in genetics that establishes how dominant and recessive genetic traits spread in a large population. Hardy is known as Fundamental mathematician, specialist in number theory, but has contributed significantly to the study of population genetics by the results presented in this article, found independently by the obstetrician-gynecologist Wilhelm Weinberg.

\textbf{+1907}\\
The physicist Albert Einstein deduced the expression of the famous equivalence between mass and energy, and the same year he established the expression of the heat capacity of crystalline solids and calculates the gravitational redshift. The mathematician and physicist Hermann Minkowski unified space-time together in a unified mathematical structure. The mathematician Guido Fubini proves the multiple integral theorem that bears his name.

\textbf{+1906}\\
The physicist Walther Nernst presents a formulation of the third law of thermodynamics. The same year, the mathematician Andrei Markov published the first work on the chains of events that will later bare his name and that have occupied an important place in the quantum physics of his time. 

\textbf{+1905}\\
The physicist Albert Einstein explained the photoelectric effect by the existence of quantums and the same year he explained mathematically the Brownian motion as a result of the random motion of molecules and published his research on his theory of Special Relativity that proves the equivalence of mass and energy. The physicist Paul Langevin publish his theory on the susceptibility of paramagnetic materials.

\textbf{+1904}\\
The physicist Antoon Lorentz discovers the contraction of time in the direction of movement of the body relatively to the constant speed of light and proposed the transformation equations of the electromagnetic forces. The same year, the physicist Hantaro Nagaoka proposed a theoretical Saturnian model of the atom where electrons revolve around a massive positive nucleus like the rings of Saturn.

\textbf{+1903}\\
Radioactivity is explained in terms of fission of atoms by the physicist Ernest Rutherford and by the radiochemist Frederick Soddy.

\textbf{+1902}\\
The physicist Philipp Lenard observed that the photoelectric effect does not depend on the power of the light beam but its frequency. The same year, the chemist Theodor Svedberg suggests that fluctuations of molecular bombardment creates Brownian motion and the logician Bertrand Russell propose his "ultimate" paradox undermining the naive set theory. The physicist James Jeans describes the gravitational collapse phenomenon that can occur, for example in a cloud of gaseous material based on a critical mass or radius. The mathematician Henri Léon Lebesgue put the basis of measure theory and introduce the Lebesgue integral.

\textbf{+1901}\\
The mathematicians Gregorio Ricci-Curbastro and his assistant Tullio Levi-Civita developed tensor calculus.

\textbf{+1900}\\
The physicist Max Planck suggested that light can be emitted in discrete frequency generalizing the law of black body radiation. The same year, the physicist Johannes Rydberg refines the mathematical expression for the wavelengths of the Balmer's lines of the hydrogen and the physicist Paul Villard discovered gamma rays by studying the decay of uranium. In the field of Applied Mathematics, the mathematician Louis Bachelier developed the Brownian model motion applied to game and speculation theory that will be the mainstay of quantitative financial tools of the 20th century. The same year, the mathematician Karl Pearson defines the statistical distribution of the Chi-2 and explores important properties of this distribution for statistical inference. The physicist Paul Drude adaptated the kinetic theory of gases to electrons in metals and gets a model that still bears his name.

\textbf{+1899}\\
The physicist Ernest Rutherford discovered that the radiation emitted by uranium compounds are positively charged alpha particles and beta particles negatively charged. The same year, the mathematician David Hilbert replaces the five usual axioms of Euclidean geometry axioms by 21 to eliminate the weaknesses of Euclidean geometry. 

\textbf{+1898}\\
The mathematician David Hilbert gives a first approach of the class field. The same year, the physicists Marie and Pierre Curie isolate and study the radium and polonium and the physicist Wilhelm Wien Carl Werner identifies a new particle with a positive charge approximately equal to the mass of hydrogen that he will name the "proton". The engineer Alfred-Marie Liénard calculates the electromagnetic field produced by a point charge, animated with any movement. The mathematical expressions that have been independently established, but 2 years later by the physicist Emil Wiechert.

\textbf{+1897}\\
The physicist Joseph John Thomson measure the charge/mass ratio of certain negative particles created by cathodic rays. He measures their charge, and he concludes that their mass is about 2'000 times smaller than hydrogen. These particles were later named "electrons", a term suggested by the physicist George Johnstone Stoney. Televisions and other CRTs are improved versions of the Thomson's device.

\textbf{+1896}\\
The physicist Henri Becquerel discovered radioactivity of uranium and the same year the physicist Pieter Zeeman studied the decomposition of the sodium D lines when it is heated in a strong magnetic field and he discovered that the spectral lines of a light source subject to a magnetic field has many components, each having a certain polarization. To explain this phenomenon, one must add additional quantum number named "magnetic quantum number". The same year, the physicist Wilhelm Wien Carl Werner establishes the law that bears his name for the energy emitted by the black body.

\textbf{+1895}\\
The physicist Wilhelm Röntgen discovered X-rays and the same year the physicist and inventor Guglielmo Marconi carried out in the Swiss Alps at Salvan the first wireless "long distance" of 1.5 kilometers.

\textbf{+1893}\\
The mathematician, physicist, engineer and philosopher Henri Poincaré published his studies on the three-body problem and introduces the qualitative study of differential equations and chaos theory. The same year the mathematician Georg Cantor developed the theory of transfinite sets and the proposal of the engineer Nikola Tesla use AC instead of DC current is adopted by the first U.S. state. In his \textit{Uniplanar Algebra}, the mathematician Irving Stringham uses the symbol $\ln$ for the natural logarithm instead of the traditional $\log_e$.

\textbf{+1892}\\
The autodidact physicist Oliver Heaviside reduced the 8 equations of electrodynamics of Maxwell to $4$ differential equations. 

\textbf{+1891}\\
In his \textit{Arithmetices Principia, ova methodo exposita} the mathematician Giuseppe Peano introduces the axioms to build the Natural numbers set $\mathbb{N}$ and the symbol of appartenance and a first version of the quantifiers symbolic. Their final form will be given by the mathematician David Hilbert. He provides more than $40,000$ definitions in a language that he wants as universal.

\textbf{+1890}\\
The systematic study of groups is growing with the mathematician Sophus Lie, Issai Schur and Élie Cartan. This last one introduces the notion of algebraic group and continuous groups. 

\textbf{+1889}\\
The mathematician Giuseppe Peano postulates $5$ properties of integers as axioms in the idea to do with integers what Euclid did for geometry. He defines also the axiomatic of vectorial space in $\mathbb{R}$ and introduce the concept of linear application. He introduce also the notation $\cup$ and $\cap$ for the union and intersection of sets.

\textbf{+1888}\\
The anthropologist, explorer, geographer, inventor, meteorologist, proto-geneticist, psychometrician, and statistician ... Francis Galton defines the concept of statistical correlation coefficient. The same year, the mathematician Richard Dedekind proposes the definition of a finite set. 

\textbf{+1887}\\
The physicists Albert Michelson and Edward Morley measured the speed of light to test the hypothesis of ether that their experimental results reject and the same year the physicist Heinrich Hertz discovered the photoelectric effect and conducted experiments on electromagnetic waves (production and reception).

\textbf{+1886}\\
The mathematician and physicist Oliver Heaviside introduces the handling differential operators as algebraic entities which will bring up later to the Laplace transforms.

\textbf{+1885}\\
The chemist Johan Balmer found the mathematical expression which gives the wavelength of the different lines of the spectrum of hydrogen.

\textbf{+1884}\\
The physicist and chemist Willard Gibbs defines the notation still in use in the early 21st century for the scalar and vector product as well as vector differential operators in its books about to vector calculus. The same year, the physicist Ludwig Boltzmann derives the Stefan-Boltzmann black body radiant flux law from thermodynamic considerations and the physicist John Henry Poynting introduces the vector that still today bear his name. The mathematician Karl Hermann Amandus Schwarz proves that the sphere is the solid with the minimal surface for a given volume (this result explains many shapes visible in the Universe).

\textbf{+1882}\\
The mathematician Ferdinand von Lindemann proved the transcendence of $\pi$. The same year, the astronomer, mathematician, economist and statisticien Simon Newcomb observes a $43''$ per century excess precession of Mercury's orbit. Where Isaac Newton mechanics explains correctly the precession period of other planets, it failed to explains that of Mercury.

\textbf{+1880}\\
The mathematician and physicist Oliver Heaviside reduced the $8$ Maxwell equations to 4 equations and introduces the step function that always bear his name today (Heaviside step function).

\textbf{+1879}\\
The mathematician and physicist Joseph Stefan publishes Stefan's law which states that the power transmitted across the whole spectral range is proportional to the fourth power of the absolute temperature of a star and the surface of it. For the same surface temperature, a star is also brighter the more it is big. The same year, the physicist Edwin Herbert Hall discovered that an electric current through a material immersed in a magnetic field generates a voltage perpendicular to the initial direction of the electric current. The philosopher, logician, and mathematician Gottlob Frege publish his \textit{Begriffsschrift, eine der arithmetischen nachgebildete Formelsprache des reinen Denkens} where he introduces the axiomatic predicate logic, the quantifiers (for all $\forall$, it exists $\exists$), the theory of quantified variable, the rigorous concept of formula/function and of variable.

\textbf{+1878}\\
The mathematician and philosopher William Kingdon Clifford introduces the divergence operator.

\textbf{+1877}\\
The physicist and chemist Willard Gibbs defines for chemical reactions two useful quantities, namely the enthalpy that represents the heat of reaction at constant pressure and the free energy that determines whether a reaction can proceed as spontaneous at constant temperature and pressure. The physicist John William Strutt (3rd Baron Rayleigh)introduces the foundations of modern sound theory. 

\textbf{+1876}\\
The mathematician and philosopher William Kingdon Clifford suggests that the motion of matter may be due to changes in the geometry of space.

\textbf{+1874}\\
The physicist Lord Kelvin formally states the second law of thermodynamics. The same year, the mathematical economist Léon Walras published his Elements of Pure Economics.

\textbf{+1873}\\
The mathematician Georg Cantor laid the foundations of the theory of sets and cardinals and shows that the algebraic numbers are in fact countable and defines rigorously the real numbers $\mathbb{R}$, and introduce the his famous diagonal method. The same year, the physicist James Clerk Maxwell showed that light is an electromagnetic phenomenon and reduces the equations of electrodynamics to $8$ instead of $20$ equations (at the same times he defines the curl operator) and the physicist Johannes van der Waals introduces the idea that there are weak attractive forces between molecules. In his \textit{Sur la fonction exponentielle} the mathematician Charles Hermite proves the transcendence of $e$ (two proofs, one of $11$ pages and the other of $20$ pages).

\textbf{+1872}\\
The mathematician Karl Weierstrass presented at the Royal Academy of Sciences in Berlin an example of a function continuous everywhere but differentiable nowhere.

\textbf{+1871}\\
The chemist Dmitri Mendeleev examines his periodic table and predicted the existence of gallium, scandium and germanium. The same year the physicist James Clerk Maxwell established thermodynamic Maxwell relations. 

\textbf{+1870}\\
The physicist Rudolph Clausius proves the (scalar) Virial theorem.

\textbf{+1869}\\
The chemist Dmitri Mendeleev proposed the periodic table of elements which still bears his name.

\textbf{+1867}\\
The historian, journalist, philosopher, economist, sociologist Karl Marx published Das Kapital. 

\textbf{+1866}\\
The physicist James Clerk Maxwell elaborates, independently of the physicist Ludwig Boltzmann, the kinetic theory of gases of Maxwell-Boltzmann. The same year, the monk and botanist Gregor Johann Mendel formulated the laws of statistics hybridization (experiment on $29,000$ peas...) and the mathematician Leopold Kronecker used for the first time the symbol that bear always today his name.

\textbf{+1865}\\
The physicist James Clerk Maxwell publishes for the first time the equations of electrodynamics in the form of equations 20 with 20 unknowns using quaternions. 

\textbf{+1862}\\
The physicist Gustav Kirchhoff developed the concept of Black body that can absorb and emit radiation at all frequencies and that the energy emitted depends only on the frequency of the emitted radiation and the temperature of the black body itself.

\textbf{+1859}\\
The physicist James Clerk Maxwell discovered the law of distribution of molecular velocities. The same year the astronomer Urbain Le Verrier reported an anomaly in the motion of Mercury not predictable by Newton's law and the physicist Gustav Kirchhoff with the chemist Robert Wilhelm Bunsen developed prism spectroscopy. 

\textbf{+1858}\\
The lawyer and mathematician Arthur Cayley emerges the notion of vector space, the notion of matrix and exposes the utility by using the multiplication of matrices and determinants; he rewrites the system of linear equations in matrix form. His works are often seen as the emergence of linear algebra.

\textbf{+1855}\\
The astronomer and physicist Léon Foucault discovered that the force required for the rotation of a copper disc increases when must rotate with its rim between the poles of a magnet, the disk heating at the same time because of the "Foucault's currents" induced in the metal. 

\textbf{+1854}\\
The mathematician George Boole published his system of symbolic logic, now known as Boolean algebra. The same year, the mathematician Arthur Cayley shows that quaternions can be used to represent rotations in four-dimensional space and the mathematician Georg Friedrich Bernhard Riemann gave a new definition of the integral and lays the foundations of differential geometry.  The mathematician Charles Hermite defines the concept of orthogonal matrices and prove that their eigenvalues are real numbers.

\textbf{+1852}\\
The physicists James Joule and William Thomson Kelvin show that gas in expansion cools quickly.

\textbf{+1851}\\
The astronomer and physicist Léon Foucault made a spectacular proof of the rotation of the Earth by suspending a pendulum with a long cable attached to the dome of the Pantheon in Paris. The same year, the mathematician Georg Friedrich Bernhard Riemann published the first work on functions with a complex variable. In his \textit{Paradoxien des Unendlichen}, the  mathematician, logician, philosopher and theologian Bernardus Placidus Johann Nepomuk Bolzano discusses of the issues related to the manipulation of infinites in mathematics. 

\textbf{+1850}\\
The mathematicians Arthur Cayley and James Joseph Sylvester introduces the term matrix and the same year, the mathematician Richard Dedekind introduces the term field. The physicist Rudolf Clausius developed the mechanical theory of heat and formulated the second principle of thermodynamics. The physicist George Stokes proves the famous theorem that bears his name.

\textbf{+1849}\\
The mathematicien and astronome Edouard Roche finds the limiting radius of tidal destruction and tidal creation for a body held together only by its self gravity and uses it to explain why Saturn's rings do not condense into a satellite.

\textbf{+1848}\\
The physicist William Thomson Kelvin discovers the absolute 0 point temperature and sets its own unit of measurement. The same year the physicist and astronomer Hippolyte Fizeau transposes the results of Christian Doppler to the light that like sound has a wave nature (Doppler effect) and highlights the redshift and towards the blueshift.

\textbf{+1847}\\
The physicist Jame Joule founds experimentally the mechanical equivalent of heat and the same year, the physiologist and physicist Hermann Helmholtz formally states the law of conservation of energy.

\textbf{+1845}\\
The physicist Gustav Kirchoff defines the concept of electric potential and sets the laws of networks that bear his name (node law, mesh law). The same year, the physicist George Stokes publishes what will be the basis of the Navier-Stokes fluid mechanics and the physicist Michael Faraday discovers that light propagation in a material can be influenced by external magnetic fields.

\textbf{+1844}\\
The mathematician Joseph Liouville proves the existence of an infinite number of transcendental numbers.

\textbf{+1843}\\
The physicist, mathematician and astronomer William Rowan Hamilton defines sets of complex vector spaces (quaternions). The concept of vector space will be clearly defined by the mathematician and astronomer August Ferdinand Möbius and the mathematician and linguist Giuseppe Peano $40$ years later. The same year, hhe mathematician Laurent Pierre Alphonse publishes his memoir on what will later became the series that bear his name in complex analysis.

\textbf{+1842}\\
The principle of conservation of energy is expressed by the physicist Julius von Mayer who calculated the amount of work that can be obtained by converting heat energy, this means the mechanical equivalent of calories. The same year, the physicist Christian Doppler discovered the acoustic effect that bears his name (change in frequency with the relative movement).

\textbf{+1841}\\
The mathematician Karl Weierstrass discovers but does not publish the Laurent expansion theorem. The same year, the mathematician Carl Gustav Jacob Jacobi introduces the Jacobian matrices and reintroduced the partial derivative notation originally proposed by the mathematician André-Marie Legendre.

\textbf{+1838}\\
The astronomer and mathematician Friedrich Bessel measure that the distance that separates us from the star 61 Cygni is about 96 trillion kilometers.

\textbf{+1835}\\
The mathematician Carl Friedrich Gauss gives a rigorous construction of the complex numbers and the mathematician Augustin Louis Cauchy establishes a first theory of determinants. The same year, the mathematician and engineer Gaspard Coriolis proves that the acceleration of a mobile in a rotating frame is subjected to a complementary force perpendicular to the direction of movement of the mobile in this reference frame.

\textbf{+1834}\\
The engineer and physicist Émile Clapeyron presents a formulation of the second law of thermodynamics. The same year, the physicist Heinrich Lenz establishes the law of electromagnetic induction.

\textbf{+1832}\\
The physicist Michael Faraday established the basic theory of electrolysis.

\textbf{+1831}\\
The physicist Michael Faraday discovered electromagnetic induction, namely the obtention of an electric current from the change of a magnetic field (principle of the dynamo). The same year the mathematician and physicist Carl Friedrich Gauss provides two of the four Maxwell equations.

\textbf{+1829}\\
The mathematician Evariste Galois presents the first draft of his work on solvable equations which will cause the set-approach for solving algebraic equations by radicals. The mathematician Augustin Louis Cauchy proves that the eigenvalues of a symmetric matrices are all real. The mathematician Johann Peter Gustav Lejeune Dirichlet studies the convergence of Fourier series.

\textbf{+1828}\\
The physicist George Green proves Green's theorem.

\textbf{+1827}\\
The botanist Robert Brown discovered the Brownian motion of pollen particles and dye in water; the same year the physicist Georg Ohm establishes the law of electrical resistance and the physicist and mathematician André Ampère discovered the laws that bind the magnetic forces to electric current. The physicist, mathematician and astronomer William Rowan Hamilton presents the theory of a single function that unifies mechanics, optics and mathematics and helped to establish the wave theory of light.

\textbf{+1826}\\
The mathematician Niels Henrik Abel proves that it is impossible to solve general quintic equation ($5$th order polynomials) in radicals. In his \textit{On a Method of Expressing by Signs the Action of Machinery}, the mathematician, philosopher, inventor and mechanical engineer Charles Babbage describes of a symbolic language that will help him to design its analytical machine, the first universal mechanical machine. The "Babbage machine" is the first complete programmable computer (having the same computational capabilities as a Turing machine or a modern computer) that has been designed. 

\textbf{+1825}\\
The mathematician Augustin-Louis Cauchy presents the Cauchy integral theorem for general integration paths and introduces the theory of residues. The scientist and inventor William Sturgeon invented the first electromagnet.

\textbf{+1824}\\
The mathematician Augustin-Louis Cauchy discovers the characteristic polynomial of a matrix and proves that it is invariant by linear transformation and calculates for the first time eigenvalues and eigenvectors.

\textbf{+1824}\\
The physicist and engineer Sadi Carnot scientifically analyzes the efficiency of steam engines (Carnot cycle), showing that their performance is limited and also defines the second principle of thermodynamics.

\textbf{+1823}\\
The physicist and chemist Michael Faraday presents of a series of papers on the liquefaction of gases.  The mathematician Pierre Frédéric Sarrus introduces the vertical bar symbol for integral: $\int_a^bf(x)\mathrm{d}x=F(x)|_a^b$.

\textbf{+1822}\\
The mathematician Jean-Victor Poncelet found the projective geometry. The same year, the physicist and mathematician Joseph Fourier formally presents the use of dimensions (units) for physical quantities and introduce Fourier series and also the notation $\int_a^bf(x)\mathrm{d}x$.

\textbf{+1821}\\
The principle of the dynamo is described for the first time by the physicist and chemist Michael Faraday. The same year the physicist John Herapath proposes that heat is in reality the effect of agitation and therefore the movement of elementary bodies.

\textbf{+1820}\\
The physicist Hans Oersted discovers and proves the magnetic effects of electric current. The same year, physicists Jean-Baptiste Biot and Félix Savart determine in the field of magnetism the famous law that bears their name.

\textbf{+1819}\\
The physicist and chemist Hans Christian Örsted shows that electric current deflected a magnetized needle, thus demonstrating electromagnetism and announcing an industrial revolution.

\textbf{+1818}\\
The mathematician, geometer and physicist Simeon Poisson calculates the Poisson bright spot at the center of the shadow of a circular opaque obstacle.

\textbf{+1817}\\
By studying the polarization of light, the physicist Augustin Fresnel shows that it is a transversal wave motion and not longitudinal and also shows that the diffraction and interference can be explained if we consider light as a wave. The same year, the astronomer Friedrich Bessel publishes the works making use of the famous functions that bear his name.

\textbf{+1816}\\
The mathematician Joseph Diaz Gergonne introduces the symbol marking the inclusion in the set theory.

\textbf{+1814}\\
The physicist and optician Joseph von Fraunhofer studied for the first time the absorption lines of the solar spectrum and this using the spectroscope which he was the inventor. The mathematician, astronomer and physicist Pierre-Simon Laplace makes the assumption that a perfect knowledge of the present state of the universe would enable one to determine perfectly all its future states.

\textbf{+1812}\\
The mathematician, astronomer and physicist Pierre-Simon Laplace published a major work on probability theory (including also the method of least squares) for which he is considered as one of the founders.

\textbf{+1811}\\
The chemist Amaedo Avogadro hypothesizes that equal volumes of different gases contain the same number of molecules under the same conditions of temperature and pressure.

\textbf{+1810}\\
The mathematician, astronomer and physicist Carl Friedrich Gauss discovered the basic concepts of non-Euclidean geometry but never published his work on the subject. The same year, the physicist and mathematician Joseph Fourier models the evolution of the temperature with trigonometric series.

\textbf{+1809}\\
The mathematician, astronomer and physicist Carl Friedrich Gauss developed the method of least squares independently of Legendre. The same year, the mathematician, astronomer and physicist Pierre-Simon Laplace proved the general form of the central limit theorem. The engineer, physicist and mathematician Etienne Malus publishes the law of Malus.

\textbf{+1808}\\
The physicist and chemist John Dalton proposes what is considered as the first theory of the atom. The mathematician Christian Kramp introduces in his \textit{Éléments d'arithmétique universelle} the notation $n!$ for the factorial.

\textbf{+1806}\\
The mathematician Jean Robert Argand published the first plane representation of complex numbers and algebraic measures are used. The mathematician Johann Carl Friedrich Gauss develops the idea of adding vectors in a geometric form and introduces the notation $\ overrightarrow{ab}$ for a vector.

\textbf{+1805}\\
The mathematician André-Marie Legendre developed the method of least squares.

\textbf{+1803}\\
The physicist and chemist John Dalton has the original idea to assume that each chemical element is composed of different atoms. A chemical combination was then explained by the union of these atoms in fixed proportions and relative atomic masses became calculable from experimental facts. The same year, the economist, journalist and industrialist Jean-Baptiste Say published his Treatise on Political Economy.

\textbf{+1802}\\
The physicist Thomas Young showed the wave nature of light by an important experience that shows the interference of waves. The same year, the chemist and physicist Joseph Louis Gay-Lussac discovered the famous law connecting gas volume and temperature of a real gas, the law that bears his name (Gay-Lussac's law).

\textbf{+1801}\\
The chemist and physicist John Dalton discovered the law of the sum of the partial pressures which still bears his name.

\textbf{+1800}\\
The chemist William Nicholson and the surgeon Anthony Carlisle used electrolysis to separate water into hydrogen and oxygen. The same year, the astronomer William Herschel discovers infrared radiation and the physicist Alessandro Volta and invented the first electric battery.

\textbf{+1799}\\
The mathematician Gaspard Monge published his book of descriptive geometry. He is considered as the inventor of this field. First satisfactory but incomplete proofs of the fundamental theorem of algebra by the mathematician Johann Carl Friedrich Gauss.

\textbf{+1798}\\
An Essay on the Principle of Population by the cleric and scholar Thomas Robert Malthus (one of the founding books of the study of population dynamics) proposes one of the first models of population growth, the exponential model, which will be the basis for future work.

\textbf{+1798}\\
The mathematician Carl Friedrich Gauss gives a rigorous proof of the theorem of d'Alembert (fundamental theorem of algebra). The same year, the physicist Benjamin Thompson had the idea that heat is a form of energy and the physicist and chemist Henry Cavendish measures the gravitational constant. The economist Thomas Malthus stated his law of population.

\textbf{+1797}\\
The mathematician Caspar Wessel associates vectors with complex numbers and studies complex number operations in geometrical terms.

\textbf{+1793}\\
The National Assembly of the French Republic established the metric system.

\textbf{+1789}\\
The physicists and chemists Antoine Lavoisier states the law of conservation of mass.

\textbf{+1788}\\
In his \textit{Méchanique Analitique}, the mathematician and Joseph-Louis Lagrange formulates a new way to study classical mechanics (by Isaac Newton for recall) using the principle of least action. The same year, the Academy of Science approves the creation of a universal measurement system, the future metric system. This project will also be approved by the French National Assembly in 1790, which will give the first definition of the meter.

\textbf{+1787}\\
The physicist, chemist and inventor Jacques Alexandre César Charles experimentally determined that the volume of a fixed mass of gas at constant pressure is proportional to temperature.

\textbf{+1786}\\
The astronomer William Herschel made a detailed description of our galaxy.

\textbf{+1785}\\
The physicist Charles Augustin Coulomb proves that the forces between electric charges and between magnets applied at the inverse square of the distance.

\textbf{+1783}\\
The clergyman and natural philosopher John Michell in a paper for the Philosophical Transactions of the Royal Society of London, read on 27 November 1783, first proposed the idea that there were such things as black holes, which he called "dark stars". A few years after Michell came up with the concept of black holes, the French mathematician Pierre-Simon Laplace suggested essentially the same idea in his 1796 book, \textit{Exposition du Système du Monde}. 

\textbf{+1782}\\
The mathematician, physicist and astronomer Pierre-Simon de Laplace introduces the "Laplace transform", a transformation that solves several differential equations in physics.

\textbf{+1781}\\
The chemist and physicist Joseph Priestley creates water by combustion of hydrogen and oxygen which shows that water is not an essential element as we thought since Aristotle.

\textbf{+1778}\\
The physicists and chemists Carl Scheele and Antoine Lavoisier discovered that air is composed mainly of nitrogen and oxygen.

\textbf{+1777}\\
The physicist and mathematician Leonhard Euler re-introduces the letter $\mathrm{i}$ for the imaginary part of the complex numbers.

\textbf{+1776}\\
The moral philosopher and a pioneer of political economy Adam Smith publishes his An Inquiry into the Nature and Causes of the Wealth of Nations.

\textbf{+1774}\\
The mathematician, astronomer and physicist Pierre-Simon Laplace explicitls Euler's integral. The same year, the theologian, dissenting clergyman, natural philosopher, educator and political theorist Joseph Priestley made his major discovery, that of oxygen.

\textbf{+1772}\\
The mathematician, engineer and astronomer Joseph-Louis Lagrange studied the three-body problem and discovered the libration points today named "Lagrange points".

\textbf{+1770}\\
In his \textit{Mémoire sur les éequations aux différences partielles}, the philosopher, mathematician, and early political scientis Marie Jean Antoine Nicolas de Caritat, Marquis of Condorcet introduces the symbol of partial derivatives $\partial$.

\textbf{+1769}\\
The physicist and chemist Henry Cavendish discovered and studied hydrogen. The same year, in his \textit{Institutiones calculi integralis}, the mathematician Leonhard Euler studies for the first time the double integrals, calculates them by successive integration and changes of variable. These methods will be generalized to the triple integrals by the mathematician, engineer and astronomer Joseph-Louis Lagrange, which also gives the general formula for the change of variables (determinant of the Jacobian).

\textbf{+1766}\\
The physicist and chemist Henry Cavendish discovered and studied hydrogen.

\textbf{+1764}\\
The latent and specific heat are described for the first time by the physicist and chemist Joseph Black. It is also the first to clearly distinguish temperature and momentum. The same year, the physicist and mathematician Leonhard Euler examines the partial differential equation for the vibration of a circular drum and finds one of the Bessel function solutions.

\textbf{+1763}\\
A posthumous article of the mathematician and clergyman Thomas Bayes reveals that he discovered what is named today the "Bayes' theorem".

\textbf{+1757}
The physicist and mathematician Leonhard Euler founds modern hydrodynamics.

\textbf{+1756}\\
The mathematician, engineer and astronomer Joseph-Louis Lagrange develops analytical mechanics based on his invention of the calculus of variations independently of Leonhard Euler.

\textbf{+1755}\\
The physicist and mathematician Leonhard Euler introduces the uppercase Greek letter sigma ($\sum$) for the symbol of the sum.

\textbf{+1753}\\
In his 1749 study of the motions of the earth Leonhard Euler obtained differential equations for the orbital elements and in 1753 he applied the method of variation of constants to his study of the motions of the moon.

\textbf{+1750}\\
The mathematician Gabriel Cramer define the Cramer's rule for solving linear systems.

\textbf{+1749}\\
The astronomer and physicist Jean le Rond D'Alembert developed the first model of precession based on the theory of gravitation of Newton and gives a possible solution to the problem of three bodies.

\textbf{+1748}\\
In his \textit{Introductio in Analysin Infinitorum} the mathematician Leonhard Euler introduces the concept of function (defined as any composition of algebraic and analytic expression), defines the concepts of even and odd functions, popularize the use of the symbols $e$ and $\pi$, proves Euler's identity and defines the $\Gamma$ function that generalized the factorial.

\textbf{+1746}\\
The encyclopedist Jean le Rond D'Alembert gives the first evidence (acceptable but will be corrected later) of the fundamental theorem of algebra. The following year (1747) he published the equation of vibrating strings, which was the first example of the wave equation. This makes D'Alembert, one of the founders of mathematical physics.

\textbf{+1744}\\
The philosopher, mathematician, physicist, astronomer and naturalist Pierre Louis Moreau de Maupertuis states the principle of least action which will be formalized mathematically 22 years later by the mathematician, engineer and astronomer Joseph-Louis Lagrange. The same year, the physicist and mathematician Leonhard Euler shows the existence of transcendental numbers and introduces variations calculus.

\textbf{+1742}\\
The astronomer Anders Celsius defines its own unit of measurement for temperature.

\textbf{+1739}\\
The physicist and mathematician Leonhard Euler solves the general homogeneous linear ordinary differential equation with constant coefficients.

\textbf{+1738}\\
The physician, physicist and mathematician Daniel Bernoulli published a book on hydrodynamics introducing the kinetic theory of gases and the famous Bernoulli theorem (pressure balance). The same year in his \textit{Doctrine of chance}, the mathematician Abraham De Moivre introduces the Gaussian distribution as a means of approximating the binomial law for large number of experiments and demonstrates a partial version of the central limit theorem.

\textbf{+1737}\\
The physicist and mathematician Leonhard Euler solves the problem of graph theory on the bridges of Königsberg. The resolution of this problem is considered as the first theorem of graph theory. He establishes the same time the "Euler's formula" linking the number of vertices, edges and faces of a convex polyhedron, and hence of a planar graph.

\textbf{+1736}\\
The inventor Jonathan Hulls puts the first patent for a boat propelled by a steam engine.

\textbf{+1734}\\
The physicist and mathematician Leonhard Euler introduces the notation $f(x)$ for a function applied to the argument $x$.

\textbf{+1733}\\
The mathematicien Geralamo Saccheri studies what geometry would be like if Euclid's fifth postulate were false.

\textbf{+1729}\\
The dyer and amateur of physics and astronomy Stephen Gray was the first to discover the transmission of electricity in materials that he named "conductors".

\textbf{+1727}\\
The physicist and mathematician Leonhard Euler introduces the modern notation for the trigonometric functions and the letter $e$ for the base of the natural logarithm (occasionally also known as the "Euler number").

\textbf{+1724}\\
The mathematician Abraham De Moivre studies mortality statistics and the foundation of the theory of annuities in Annuities on lives.

\textbf{+1715}\\
The mathematician Brook Taylor publishes the tools that gives the possibility to make integration by parts and series expansions of functions (the famous Taylor series).

\textbf{+1714}\\
The mathematician Brook Taylor derives the fundamental frequency of a stretched vibrating string in terms of its tension and mass per unit length by solving an ordinary differential equation.

\textbf{+1713}\\
The mathematician and physicist Jacques Bernoulli publishes the rigorous principles of basic probabilities and statistics.

\textbf{+1705}\\
The astronomer Edmund (or Edmond) Halley predicted with an almost negligible calculation error with that the comet passed near the Earth in 1682 will return in 1758.

\textbf{+1704}\\
The physicist and mathematician Isaac Newton found experimentally that white light is composed of many colors. It also assumes that a light ray is composed of particles.

\textbf{+1701}\\
In his \textit{Explication de l'Arithmétique Binaire} Gottfried Wilhelm Leibniz introduces binary arithmetic (Leibniz may have been the first computer scientist and information theorist). He anticipated Lagrangian interpolation and algorithmic information theory. His \textit{Calculus ratiocinator} anticipated aspects of the universal Turing machine. In 1961, the mathematician and philosopher Norbert Wiener suggested that Leibniz should be considered the patron saint of cybernetics.

\textbf{+1698}\\
The mathematician and physicist Jacques Bernoulli clearly poses the problem of the brachistochrone curve (which belongs to the family of cycloid curves) and proposes a solution. The same year, the mathematician Guillaume de L'Hopital states his rule for the examination of indeterminate forms. Gottfried Leibniz proposes the use of the point $\cdot$ to denote multiplication, instead of the cross $\times$, which is too easily confused with the variable $x$ in the equations.

\textbf{+1696}\\
The mathematician and physicist Jacques Bernoulli clearly poses the problem of the brachistochrone curve (which belongs to the family of cycloid curves) and proposes a solution. The same year, the mathematician Guillaume de L'Hopital states his rule for the examination of indeterminate forms.

\textbf{+1693}\\
The astronomer and engineer Edmund Halley discovered the relation between the focal length of a lens with the distance of the image to its axis and the real object to its axis. The same year, he prepares the first mortality tables statistically relating death rate to age.

\textbf{+1691}\\
The philosopher and mathematician Gottfried Leibniz discovers the technique of separation of variables for ordinary differential equations.

\textbf{+1690}\\
The wave theory of light is put forward by the physicist and astronomer Christiaan Huygens. The same year the physicist and mathematician Johann Bernoulli developed the exponential calculus and find the equation of the catenary. The same year, the mathematician and physicist Jacques Bernoulli (brother of Jean Bernoulli) develops integral calculus.

\textbf{+1687}\\
The physicist and mathematician Isaac Newton published \textit{Naturalis Principia Mathematica} in which he explains the force of gravity and planetary orbits. He also describes the three laws of dynamics and derivates Kepler laws for his gravitational law. This is the first scientific revolution (before Special/General Relativity and quantum physics).

\textbf{+1685}\\
The philosopher and mathematician Gottfried Leibniz solves linear systems using without theoretical justification matrices and determinants.

\textbf{+1682}\\
The physicist and mathematician Isaac Newton establishes the law of gravitation, which now bears his name.

\textbf{+1679}\\
The philosopher and mathematician Gottfried Leibniz introduces binary arithmetic and develops a calculating machine that performs 4 operations. The same year, the physicist, mathematician and inventor Denis Papin shows experimentally the influence of atmospheric pressure on the boiling point of water.

\textbf{+1678}\\
The mathematicien, astronomer and physicist Christiaan Huygens states his principle of wavefront sources.

\textbf{+1676}\\
The physicist Robert Hooke states that stretching a spring is proportional to the voltage.

\textbf{+1675}\\
The astronomer Olaus Roemer makes accurate measurements of the speed of light. The same year, tThe apothecary chemist Nicolas Lemery writes \textit{Cours de chymie} which is considered as the first great chemistry treatise where the mixtures are defined, the first theory of bases and acids, and so on are introduced.

\textbf{+1673}\\
The philosopher and mathematician Gottfried Leibniz invents his differential calculus, introduces the symbol $\int$ and use the term "convergence test" for alternated series and use the definition of the convergence for the first time. Leibniz also use the notation $\mathrm{d}y/\mathrm{d}x$ and $\mathrm{d}x$, $\mathrm{d}x$.

\textbf{+1671}\\
First attempt to calculate life annuities (comparable to life insurance) by the politician Johan de Witt in collaboration with the mathematician Christian Huygens and first calculations of life expectancy.

\textbf{+1670}\\
The mathematician John Wallis introduces the symbols $\le$ and $\ge$.

\textbf{+1669}\\
The mathematician, astronomer and physicist Christian Huygens published results on the observation of the conservation of kinetic energy becoming verbatim the discoverer of the concept of kinetic energy. The same year, in his manuscript \textit{De analysi per aequationes numero terminorum Infinitas}, the physicist and astronomer Isaac Newton gives the first description of the Newton's method which makes it possible to find approximations of function roots by iterated process. The description of Newton only applies to polynomials and does not use the notion of derivative (the manuscript will be published in 1711).

\textbf{+1668}\\
The physicist and astronomer Isaac Newton made the first reflecting telescope and the same year the mathematician John Wallis suggests the law of conservation of momentum.

\textbf{+1667}\\
In his \textit{Vera Circuli and Hyperbolae Quadratura} the mathematician and astronomer James Gregory gives the first proof of the fundamental theorem of calculus and independently discovers Taylor series. Draft of the concept of transcendental number elaborated in relation to the problem of the quadrature of the circle.

\textbf{+1665}\\
The physicist Isaac Newton formulated the three laws of mechanics. He lays the foundations of differential calculus, these techniques allows him starting from the expression of a force inverse of square of the distance to find the general form of Kepler's laws.

\textbf{+1664}\\
The physicist Isaac Newton begins to work on differential and integral calculus.

\textbf{+1661}\\
The founder of statistical demography John Graunt published the first mortality table, the same year the physicist and chemist Robert Boyle determines the laws of compressibility of gas bearing his name and sometimes attached to that of the physicist Edme Mariotte who rediscovered a few years after the same laws.

\textbf{+1659}\\
The mathematician, astronomer and physicist Christian Huygens discovered the rigorous isochronism formula (when the end of the pendulum travels an arc of cycloid, the period of oscillation is constant regardless of the amplitude).

\textbf{+1658}\\
The mathematician, un astronome and physicist Christian Huygens experimentally discovers that balls placed anywhere inside an inverted cycloid reach the lowest point of the cycloid in the same time and thereby experimentally shows that the cycloid is the isochrone.

\textbf{+1657}\\
The lawyer and mathematician Pierre de Fermat states his "Fermat's principle" in optics as how the light propagates from one point to another on trajectories such that the duration of the propagation is locally minimal.

\textbf{+1655}\\
The mathematician, astronomer and physicist Christian Huygens was the first to use the concept of expected mean in probabilities. The mathematician John Wallis introduces the symbol $\infty$ in his \textit{Mathesis Universalis}.

\textbf{+1654}\\
The mathematician, physicist, inventor, philosopher, moralist and theologian Blaise Pascal and the lawyer and mathematician Pierre de Fermat create the theory of probability.

\textbf{+1644}\\
The physicist and mathematician Evangelista Torricelli has the idea of substituting water by mercury in the so named Torricelli's experiment to highlight the "grosso-vido"; later will the works of the mathematician, physicist, inventor, philosopher, moralist and theologian Pascal Blaise follow (experience Puy de Dôme experiment).

\textbf{+1638}\\
The mathematician, geometer, astronomer and physicist Galileo Galilei publishes the mathematical relationship that defines the period of the simple pendulum.

\textbf{+1637}\\
The philosopher and mathematician René Descartes renames the unknowns $x$, $y$, $z$ and the parameters $a$, $b$, $c$ and extends the use of algebra to the lengths and plane, creating analytical geometry with Pierre de Fermat. The same year, always René Descartes, quantitatively derives the angles at which primary and secondary rainbows are seen with respect to the angle of the Sun's elevation.

\textbf{+1631}\\
The mathematician Thomas Harriot introduces, in a posthumous publication, the symbols $>$ and $<$. The same year the theologian and mathematician William Oughtred provides for the first time the multiplication symbol $\times$ and the symbol $\pm$.

\textbf{+1629}\\
The lawyer and mathematician Pierre de Fermat develops a rudimentary differential calculus. The same year, in his Invention nouvelle en algèbre Albert Girard states, without proof, for the first time the fundamental theorem of algebra (a polynomial of degree $n$ has $n$ complex distinct or non-distinct roots) using complex numbers.

\textbf{+1626}\\
Tables of sine, tangent and secant of the engineer Albert Girard with the use of abbreviations sin, cos and tan.

\textbf{+1624}\\
Invention of the first thermometer (whose graduations are obviously not standardized...) by the physician Santorio Santorio.

\textbf{+1621}\\
The astronomer and physicist Willebrord Snell discovered that the angle of refraction of light is determined by the sine of the angle of the incident light with the normal of the dioptre.

\textbf{+1620}\\
The engineer Francis Thomas Bacon defends the experimental method and leads numerous observations on the heat. He suggested that the heat is related to the movement.

\textbf{+1619}\\
Johannes Kepler finished to publish the three laws of planetary motion.

\textbf{+1614}\\
The mathematician John Napier (John Napier in French) invented logarithms, which bring the operations of multiplication and division to simple additions or subtractions.

\textbf{+1611}\\
Johannes Kepler discovers total internal reflection, a small angle refraction law and thin lens optics.

\textbf{+1610}\\ In his \textit{Sidereus Nuncius}, Galileo Galileo reports the first observations at the telescope: the discovery of the satellites of Jupiter, the confirmation that the milky way is constituted of stars, the discovery of the rings of Saturn. These
Observations will have an important effect because they contradict some of the ideas of the Universe models of the time and the Bible writings.

\textbf{+1609}\\
In his \textit{Astronomia Nova}, the astronomer and mathematician Johannes Kepler explains Kepler's first two laws for the motion of the planets.

\textbf{+1608}\\
The optician Hans Lippershey invented the telescope to will be used and improved (with a random quality) the following year by the mathematician, geometer, astronomer and physicist Galileo Galilei to confirm the theories of Copernicus.

\textbf{+1604}\\
In a letter to Paolo Sarpi, Galileo states the law of the fall of bodies: the distance traveled is proportional to the square of the time of fall.

\textbf{+1603}\\
The mathematician and astronomer Thomas Harriot determines how to calculate qualitatively the surface of a spherical triangle.

\textbf{+1591}\\
The mathematician François Vieta opens a new period int algebra by making calculations with letters, using vowels for unknowns and consonants for parameters. Moreover, he gives the development of the Newton binomial theorem.

\textbf{+1590}\\
The astronomer Galileo Galilei demonstrated experimentally that all falling bodies have the same acceleration. The same year, the opticians Hans and Zacheraius Janssen created the first microscope by combining several lenses that define the beginnings of scientific medicine and biology.

\textbf{+1586}\\
The engineer and physicist Simon Stevin proved the method of the parallelogram of forces and discovered that the pressure of a liquid on the bottom of a container is independent of its shape, and also of the bottom surface and depends only on the height water in the container. He also gave the pressure measurement on any portion of the side of a container.

\textbf{+1576}\\
The astronomer Tycho Brahe observed a new star in the constellation of Cassiopeia and built an observatory on the island of Hveen.

\textbf{+1572}\\
The mathematician Rafaelle Bombelli gives a formulation of complex numbers and the rules of actual calculations. He introduces the terms più di meno (pdm) and meno di meno (mdm) to represent $+\mathrm{i}$ and $-\mathrm{i}$.

\textbf{+1548}\\
The mathematician and physicist Simon Stevin wrote the tenth powers identified with an exponent. He also gives the first writing of vectors. 

\textbf{+1545}\\
The mathematician Ludovico Ferrari gives the solution of equations of degree $4$ (know under "Cardan formula").

\textbf{+1543}\\
The work of the astronomer Nicolas Copernicus summarizing 26 years of research and observations is published and clearly highlights that the heliocentric system of Ptolemy is not valid.

\textbf{+1536}\\
The mathematician Niccolò Fontana launched the new science of ballistics.

\textbf{+1530}\\
The mathematician and physicist Robert Recorde introduces the "$=$" sign and the mathematician Michael Stifel developed an early form of algebraic notation.

\textbf{+1525}\\
The mathematician Christoff Rudolff introduces the notation for square roots $\sqrt{\phantom{a}}$.

\textbf{+1515}\\
Publication of Scipione del Ferro's works where he find a formula giving the general solution of the polynomial equations of degree $3$. These solutions involve the implicit manipulation of imaginary numbers

\textbf{+1510}\\
The painter, engraver and mathematician Albrecht Dürer develops the basics of descriptive geometry and perspective.

\textbf{+1500}\\
The Italian mathematician Scipione del Ferro succeeds for the first time to solve a large algebraic type of cubic equations.

\textbf{+1490}\\
The painter, sculptor, architect, musician, mathematician, engineer, inventor, anatomist, geologist, cartographer, botanist, and writer... Leonardo da Vinci describes capillary action.

\textbf{+1489}\\
\textit{Behende und hupsche Rechnung auf allen kauffmanschafft} of Johannes Widmannwe the symbols "$+$" and "$-$" are introduced for the first time (before plus was denoted "P" and minus "M").

\textbf{+1464}\\
\textit{Triparty in the science of numbers} by  Nicolas Chuquet we we can find the first use of negative powers and zero power - states the property of the exponents we still use $x^{n+m}=x^nx^m$.

\textbf{+1420}\\
The mathematician and astronomer Jamshid al-Kashi computed and observed the solar eclipses of 1406, 1407 and 1408. He is also the first to use decimal notation in arithmetic and in arabic numerals.

\textbf{+1400}\\
The Mathematician and astronomer Jamshid al-Kashi developed an early form of Newton's Regula falsi method.

\textbf{+1200-1400}\\
Madhava and the Kerala School (India) discover several infinite series for numbers like $\pi$ and specific values of trigonometric functions - these works ahead of those of the Europeans on the differential and integral calculus and the series of powers.

\textbf{+1350}\\
\textit{Tractatus of configurationibus qualitatum and motuum} of Oresme is draft of geometry using coordinates, and use axes for different sizes, which is an important step in the transition from qualitative science based mainly on Aristotle to quantitative science. Proof of the mean velocity theorem, which anticipates the results of Gallileo on the uniform rectilinear motion and the bodies in free fall by linking area under the curve of the velocity to the position in a graph.

\textbf{+1303}\\
\textit{Siyuan Yujian }(translation:\textit{ Precious mirror of for elements}) of Zhu Sjijie describes the elimination method for solving systems of equations containing up to four unknowns and up to degree 14 for some form of equations. We also find there the definition of the Pascal's triangle and the formulas of summations for some series.

\textbf{+1300}\\
Raymon Lulle developed a geometric (useless) machine to automate theist logic. This idea will influence Leibniz in his research of a universal language for reasoning, research that will lead him to take an interest in Chinese writing and binary arithmetic. Lulle's ideas anticipate modern ideas of formal deduction systems.

\textbf{+1269}\\
The scholar Pierre de Maricourt coined the expressions of the magnetic "north" and "south poles" and he was the first who wrote that opposite poles attract each other

\textbf{+1268}\\
The philosopher, scientist and alchemist Roger Bacon publishes proposals to reform school, arguing that to study nature, the use of observations of the measures is the only basis of rigorous testing and verification while affirming at the same time the need of mathematics for this purpose.

\textbf{+1200}\\
The mathematician Jordan Nemore introduces the notation for unknowns with symbols.

\textbf{+1150}\\
Creation of modern notation for fractions (horizontal bar) by Al-Hassãr. At the same period, we have the Latin translation of the 820 treaty by Al-Khwarizmi on the Indian calculation which allows the decimal system and the use of zero to spread in Europe and also Gérard de Cremona publishes a translation in Latin of the Arabic version of the Almagest of Ptolemée, the name "sinus" comes from this translation...

\textbf{+1121}\\
The astronomer, physicist, biologist, chemist, mathematician and philosopher Abu al-Fath Khazini published a book in which he proposed that gravity and gravitational potential energy vary with distance from the center of the Earth. He also makes a distinction between force, mass and weight. He also invented several scientific instruments, including a steelyard and hydrostatic balance. He also introduces experimental scientific methods to static and dynamic, unifies them in the science of mechanics and combines hydrostatics with the dynamic to create hydrodynamics.

\textbf{+1114}\\
The mathematician Bhaskara provides a comprehensive summary of Hindu mathematics, as developed from the 5th to the 7th century AD. He also recognizes the negative square root, solves quadratic equations with several unknowns, equations of higher order such as Fermat and the general quadratic equations. He was also a pioneer in the principle of differential calculus nearly 500 years before Newton and Leibniz.

\textbf{+1100}\\
The philosopher and physicist Allah Abu'l-Barakat Hibat al-Baghdaadi is the first to deny Aristotle's idea that a constant force produces uniform motion what prepares the Newton's second law of motion. Like Newton, he described acceleration as the variation of speed.

\textbf{+1037}\\
The mathematician, physicist and philosopher Ibn al-Haytham is aware of the magnitude of the acceleration due to gravity. He discovers the law of inertia, known today as the first law of motion Newton.

\textbf{+1030}\\
The philosopher, writer, physician and scientist Abu Ali al-Husayn ibn Abd Allah Ibn Sina (known in Occident as Avicenna) note that if the perception of light is due to the emission of some sort of particles by a light source, the speed of light has to be finished. He also provided a sophisticated explanation for the phenomenon of rainbow. The mathematician, astronomer, physicist, scholar, encyclopedist, philosopher, astrologer, traveler, historian, pharmacologist Ab? al-Rayhan Muhammad ibn Ahmad al-Biruni, and later the astronomer Abu al'Fath Khazini, were the first to apply scientific methods in experimental mechanics, especially in the fields of statics and dynamics, to determine the specific weight, such as those based on the theory of balances and weighting.

\textbf{+1021}\\
The philosopher, mathematician and physicist Ibn al-Haytham is considered the father of optics and a pioneer of the scientific method describes correctly the light and vision, and introduces the experimental scientific method, laying the foundations of experimental physics. He also discusses experimental psychology and describes various optical instruments such as the darkroom. He wwas able to estimate atmosphere width with an accuracy of $1$ [km], he defined the inertia principia (first Newton's law), the linear momentum and he calculated $\sum_{k=1}^n k^4=\frac{n(2n+1)(n+1)(3n^2+2n-1)}{30}$.

\textbf{+1019}\\
The mathematician, astronomer and physicist Abu Rayhan Al-Biruni observed and described the solar eclipse of April 8, 1019, and the lunar eclipse of September 17, 1019, in detail; he gave the exact location of the stars during the lunar eclipse. He invented the astrolabe and the planisphere.

\textbf{+1010}\\
The mathematician Al-Sijistani Zuraqi invented a astrolabe designed for a single heliocentric planetary model in which the Earth is moving, rather than the sky.

\textbf{+1000}\\
The mathematician, physicist and astronomer Abu Sahl al-Qouhi discovers that the weight of bodies varies with their distance from the center of the Earth, and solves equations higher than the second degree. During the same decade, the mathematician and engineer Al'Karkhi wrote a book containing the first known proof by mathematical induction. He uses it to prove the binomial theorem, the Pascal's triangle, and the sum of the cubes integrals.

\textbf{+996}\\
The mechanical oriented astrolabe, with 8 gears is invented by the mathematician, astronomer and physicist Abū Rayhān Al-Biruni who is also the author of works on the summation of series and combinatorics.

\textbf{+980}\\
Abitu al-Wafaa do the first calculation of values of trigonometric functions and publish the sinus law adapted for triangles on sphere. He also proved by induction $\sum_{k=0}^n k^3=\frac{n^2(n+1)^2}{4}$.

\textbf{+964}\\
The mathematician, physicist and philosopher Abd al-Rahman al-Sufi explains the magnifying power of lenses and was the first to use a scientific method of analysis that will greatly influence future scientists.

\textbf{+953}\\
The engineer and mathematician Al-Karkhi defines different monomials and gives rules for products of any two of them. He also discovered the binomial theorem for integer exponents.

\textbf{+952}\\
The mathematician Abu'l-Hasan al-Uqlidisi modifies the calculation methods for the numerical Indian system to make it possible for feathers and paper usage. Until then, do calculations with Indian numerals necessitated the use of a board.

\textbf{+900}\\
The first reference to a viewing tube can be found in the work of the astronomer and mathematician Al-Battani, and the first accurate description of the observation tube was given by the mathematician, astronomer, physicist, scholar, encyclopedist, philosopher, astrologer, traveler, historian, pharmacologist Al-Biruni, in a section of his work dedicated to verify the presence of the new crescent moon at the horizon. Although these preliminary observations tubes do not have lenses, they allow an observer to focus on a part of the sky by eliminating light interference. These observation tubes were later adopted in Europe, where they influenced the development of the telescope.

\textbf{+880}\\
The astronomer and mathematician Al-Battani discovered the motion of the apogee of the Sun, calculate the values of the precession of the equinoxes and the inclination of the Earth's axis. He is at the origin of the definition of the tangent and cotangent trigonometric functions.

\textbf{+820}\\
The word "algebra" appears. The mathematician, geographer, astrologer and astronomer Muhammad ibn Musa Al'Khwarizmi is often regarded as the father of medieval algebra because he releases it from geometry. He is also the origin of the quadrant, the mural instrument, the sinus quadrant that was used to solve trigonometric problems and make astronomical observations.

\textbf{+800}\\
Astronomers invent the universal sundial and universal time dial in Baghdad.

\textbf{+780}\\
The alchemist Jabir Ibn Hayyan introduces the experimental scientific method for chemistry and also laboratory equipment such as still and processes such as pure distillation, liquefaction, crystallisation and filtration. He also invented more than 20 types of laboratory equipment, which resulted in the discovery of several chemicals. He also developed recipes for colored glass.

\textbf{+773}\\
Arabic numerals (adapted from India) made their first apparition in Europe.

\textbf{+628}\\
The mathematician Brahmagupta gives rules for solving linear and quadratic equations. He discovers that the quadratic equations have two roots: the negative one and the irrational and give the modern form of the solution we know today. He also give the rules to calculates with negative signs (arithmetic of negative numbers).

\textbf{+550}\\
Hindu mathematicians give zero a numeral representation in a positional notation system.

\textbf{+499}\\
The mathematician Âryabhat gets the full number of solutions of a system of linear equations by methods equivalent to modern methods, and describes the general solution of such equations. He also provides solutions of differential equations. He also asserts that the moon and celestial objects moon other than the stars reflect sunlight, he correctly explains the causes of lunar and solar eclipses, gives the length of the sidereal year to a few minutes, approximate $\pi$ by $62832/20000$, calculates the Earth's diameter with more precision than Erathostene, described the calculation with the Indian numbering system, gives the oldest sinus table for $24$ angles.

\textbf{+275}\\
The mathematician Diophantus of Alexandria considered as the father of algebra equations studied equations with rational variables (thus including quadratic equations) and Diophantine equations.

\textbf{+195}\\
\textit{Suàn shù shu}(translation: \textit{book on numbers and calculation}) in China is one of the oldest known Chinese mathematical texts that contains among other things, calculations of sums of Geometric progression for interest.

\textbf{+130}\\
\textit{Mathematical Composition} (also know under the name of \textit{Almagest's}) of Ptolemy in Alexandria presents a geometric model of the solar system that attempts to describe the motion of the planets the model is inspired by a geometric idea of Apollonius and uses circles whose center move in circular orbits this model puts the Earth at the center of the solar system but gives a fairly good description of the observed movements of the different stars it will be the dominant model until Copernicus some ideas prefigure the Fourier series that will be introduced in the 19th century.

\textbf{+125}\\
The \textit{Yale Music Papyrus} and the \textit{Michigan Instrumental Papyrus} seems to contain the oldest known examples of musical notation.

\textbf{+121}\\
Year corresponding to the oldest document mentioning the magnetic stone.

\textbf{+120}\\
Star catalog of Zhang Heng which contains 2500 stars. Zhang Heng correctly describes the cause of the eclipses and demonstrates that the moon is spherical.

\textbf{+100}\\
The engineer, mechanician and mathematician Hero of Alexandria rediscovered (after the Chinese) the concept of force. He also invented a system of gears to lift weights using steam power. He gives the first description of the sextant (but did not, however, invented it). His contemporary astronomer Claudius Ptolemy invented the sextant and described the astrolabe (perhaps invented by the astronomer, geographer and mathematician Hipparchus) and studied the refraction and reflection. During the same century the mathematician and philosopher Nicomachus of Gerase defines the even and odd numbers, prime and composite numbers, and perfect numbers. Also during the same century the final version of the \textit{Nine Chapters on Mathematical Art} (almost $1180$ pages), written over ten years by several anonymous Chinese authors contains the first use of negative numbers, chapter 9 uses the Pythagorean theorem, chapter 8 uses matrices and the elimination of Gauss to solve systems of equations (at least 1700 years before Gauss!!!)

\textbf{+80}\\
The scholar Wang Ch'ung made the first magnetic compass on a plate of brass.

\textbf{-87}\\
Year corresponding to the dating of the Antikythera mechanism, considered as the first calculator and the first antique analog gear (thirty!) machine to calculate complexes astronomical positions. The care and skill with which this machine was made, as well as the necessary mechanical and astronomy capacities question the historical knowledge of Greek science before its discovery. Indeed, any object of the same age and same complexity was known in the world and it takes nearly a millennium to see similar mechanisms appear! The physicist, mathematician and engineer Archimedes of Syracuse is the hypothetical creator.

\textbf{-100}\\
The indian text \textit{Anuyoga Dwara Sutra} contains several identities involving square roots and squares that seem to imply some knowledge of the laws of exponents or logarithms. An identity taken from this text in modern notation: $\sqrt{a}\sqrt{\sqrt{a}}=(\sqrt{\sqrt{a}})^3$.

\textbf{-134}\\
The astronomer, geographer, and mathematician Hipparchus of Nicaea discovers the precession of the equinoxes.

\textbf{-150}\\
The astronomer, geographer and mathematician Hipparchus is often referred as the founder of trigonometry and corresponding numerical tables. He calculates the first period of revolution of the Sun around the Earth (but the numerical results are in fact those of the rotation of the Earth around the Sun) and develops the theory of eccentrics and epicycles.

\textbf{-200}\\
During the century, the Chinese had invented the tide gate, the rudder, the principle of the steam engine several hundred years before the occident! During this century, the scientist and engineer Philo of Byzantium wrote treatises on levers, pneumatics, automation, traction and water clocks.

\textbf{-225}\\
The astronomer and mathematician Apollonius of Perga published the first study on conics giving to the ellipse, the parabola and the hyperbola the names we know today. He is also credited for the hypothesis of eccentric orbits to explain the apparent motion of the planets and the speed variation of the Moon.

\textbf{-250}\\
The mathematician, physicist and engineer Archimedes of Syracuse study simple machines such as the lever, the famous screw for pumping water ("Archimedes screw") and discovered the Archimedes's principle explaining buoyancy. In the same decade, the astronomer, geographer, philosopher and mathematician Eratosthenes of Cyrene calculated the diameter of the Earth using a gnomon and its shadow and demonstrated the inclination of the ecliptic, the distance Earth-Moon and Earth-Sun and also created a method to determine prime numbers (Sieve of Eratosthenes).

\textbf{-260}\\
The mathematician, physicist and engineer Archimedes of Syracuse computes to two decimal places using inscribed and cirumscribed polygons and computes the area under a parabolic segment.

\textbf{-281}\\
The astronomer and mathematician Aristarchus of Samos assumed that the Sun is the center of the solar system and uses trigonometry to estimate the radius of the Moon and its distance from the Earth using the Earth's shadow during a lunar eclipse.

\textbf{-300}\\
The mathematician and geometer Euclid published his \textit{Elements}, where he reorganized the entire knowledge of the geometry including logical proofs, the construction of the $5$ Platonic solids. In his \textit{Optica} he noted that the light goes in a straight line and describes the law of reflection. We have also that the conical, elliptical, parabolic and hyperbolic concepts appears in the works of two mathematicians of the ancient Greece, namely Menaech-mus and Appolonius de Perge that also introduces the concept of tangent. The same period Bhagabati Sutra calculates the permutations and combinaisons of order $1$, $2$ and $3$.

\textbf{-310}\\
The scholarly Autolycus of Pitane defines uniform motion as an object that travels an equal distance in a equal amount of time.

\textbf{-370}\\
The philosopher Aristotle develops the logic with a theory of naive proposals, quantities and inferential reasoning.

\textbf{-388}\\
The philosopher and astronomer Heraclides of Pontus assumes the rotation of the Earth itself to explain the apparent motion of stars in the night (but still in a geocentric context) and suggests that each planet is a body like the Earth.

\textbf{-400}\\
The Stoic School develops the composed proposals and the logical connectors: "implies", "and", "or" and the inferences "Modus ponens" and "Modus tollens".

\textbf{-430}\\
The philosopher Democritus of Abdera advance the idea that matter is composed of tiny and indentical particles he name "atoms". In reality it is rather an extension of the ideas of his teacher, the philosopher Leucippus of Miletus developed ten years before. Hippasus, a disciple of Pythagoras, would have given what is probably the first rigorous proof of the irrationality of $\sqrt{2}$. The proof uses a Reductio ad absurdum reasoning to show that the sides of a square are incommensurable with its diagonal. 

\textbf{-500}\\
The philosophers Leucippus and Democritus are the founders of atomism.

\textbf{-540}\\
The philosopher, mathematician Pythagoras studies propositional geometry and vibrating lyre strings. 

\textbf{-600}\\
The philosopher Empedocles of Acragas calls for decomposition of the world into four fundamentals elements: water, earth, air and fire. The same century, the mathematician Thales of Miletus highlights electrostatic by rubbing a piece of amber, predicted an eclipse and develops the geometry of the triangle.

\textbf{-750}\\
Manava Sulbar Sutras of Manava finds irrationality of $\sqrt{2}$ and $\sqrt{61}$ and agrees to use irrational numbers in his calculations.

\textbf{-800}\\
Assyrians use water-clocks and Chinese plot planetary movements for their calendar

\textbf{-1500}\\
Indians develops theory of the $4$ elements (fire, air, water, earth)

\textbf{-1400}\\
The Neolithic peoples of Scotland nowadays build stone models of Plato's five solids (regular polyhedra).

\textbf{-1700}\\
The mathematician Apastamba solves general linear equations and uses Diophantine systems of equations with up to five unknowns. The same century, egyptian mathematicians employ primitive fractions.

\textbf{-1800}\\
Babylonian scribes seek the solution of a quadratic equation.

\textbf{-2000}\\
Babylonian priests do the first records of celestial observations.

\textbf{-2300}\\
Chinese astronomers make the first observations of the sky.

\textbf{-2500}\\
The Mesopotamians imagined a position numbering system composed of symbols whose value is based on their rank within a number.

\textbf{-2600}\\
Oldest known mathematical table where a multiplication table is engraved to calculate areas.

\textbf{-3000}\\
Chinese and Babylonians invented the abacus as first adding machine. Geometric concepts are developed for land surveying (hypotenuse calculus). It is also the period corresponding to the oldest know tool used by Incas to record numbers thanks to knots on a string and also the wall representation of wheels.

\textbf{-3500}\\
Oldest Weather Report Found on Stone in Egypt. The unusual weather patterns described on the slab were the result of a massive volcano explosion at Thera, the present day island of Santorini in the Mediterranean Sea

\textbf{-3700}\\
A Babylonian tablet seems to contain first remarkable trigonometric angles and the evidence that Pythagoras' theorem was already known by Babylonians.

\textbf{-4900}\\
The Goseck circle (German: Sonnenobservatorium Goseck) may be one of the oldest Solar observatories in the world.

\textbf{-5000}\\
The decimal system is used in Ancien Egypt (it seems that the consensus is between -6000 and -3000).

\textbf{-8000}\\
Warren Field is the location of a mesolithic calendar. It includes $12$ pits believed to correlate with phases of the Moon and used as a lunar calendar. It is considered to be the oldest lunar calendar yet found.

\textbf{-5200}\\
Radiocarbon dating of the oldest founded wheels (Ljubljana Marshes wooden Wheel).

\textbf{-8000}\\
Marks one bones or woods are slowly replaced by tokens of various shapes to count.

\textbf{-20000}\\
The Ishango bone is a bone tool, dated to the Upper Paleolithic era. It is a dark brown length of bone, the fibula of a baboon, with a sharp piece of quartz affixed to one end, perhaps for engraving. It was first thought to be a tally stick, as it has a series of what has been interpreted as tally marks carved in three columns running the length of the tool. It has also been suggested that the scratches might have been to create a better grip on the handle or for some other non-mathematical reason.

\textbf{-200000}\\
Claims for the earliest definitive evidence of control of fire by a member of Homo range from 0.2 to 1.7 million years ago. Evidence for the controlled use of fire by Homo erectus, beginning some 400,000 years ago, has wide scholarly support.